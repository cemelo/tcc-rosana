% ------------------------------------------------------------------------
% ------------------------------------------------------------------------
% Trabalho de Conclusão de Curso
% Bacharelado em Psicologia
% Faculdade de Educação
% Universidade Federal de Goiás
% ------------------------------------------------------------------------
% ------------------------------------------------------------------------

% verso e anverso:
%\documentclass[12pt,openright,twoside,a4paper,english,french,spanish]{abntex2}

% apenas verso:	
\documentclass[12pt,oneside,a4paper,english,french,spanish]{abntex2} 


% ---
% PACOTES
% ---

% ---
% Pacotes fundamentais 
% ---
\usepackage{cmap}				% Mapear caracteres especiais no PDF
\usepackage{lmodern}			% Usa a fonte Latin Modern	
%\usepackage{fourier}		
\usepackage[T1]{fontenc}		% Seleção de códigos de fonte.
\usepackage[utf8]{inputenc}		% Determina a codificação utiizada (conversão automática dos acentos)
\usepackage{makeidx}            % Cria o indice
\usepackage{hyperref}  			% Controla a formação do índice
\usepackage{lastpage}			% Usado pela Ficha catalográfica
\usepackage{indentfirst}		% Indenta o primeiro parágrafo de cada seção.
\usepackage{color}				% Controle das cores
%\usepackage{graphicx}			% Inclusão de gráficos
% ---
	
% ---
% Pacotes adicionais, usados apenas no âmbito do Modelo Canônico do abnteX2
% ---
\usepackage{lipsum}				% para geração de dummy text
% ---

% ---
% Pacotes de citações
% ---
\usepackage[brazilian,hyperpageref]{backref}	 % Paginas com as citações na bibl
\usepackage[alf]{abntex2cite}	% Citações padrão ABNT

% --- 
% CONFIGURAÇÕES DE PACOTES
% --- 

% ---
% Configurações do pacote backref
% Usado sem a opção hyperpageref de backref
\renewcommand{\backrefpagesname}{Citado na(s) página(s):~}
% Texto padrão antes do número das páginas
\renewcommand{\backref}{}
% Define os textos da citação
\renewcommand*{\backrefalt}[4]{
	\ifcase #1 %
		%Nenhuma citação no texto.%
	\or
		%Citado na página #2.%
	\else
		%Citado #1 vezes nas páginas #2.%
	\fi}%
% ---

% ---
% Informações de dados para CAPA e FOLHA DE ROSTO
% ---
\titulo{A Psicologia Jurídica Frente ao Abuso Sexual Infantil:
  Intervenções, Possibilidades e Limites}
\autor{Rosana Rabelo}
\local{Goiânia}
\data{Março 2013}
\orientador[Orientadora:]{Juliana de Castro Chaves}
\instituicao{%
  Universidade Federal de Goiás - UFG
  \par
  Faculdade de Educação
  \par
  Bacharelado em Psicologia}
\tipotrabalho{Trabalho de Conclusão de Curso}
% O preambulo deve conter o tipo do trabalho, o objetivo, 
% o nome da instituição e a área de concentração 
\preambulo{Trabalho apresentado à banca examinadora da
  Faculdade de Educação da Universidade Federal de Goiás
  - UFG como exigência parcial para obtenção do grau de
  Bacharel em Psicologia, sob orientação da Prof. Dra.
  Juliana de Castro Chaves.}
% ---

% alterando o aspecto da cor azul
\definecolor{blue}{RGB}{41,5,195}

% informações do PDF
\hypersetup{
     	%pagebackref=true,
		pdftitle={\imprimirtitulo}, 
		pdfauthor={\imprimirautor},
    	pdfsubject={\imprimirpreambulo},
		pdfkeywords={psicologia}{juridica}{intervenções}{abuso}{sexual}{infantil},
	    pdfproducer={LaTeX}, 	% producer of the document
	    pdfcreator={\imprimirautor},
    	colorlinks=false,       		% false: boxed links; true: colored links
    	linkcolor=blue,          	% color of internal links
    	citecolor=blue,        		% color of links to bibliography
    	filecolor=magenta,      		% color of file links
		urlcolor=blue,
		bookmarksdepth=4
}
% --- 

% --- 
% ESPAÇAMENTOS ENTRE LINHAS E PARÁGRAFOS
% --- 

% O tamanho do parágrafo é dado por:
\setlength{\parindent}{1.3cm}

% Controle do espaçamento entre um parágrafo e outro:
\setlength{\parskip}{0.2cm}  % tente também \onelineskip
\makeindex

% ----
% INÍCIO DO DOCUMENTO
% ----
\begin{document}
\frenchspacing 

% ----------------------------------------------------------
% ELEMENTOS PRÉ-TEXTUAIS
% ----------------------------------------------------------
% \pretextual

\imprimircapa
\imprimirfolhaderosto*

% ---
% FICHA CATALOGRÁFICA
% ---

% Isto é um exemplo de Ficha Catalográfica, ou ``Dados internacionais de
% catalogação-na-publicação''. Você pode utilizar este modelo como referência. 
% Porém, provavelmente a biblioteca da sua universidade lhe fornecerá um PDF
% com a ficha catalográfica definitiva após a defesa do trabalho. Quando estiver
% com o documento, salve-o como PDF no diretório do seu projeto e substitua todo
% o conteúdo de implementação deste arquivo pelo comando abaixo:
%
% \begin{fichacatalografica}
%     \includepdf{fig_ficha_catalografica.pdf}
% \end{fichacatalografica}
\begin{fichacatalografica}
	\vspace*{\fill}					% Posição vertical
	\hrule							% Linha horizontal
	\begin{center}					% Minipage Centralizado
	\begin{minipage}[c]{12.5cm}		% Largura
	
	\imprimirautor
	
	\hspace{0.5cm} \imprimirtitulo  / \imprimirautor. --
	\imprimirlocal, \imprimirdata-
	
	\hspace{0.5cm} \pageref{LastPage} p. ; 30 cm.\\
	
	\hspace{0.5cm} \imprimirorientadorRotulo~\imprimirorientador\\
	
	\hspace{0.5cm}
	\parbox[t]{\textwidth}{\imprimirtipotrabalho~--~\imprimirinstituicao,
	\imprimirdata.}\\
	
	\hspace{0.5cm}
		1. Psicologia Jurídica.
		2. Abuso sexual infantil.
    3. Práticas.
		I. Juliana de Castro Chaves.
		II. Universidade Federal de Goiás.
		III. Faculdade de Educação.
		IV. Título\\
	
	\hspace{8.75cm} CDU 02:141:005.7\\
	
	\end{minipage}
	\end{center}
	\hrule
\end{fichacatalografica}
% ---

% ---
% FOLHA DE APROVAÇÃO
% ---

% Isto é um exemplo de Folha de aprovação, elemento obrigatório da NBR
% 14724/2011 (seção 4.2.1.3). Você pode utilizar este modelo até a aprovação
% do trabalho. Após isso, substitua todo o conteúdo deste arquivo por uma
% imagem da página assinada pela banca com o comando abaixo:
%
% \includepdf{folhadeaprovacao_final.pdf}
%
\begin{folhadeaprovacao}

  \begin{center}
    {\ABNTEXchapterfont\large\imprimirautor}

    \vspace*{\fill}\vspace*{\fill}
    {\ABNTEXchapterfont\bfseries\Large\imprimirtitulo}
    \vspace*{\fill}
    
    \hspace{.45\textwidth}
    \begin{minipage}{.5\textwidth}
        \imprimirpreambulo
    \end{minipage}%
    \vspace*{\fill}
  \end{center}
    
  Trabalho aprovado. \imprimirlocal, 05 de março de 2013:

  \assinatura{\textbf{\imprimirorientador} \\ Orientadora}
  \assinatura{\textbf{Liliane Domingos Martins} \\ Avaliadora}
      
  \begin{center}
    \vspace*{0.5cm}
    {\large\imprimirlocal}
    \par
    {\large\imprimirdata}
    \vspace*{1cm}
  \end{center}
  
\end{folhadeaprovacao}
% ---

% ---
% Dedicatória
% ---
\begin{dedicatoria}
    \vspace*{\fill}
    \centering
    \noindent
    \textit{A fazer.} \vspace*{\fill}
\end{dedicatoria}
% ---

% ---
% Agradecimentos
% ---
\begin{agradecimentos}
Agradeço a Deus, Senhor do tempo e da minha vida. Sem Ele esta vitória jamais seria possível. 

À minha família, minha base aqui na Terra, pelo amor e pelo suporte destinados a mim em todo o tempo.

Ao meu namorado, Eduardo, meu apoio incondicional, pelo carinho, pela paciência e pela imensa ajuda na concretização desse trabalho.

À minha orientadora, Juliana, por todo auxílio cedido durante este último ano em forma de compartilhamento de ideias e sugestões a cada orientação.
\end{agradecimentos}
% ---

% ---
% Epígrafe
% ---
\begin{epigrafe}
    \vspace*{\fill}
	\begin{flushright}
		\textit{``Tudo tem o seu tempo determinado e há tempo para todo propósito debaixo do céu.''}

    Ec. 3:1
	\end{flushright}
\end{epigrafe}
% ---

% ---
% RESUMOS
% ---

% ---
% Resumo
% ---
\begin{resumo}
  A Psicologia Jurídica é área recente de atuação, tendo sido constituída no Brasil somente na década de sessenta quando do reconhecimento da profissão em Psicologia. Suas contribuições são diversas e ainda há um vasto campo de práticas a ser descoberto. Este trabalho trata da discussão sobre a atuação da Psicologia Jurídica brasileira frente a casos de abuso sexual infantil, abordando também o que se constitui como limite ou possibilidade nesses casos. O abuso sexual infantil é uma das formas de violência contra crianças que mais se comete atualmente e se estabelece como um fenômeno complexo de ser compreendido e combatido. Como um crime, fere diretamente os direitos da criança. A reflexão dos profissionais sobre a atuação da Psicologia Jurídica diante dos casos de abuso sexual infantil se faz necessária para a possibilidade de mudanças tanto na esfera individual da vítima quanto no contexto social em que esta se insere.

  \vspace{\onelineskip}
    
  \noindent
  \textbf{Palavras-chaves}: psicologia jurídica. abuso sexual infantil. intervenções.
\end{resumo}

% ---
% Abstract
% ---
\begin{resumo}[Abstract]
  \begin{otherlanguage*}{english}
    The Juridical Psychology is a relatively new field of knowledge, being established in Brazil not before the decade of 1960, when Psychology was acknowledged as a profession. Its contributions are numerous and there is still space for research. This work focuses on the discution of the brazilian Juridical Psychology facing child sexual abuse, aproaching also what constitutes possibilities and limitations in such cases. Child sexual abuse is one of the commonest violence types against children, and establishes itself as a complex phenomenon to be comprehended and confronted. As a felony, it strikes directly against children rights. The study of this field of knowledge by professionals amongst cases of child molestation is a requirement for changes either in the private space of the victms as well as in the social context it falls.

    \vspace{\onelineskip}
 
    \noindent 
    \textbf{Keywords}: juridical psychology. child sexual abuse. intervention.
  \end{otherlanguage*}
\end{resumo}
% ---

%\renewcommand{\ABNTEXpartfont}{\fontfamily{cmr}\fontseries{b}\selectfont}
%\renewcommand{\ABNTEXchapterfont}{\fontfamily{cmr}\fontseries{b}\selectfont}
%\renewcommand{\ABNTEXsectionfont}{\fontfamily{cmr}\fontseries{b}\selectfont}

% ---
% Criação do Sumário
% ---
\pdfbookmark[0]{\contentsname}{toc}
\tableofcontents*
\cleardoublepage
% ---

\mainmatter

% ----------------------------------------------------------
% Introdução
% ----------------------------------------------------------
% ==================================
% Introdução
% ==================================

\chapter*[Introdução]{Introdução}
\addcontentsline{toc}{chapter}{Introdução}

A Psicologia Jurídica -- campo em que os conhecimentos de Psicologia são diretamente aplicados aos assuntos relacionados ao Direito -- é campo relativamente recente de atuação dos profissionais psicólogos, sobretudo no Brasil. A nível internacional, os primeiros indícios da relação entre a Psicologia e o Direito se deram por volta do século XVIII através da busca, no âmbito jurídico, de se estabelecer normas para o convívio entre as pessoas conforme os preceitos de conduta. Foi no final do século XIX que a necessidade da Psicologia aplicada ao Direito se firmou, sendo consolidada com a publicação de diversas obras sobre o assunto \cite{JESUS2001}.

Em termos gerais, essas duas disciplinas se aproximam pela atenção ao comportamento humano e se diferenciam na medida emque uma se volta para o mundo do ``ser'' e a outra para o do ``dever ser'' \apud{RIVEROS1995}{ROVINSKI2004}.

No Brasil, a constiuição da Psicologia Jurídica se deu mais tardamente, aproximadamente em meados do século XX, no contexto do reconhecimento da profissão em Psicologia, na década de sessenta. Os psicólogos iniciaram sua atuação de forma prioritária em Varas de Família, Cível, Criminal, da Criança e do Adolescente, elaborando laudos sob o modelo pericial \cite{COSTA2009}. Primeiros  registros de trabalhos de psicólogos em instituições de Justiça no Brasil datam das décadas de setenta e oitenta.

A atuação do psicólogo jurídico brasileiro é regulamentada em legislações específicas e reconhecida pelo Conselho Federal de Psicologia. Muitos são os setores possíveis de intervenção do psicólogo no contexto jurídico, sendo alguns deles a Psicologia Criminal, a Psicologia Jurídica e Direitos de Família, a Psicologia do Testemunho etc. 

Observa-se que uma importante contribuição do psicológo à justiça é no sentido de, através de seus estudos, formular conhecimento acerca da esfera psicológica de agentes envolvidos em processos, colocando-o à disposição do juiz como forma de auxílio às futuras decisões ou sentenças do mesmo \apud{SILVA2007}{LEAL2008}.

Atualmente um trabalho que com frequência tem sido solicitado ao psicólogo jurídico é o de perícia psicológica. A perícia refere-se ao exame de fatos ou situações, praticado por especialista na matéria que lhe é submetida, e tem por objetivo elucidar determinados aspectos técnicos \apud{BRANDIMILLER1996}{ROVINSKI2004}. As solicitações podem ser por motivos de guarda de crianças, interdições, verificação da ocorrência de abuso sexual infantil etc.

O abuso sexual infantil constitui-se em uma das diversas formas de violência contra crianças. Várias são as maneiras como ele pode ser praticado, portanto sua descrição é extensa. No entanto, de uma forma geral, é definido, segundo a Organização Mundial da Saúde, como a participação de crianças em atividades não compatíveis com a sua idade e com as quais ela não está apta a consentir ou compreender completamente.  O fenômeno atinge meninas e meninos de todas as classes econômicas, seja no contexto familiar ou externo a ele. Baseado nisso, ele pode ser classificado como intrafamiliar ou extrafamiliar.
	
O abuso é permeado pelo aspecto do poder, ou seja, o abusador sempre se encontra em uma posição de autoridade com relação à criança, o que lhe incentiva a utilizar-se dela para atingir seu objetivo de gratificação sexual. 
	
As consequências do abuso sexual são múltiplas na vida da criança, podendo atingir as áreas física, psicológica, afetiva, relacional e comportamental, a curto ou a longo prazo. 

Um componente recorrente nas situações de abuso sexual infantil é o silêncio da vítima, por detrás do qual está, principalmente, o medo da criança do que possa ocorrer se o fato vier à tona. 

Pensar o abuso sexual infantil, sua recorrência, as formas crueis de que ocorrem e as relações de poder nele enredadas permite constatar que os direitos humanos da criança não vêm sendo efetivados, mas negligenciados, sobretudo os direitos sexuais.  

Nem sempre na historia esses direitos existiram. Aliás, nem sempre houve infância, tal como a concebemos na modernidade. O sentimento de infância, com as noções de ingenuidade e inocência, surgiu, foi modificado e determinado por uma complexidade de fatores sociais, culturais, políticos e econômicos e por movimentos culturais iluministas e religiosos protestantes da sociedade europeia dos séculos XVII e XVIII \cite{ARIES1981}. Até o século XVII, as crianças eram consideradas como miniaturas do homem, sendo misturadas no meio de adultos e com eles se envolvendo em todos os tipos de práticas, inclusive as sexuais \cite{ARIES2011}. 
	
A família moderna, que teve o início do seu estabelecimento na burguesia do século XVIII, instalou padrões de intimidade e de vida privada, que culminava na união sentimental entre casal e entre os pais e filhos \cite{ARIES2011} e uma nova maneira de agir frente a questões sexuais, implementando uma série de fatores de controle e de educação da criança. 

No Brasil a infância adquire importância a partir do século XVI, sofrendo a influência do processo que ocorria no contexto europeu, mas diferenciando-se desse na medida em que atribuía à criança outros status, provindos, principalmente, de um contexto de preconceito, exploração, abandono e pobreza no Brasil da época – contexto esse que propiciou para que os direitos e o reconhecimento à criança fossem pensados e elaborados. 

Este trabalho pretendeu produzir conhecimentos acerca das práticas da Psicologia Jurídica brasileira frente ao abuso sexual infantil, abrangendo também suas possibilidades e limites. Para tanto, se fez necessário compreender alguns aspectos do fenômeno. Antes disso, foi preciso ainda entender e contextualizar a ``criança'' de que se fala. Isso porque nem sempre ela existiu, foi vista e tratada como é hoje. 

Compreendendo então que o conceito de criança da modernidade foi construído e moldado historicamente a partir de uma gama de mudanças nas sociedades, até que a ela foram atribuídos status de sujeito dependente e em condição especial de desenvolvimento, é possível realizar reflexões sobre a transformação na forma das relações entre crianças e adultos que legitima o abuso de poder e de autoridade por parte desses.

Entendendo também que a violência sexual é um problema social multifatorial e que cada vez mais tem sido noticiado, é possível ponderar algumas práticas diante dela atualmente.

Com base no interesse na Psicologia Jurídica como uma área crescente de atuação do psicólogo, no estágio realizado nesse contexto durante o último ano do curso, bem como pesquisas e o próprio contato com uma perícia psicológica em que uma criança supostamente havia sido abusada sexualmente por um membro de sua família, pretendeu se aprofundar no assunto, por meio de revisão da literatura, a fim de levantar dados sobre a contribuição da Psicologia Jurídica brasileira diante da suspeita e/ou da ocorrência de abuso sexual infantil. Além disso, algumas possibilidades e limites com os quais ela se depara nessa atuação. 

O trabalho está estruturado em três capítulos. O primeiro delineia o caminho que a criança percorreu na história, a partir do século XIII, que lhe concedeu o status que possui na modernidade. Esse capítulo se subdivide em dois pontos: A Constituição da Infância e A Infância no Brasil. O segundo capítulo abrange os aspectos gerais do Abuso Sexual Infantil e o terceiro das práticas da Psicologia Jurídica brasileira diante desse fenômeno, estando subdivido em três pontos: Breve Histórico da Constituição da Psicologia Jurídica, A Psicologia Jurídica no Brasil e Práticas e Limites Frente ao Abuso Sexual Infantil.

A título de esclarecimento, utilizou-se aqui as nomenclaturas \emph{violência sexual infantil} e \emph{abuso sexual infantil} como equivalentes. 


% ----------------------------------------------------------
% Capítulo I
% ----------------------------------------------------------
% ==================================
% Capítulo I
% ==================================

\chapter{Infância e suas transformações}

\section{Constituição da Infância}

Muitas vezes a infância é naturalizada, sendo abordada como um estágio da vida que ``sempre existiu'', concebida da mesma maneira em tempos históricos diferenciados e tratada, portanto, de forma homogênea pela sociedade. Alguns delimitam essa fase da vida pela idade, cronologia ou agem como se o fenômeno fosse suficientemente conhecido, o que indica a necessidade de aprofundamento de estudos e de esclarecimentos. É por existirem visões generalistas e naturalizadas sobre a infância, como analisa \citeonline{FROTA2007}, que se ressalta a importância da problematização sobre esse período da vida no sentido de se tentar compreender dilemas novos ou que se recriam ao longo do tempo. Antes disso, é fundamental buscar as raízes do que atualmente chamamos de Infância. Ela sempre existiu como tal? Se não, como era abordada? Quais elementos da materialidade histórica que podem nos auxiliar a compreender a infância na atualidade?

\citeonline{ALMEIDA2004} apontam que as delimitações das fases da vida se apoiam em visões de homem, de mundo e em valores e normas vigentes da sociedade. Embora em cada sociedade existam concepções predominantes de criança, ainda é possível encontrar diferenciações na forma como ela é tratada e, até certo ponto, na maneira como seus direitos são efetivados, a depender da classe social a que pertence.

A criança, por volta do século XIII na sociedade europeia, não era concebida como um ser completo, mas dotada de propriedades limitadas, chegando a ser considerada algumas vezes como uma forma reduzida de homem \cite{ARIES2011}. A partir de determinantes concretos e históricos, baseados em novas exigências sociais e em estudos, essa visão foi se modificando e o tratamento e o cuidado com as crianças se efetivando de novas maneiras. 

Anterior ao século XII não se conhecia a infância como um período de vida diferenciado dos demais ou não se tentava representá-la nas obras, pelo fato provável de que seu espaço na sociedade ainda não estivesse estabelecido\footnotemark, ou ainda devido ao fato de ela ser considerada um período de passagem na vida da pessoa, uma fase que rapidamente seria transposta e esquecida \cite{ARIES2011}.

\footnotetext{Uma ``miniatura otoniana'' do século XI traz a ideia da modificação imposta aos corpos da criança. Ela retrata a cena do Evangelho em que Jesus pede que se deixe vir a ele as criancinhas. Foram agrupadas em torno de Jesus oito ``verdadeiros homens'', apenas retratados numa escala menor, sem nenhuma das particularidades da infância \cite[p. 17]{ARIES2011}.}

Do século XIV ao século XVII, essencialmente nas artes, houve a predominância da ilustração da infância de forma religiosa. Nesses séculos constam, inicialmente, obras da figura do menino Jesus, da Nossa Senhora menina e, posteriormente, outras infâncias sagradas, com as figuras de São João, São Tiago, Maria Zebedeu e Maria Salomé. Segundo \citeonline[p. 19]{ARIES2011}, ``com a maternidade da Virgem a tenra infância ingressou no mundo das representações'', o que influenciou na representação de cenas de famílias em que as crianças já apareciam desenhadas com traços ternos e ingênuos, ainda que em forma de homens em miniatura. 

No século XV surgiu a representação da criança nua -- o putto -- que perdurou mais fortemente no século XVII. Anterior a esse período, quase nunca se representava a criança de forma nua, especialmente o Menino Jesus, que só apareceu desnudado no final da Idade Média, próximo ao século XV. O gosto por esse tipo de representação, segundo \citeonline[p. 26]{ARIES2011}, ia além do gosto pela nudez clássica e se relacionava com um ``amplo movimento de interesse em favor da infância''. Esse movimento nas artes de representação da criança de diversas formas\footnotemark eram indícios de seu reconhecimento e da saída de sua condição de anonimato \cite{ARIES1981}.

\footnotetext{Além das representações comuns, surgiu também o retrato da criança morta, denotando um sentimento que começava a surgir para com a criança, cujo falecimento já não poderia ser considerado tão tolerável, como tinha sido até então \cite{ARIES2011}.}

No século XVII, embora tenha se dado a percepção de que criança era um ser diferenciado dos demais, em alguns aspectos o tratamento dirigido a elas permanecia ainda indiferenciado com relação aos adultos. Não se omitia ou restringia-lhes assuntos e práticas relacionadas ao sexo, coisas que viriam a ser desrespeitosas e reprováveis à inocência infanto-juvenil em sociedades vindouras \cite{MOTT1998}. Pelo contrário, a iniciação sexual infantil em algumas sociedades, como a Grécia Antiga, era ``conduta normal, método pedagógico ou ritual de iniciação no mundo adulto'' \apud[p. 45]{dover78}{MOTT1998}. Segundo \citeonline{MOTT1998}, alguns historiadores discutem que a necessidade do afastamento da criança das questões sexuais é recente na história ocidental. Esse afastamento pode ter relação, dentre outros fatores, com a ideia de inocência que se tornou diretamente ligada à infância à medida que a conjuntura das sociedades se alterava. A não censura de questões sexuais às crianças, existente nos séculos passados, se dava, na opinião geral:

\begin{citacao}
	Primeiro, porque se acreditava que a criança impúbere fosse alheia e indiferente à sexualidade. Portanto, os gestos e as alusões não tinham consequência sobre a criança, tornavam-se gratuitos e perdiam sua especificidade sexual -- neutralizavam-se. Segundo, porque ainda não existia o sentimento de que as referências aos assuntos sexuais, mesmo que despojadas na prática de segundas intenções equívocas, pudessem macular a inocência infantil -- de fato ou segundo a opinião que se tinha da inocência. Na realidade, não se acreditava que essa inocência realmente existisse \cite[p. 132]{ARIES1981}.
\end{citacao}

Embora essa fosse a opinião predominante, alguns educadores e moralistas da época já haviam iniciado um movimento de reconhecimento das particularidades das crianças dois séculos antes, no século XV. Esse interesse específico pela infância se pautava principalmente na preocupação com a naturalidade das práticas sexuais perante os menores. A ideia de que certos comportamentos deviam ser associados à culpa -- embora os pequenos não tivessem noção dela -- e que poderiam beirar a ``sodomia'', começou a ser difundida, principalmente pelo moralista Gerson \cite[p. 80]{ARIES2011}. Ele foi responsável por estudar o comportamento das crianças no tocante à sexualidade, a fim de que o sentimento de culpa fosse despertado nos pequenos -- de 10 a 12 anos de idade -- através dos confessores. Além disso, Gerson considerava o ato da masturbação grave, ainda que fosse uma prática comum e da qual nenhuma criança se sentisse culpada, mas acreditava que a educação deveria exercer o papel de conservá-la dos riscos da sexualidade \cite{ARIES2011}.

A preocupação de Gerson e de outros educadores e moralistas do século XV contribuiu para uma determinação do conhecimento acerca do comportamento da criança, estabelecendo a necessidade de modificações nas rotinas educacionais, em que uma nova forma de relação com as crianças se daria \cite{ARIES2011}.

É a partir do final século XVII que o significado das crianças para os adultos passa a se transformar: de objeto de paparicação e risos para objeto de preocupações disciplinares. Segundo \citeonline{ARIES2011},

\begin{citacao}
	É entre os moralistas e os educadores do século XVII que vemos formar-se esse outro sentimento de infância (…) que inspirou toda a educação até o século XX, tanto na cidade como no campo, na burguesia como no povo. O apego à infância e à sua particularidade não se exprimia mais através da distração e da brincadeira, mas através do interesse psicológico e da preocupação moral (p. 104).
\end{citacao}

Esse contexto europeu específico do século XVII foi permeado pela percepção dos adultos, através da influência dos moralistas, de que as crianças não eram capazes, por conta própria, de enfrentar a vida. Sendo assim, aos pais, gradativamente, foi incumbida a função de constituição moral e espiritual da criança enquanto que à Escolarização atribuiu-se a formação restante necessária e apropriada a uma vida adulta, conforme o pensamento vigente da sociedade \cite{MIRANDA1989}.  

É importante ressaltar, contudo, que até o século XVII, a escolarização não foi geral, ou seja, nem todas as crianças passavam pela escola. Segundo \citeonline{ARIES2011}, a separação entre criança e vida adulta, pautada no método da escolarização, estava intimamente ligada ao movimento de moralização promovida pelos reformadores católicos ou protestantes, vinculados à Igreja, às leis ou ao Estado, momento também de ascendência da burguesia. Sendo assim, às meninas de famílias não burguesas ainda eram repassados ensinamentos domésticos por suas mães e ensinamentos religiosos em conventos. Esses fatores colaboraram para que, entre estas, se mantivessem inalteradas as características de infância curta e a conduta precoce \cite{ARIES2011}. 

Essas mudanças que se deram mediante uma reorganização social em que a burguesia ascendia e impunha seus direitos determinaram a formação de uma ideia de infância que, progressivamente, adquiriu status cristalizado, absoluto e global, encobrindo todo o aspecto social embutido nas relações anteriores entre criança e adulto ou sociedade \cite{MIRANDA1989}. A nova imagem que ia se atribuindo à criança contribuiu para o estabelecimento de um modelo, que seria internalizado ou rejeitado por ela. Conforme \citeonline{MIRANDA1989}, 

\begin{citacao}
	Tanto a assimilação do modelo quanto a sua recusa são plenamente justificadas pela idéia de natureza infantil. Ideologicamente, fica legitimada a necessidade de se auxiliar a criança no seu processo de assimilação das normas e penalizar aquelas que as recusam, em nome de uma condição natural na criança.
\end{citacao}

A constituição da mentalidade moderna de infância como algo que exige cuidados diferenciados em aspectos específicos se fortaleceu por volta do final do século XVII e pôde ser manifesta em âmbitos como o das artes\footnotemark, com a representação da criança nos retratos de forma centralizada ou separada dos demais membros da família, e na educação -- cujas mudanças já remontavam do final do século XVI --, com a intolerância a livros de linguagem ou conteúdo impróprios às crianças, marcando o início do pudor na linguagem escrita \cite{ARIES2011}. Além da preocupação com o conteúdo dos livros acadêmicos, passou-se a orientar mudanças na forma com que os castigos eram dados nas escolas. Sendo assim, punições físicas que anteriormente expunham o corpo dos pequenos ocorreriam de maneira que a exposição fosse evitada \cite{ARIES1981}.

\footnotetext{Nas artes, a partir do início do século XVII, a infância retratada se forma sagrada ganhou maior importância que nos séculos anteriores, além de o Menino Jesus passar a ser retratado isoladamente em pinturas, gravuras e esculturas religiosas. \citeonline[p. 93]{ARIES2011} destaca a relação imediata que foi estabelecida ``entre essa devoção da santa infância e o grande movimento de interesse pela infância, de criação de pequenas escolas e colégios de preocupação pedagógica''.}

Além das transformações iniciadas na educação no final do século XVII e início do século XVIII, também surgia o sentimento de afeição pela criança no seio familiar burguês. De uma forma geral, a burguesia\footnotemark dos séculos XVIII e XIX experienciou a reforma moralizante, iniciada no século XVII, por meio de modificações nos costumes das famílias em diversos níveis e a instauração de ``escrúpulos'' e ``decência'' \cite[p. 77]{ARIES2011}. Sobre o modelo de família existente anterior à reforma, \citeonline[p. x]{ARIES2011} esboça:

\begin{citacao}
	\ldots tinha por missão -- sentida por todos -- a conservação dos bens, a prática comum de um ofício, a ajuda mútua quotidiana num mundo em que um homem, e mais ainda uma mulher isolados não podiam sobreviver, e ainda, nos casos de crise, a proteção da honra e das vidas. Ela não tinha função afetiva. Isso não quer dizer que o amor estivesse sempre ausente: ao contrário, ele é muitas vezes reconhecível, em alguns casos desde o noivado, mais geralmente depois do casamento, criado e alimentado pela vida em comum, como na família do Duque de Saint-Simon. Mas (e é isso o que importa), o sentimento entre os cônjuges, entre os pais e os filhos, não era necessário à existência nem ao equilíbrio da família: se ele existisse, tanto melhor.
\end{citacao}

\footnotetext{Sobretudo a inglesa e a francesa.}

Após o século XVII, o que se observa é o início de uma organização familiar em que a criança adquire importância e lugar central, saindo de seu estado de anonimato. Dessa forma, tornou-se ``impossível perdê-la ou substituí-la sem uma enorme dor'' \cite[p. xi]{ARIES2011}, o que com o passar dos tempos se traduziu em uma redução voluntária da natalidade por parte dos adultos, além da preocupação com a higiene e com a saúde física dos pequenos. Além disso, a prática do infanticídio existente até esse século sofreu redução por consequência da maior sensibilidade atribuída aos pequenos, decorrente do contexto de profunda cristianização dos costumes \cite{ARIES2011}. No combate ao infanticídio, os religiosos iniciaram um movimento de ``Sacralização da Infância'', que se baseava na relação da infância com a ideia de ``Pureza''. A elas foi atribuída a existência de alma e, ainda, iniciou-se o costume de batizá-las, antes que a morte lhes sobreviesse \cite[p. 34]{SANTOS1994}.

Outras transformações puderam ser constatadas, como a consolidação do desaparecimento da liberdade com relação às brincadeiras sexuais, marcando a consideração da infância como fase da vida ``ingênua, pura e angelical'' e ainda o prolongamento desse período da vida, ou seja, o retardamento da difusão dos pequenos no meio de adultos \cite{ARIES2011}. 

\begin{citacao}
	O sentido da inocência infantil resultou portanto numa dupla atitude moral com relação à infância: preservá-la da sujeira da vida, e especialmente da sexualidade tolerada -- quando não aprovada -- entre os adultos; e fortalecê-la, desenvolvendo o caráter e a razão. Pode parecer que existe aí uma contradição, pois de um lado a infância é conservada, e de outro é tornada mais velha do que realmente é. Mas essa contradição só existe para nós, homens do século XX (p. 91).
\end{citacao}

Além disso, o próprio ambiente doméstico sofreu alterações: antes aberto para o exterior, agora retraído e preparado para a intimidade da família, longe da rua e da vida coletiva. Em casa os pais passaram a ser modelos de identificação para as crianças. A sexualidade se restringiu ao âmbito do lar e um novo padrão para ela foi instaurado: o fim de reprodução. O que fugisse disso seria coibido. \citeonline[p. 9]{FOUCAULT1988} faz referência a esse momento na história:

\begin{citacao}
	A sexualidade é, então, cuidadosamente encerrada. Muda-se para dentro de casa. A família conjugal a confisca. E absorve-a, inteiramente, na seriedade da função de reproduzir. Em torno do sexo, se cala. O casal, legítimo e procriador, dita a lei. Impõe-se como modelo, faz reinar a norma, detém a verdade, guarda o direito de falar, reservando-se o princípio do segredo.
\end{citacao}

O que \citeonline{FOUCAULT1988} afirma, no entanto, não é que se tenha cessado o assunto do sexo a partir dessas mudanças observadas no final do século XVII. Ao contrário, passou-se a discutir muito sobre ele, mas discursos acerca do que se podia ou não fazer. O sexo não deveria mais ``ser mencionado sem prudência'' (p. 23), segundo a nova pastoral, ``mas seus aspectos, suas correlações, seus efeitos devem ser seguidos até às mais finas ramificações \ldots'' (p. 23). Assim, o sexo torna-se sitiado por uma fala que não lhe admite nenhuma opacidade:

\begin{citacao}
	A pastoral cristã inscreveu, como dever fundamental, a tarefa de fazer passar tudo o que se relaciona com o sexo pelo crivo interminável da palavra. A interdição de certas palavras, a decência das expressões, todas as censuras do vocabulário poderiam muito bem ser apenas dispositivos secundários com relação a essa grande sujeição: maneiras de torná-la moralmente aceitável e tecnicamente útil \cite[p. 24]{FOUCAULT1988}.
\end{citacao}

É importante destacar novamente que as transformações iniciadas no final do século XVII que envolveram as crianças, tais como as mudanças nos trajes, no linguajar, nas brincadeiras, nos comportamentos, nos papeis sociais e nos aspectos sexuais não abrangeram a princípio todas elas. Elas puderam ser mais facilmente percebidas nas famílias burguesas em que, além das melhores condições financeiras, os princípios religiosos e morais se faziam mais prementes, sendo, portanto, mantidos e reforçados, principalmente pela Igreja Católica. Crianças de famílias pobres, como as de camponeses e artesãos, continuavam convivendo com adultos, vestindo o mesmo tipo de roupas que eles e, no caso das meninas, aprendendo e desempenhando afazeres domésticos juntamente com suas mães \cite{ARIES2011}. Os meninos, por sua vez, aprendendo e ajudando nas atividades laborais diárias.

Considerando essa diferenciação entre as classes sociais das famílias, depreende-se que, embora a ideia de cuidado e de proteção à criança devido à atribuição de uma fragilidade fosse difundida nos mais diversos âmbitos, ela não era exercida em todas as classes sociais, já que na classe baixa as famílias não podiam deixar de contar com a ajuda dessa mão de obra na agricultura e nos trabalhos familiares\footnotemark.

\footnotetext{Desde o século XX, dada a industrialização incipiente no Brasil e agudização de certo conflito social, a presença de crianças e adolescentes em fábricas e oficinas de São Paulo tornou-se predominante, principalmente no setor têxtil \cite{moura98}. Constatou-se ainda que grande parte dos trabalhadores menores se acidentava durante as atividades. Verifica-se, portanto, a diferenciação nas condições a que elas são submetidas de acordo com sua classe social, além do cuidado não exercido ou exercido de forma inadequada.}

%%%%%%%% Continuação

É necessário salientar também que essa família do século XVII ainda não é a família moderna, conforme destaca \citeonline{ARIES2011}. Porque, até então, ela conserva a sociabilidade e mantém-se como um ``centro de relações sociais, a capital de uma pequena sociedade complexa e hierarquizada, comandada pelo chefe da família'' (p. 189). A família moderna, que começou a se formar a partir do século XVIII, se afasta da sociedade e se volta, principalmente, para o cuidado com o desenvolvimento de suas crianças, não se mantendo mais como um espaço aberto tal como era.

Foi no século XVIII que uma sistematização marcante da infância se deu. O filósofo suíço Jean Jacques Rousseau foi um dos que contribuíram por especificar uma diferença decisiva entre a infância e a maturidade -- denominada por ele ``idade da razão'' --, afirmando sobre a inabilidade da criança para pensar em abstrações e a limitação de seu pensamento àquilo que se pode ver ou manejar. Dentre outros trabalhos de Rousseau, encontra-se a divisão da infância em estágios, ideia que se assemelhava a Jean Piaget\footnotemark, no século XX \cite[p. 9]{GALLATIN1978}. Embora os estudos de Rousseau tenham sido, a princípio, conhecidos somente por uma camada mais favorecida, interessada em educar seus filhos e em conhecer as diferenças intelectuais entre os pequenos e os adultos, foram fundamentais para que se solidificasse posteriormente o conceito de infância na cultura ocidental.  

\footnotetext{Suiço (1896-1980) que muito contribuiu à Psicologia com estudos sobre a infância.}

No século XIX, grandes mudanças na cultura, na economia e na política marcaram as sociedades. Em muitos países -- inclusive no Brasil -- o capitalismo foi firmado e expandido, de forma que a burguesia pôde se consolidar no poder político. Nesse contexto, a concepção de infância também sofreu alterações: a criança ganhou lugar central das atenções, demandando proteção e cuidados \apud{SANTOS2907}{KULLER2909}. A instalação da vida privada e com ela a intimidade e o sentimento de união afetiva entre os membros da família finalmente se consolidou graças ao arranjo em função das necessidades do modelo capitalista, em detrimento das formas comunitárias tradicionais \cite{MIRANDA1989}.

Foi nesse século que se tornaram comuns o conceito e a sistematização dos estágios da infância iniciados no século anterior por alguns teóricos, dentre os quais Rousseau \cite{GALLATIN1978}. Dessa forma, o século XIX foi marcado pela difusão da necessidade de se compreender essa fase da vida, no que concerne ao seu desenvolvimento, demarcando melhor suas fronteiras. 

No século XX a criança foi alvo de estudos de outras ordens, dentre elas, a Psicologia. Esse interesse já remonta aos séculos anteriores quando da preocupação de Gerson com a correta educação das crianças. Observações sobre a psicologia infantil já constavam em textos do final do século XVI e do século XVII, revelando a tentativa da época de se adentrar na mentalidade das crianças para o planejamento da melhor forma de educá-las. Sobre esse assunto, \citeonline[p. 104]{ARIES2011} complementa:

\begin{citacao}
	\ldots as pessoas se preocupavam muito com as crianças, consideradas testemunhos da inocência batismal, semelhantes aos anjos e próximas de Cristo, que as havia amado. Mas esse interesse impunha que se desenvolvesse nas crianças uma razão ainda frágil e que se fizesse delas homens racionais e cristãos.
\end{citacao}

O século XX foi espaço propício para o estudo científico sobre a criança, que não era mais vista com base somente em princípios morais, mas em sua constituição como ser diferenciado. Estudos da Psicologia infantil do século XX -- denominada também ``psicologia do desenvolvimento'', ``psicologia da criança'' e ``psicologia da aprendizagem'' -- enfatizavam linguagem, motricidade, pensamento etc. \cite[p. 643]{ALMEIDA2004} e demonstravam que a infância tinha um desenvolvimento específico, diferente do desenvolvimento adulto e merecia atenção e pesquisas.

O que pôde ser observado mais fortemente com os anos e os estudos em psicologia do desenvolvimento foram a sistematização e a descrição do chamado ``ciclo vital'' \cite[p. 20]{CASTRO1998}. Modos de ser e viver da criança se tornaram determinados à medida que a ciência moderna, a Psicologia e ramos dela -- inclusive a do Desenvolvimento -- ditaram concepções de infância a partir dos estudos realizados.

Em meio às transformações que envolveram a criança no período que compreende os séculos XVII ao XIX em diante, passando pelo surgimento do que \citeonline{ARIES2011} chama de ``sentimento da infância'' até a necessidade da sistematização de seus estágios e a preocupação com a educação e a saúde nessa fase da vida, percebe-se um afastamento do mundo infantil ao que se relaciona com o mundo da sexualidade. A ideia de inocência que se tornou gradativamente associada à criança dificultou e ainda dificulta reflexões sobre qualquer assunto em que ambos -- infância e sexualidade -- se aproximem. Conforme \citeonline[p. 2]{NUNES2000} afirmam, o que ainda predomina concernente à ``sexualidade infantil'' é um contexto de ``incompreensão, a improvisação do senso comum, o repetir de preconceitos e quase sempre o descaso''.

Outra mudança fundamental -- decorrente de uma complexidade de fatores -- que também permeou esse período foi o abandono da visão de que o sexo é somente para procriação \cite[n.p.]{REZENDE2008}, portanto restrito apenas aos adultos. Dessa forma, ao assuntos sexuais foi relacionada novamente a ideia do prazer. 

Comumente a sexualidade é relacionada à genitalidade, assim como vida sexual equivalente a relação sexual \cite{BEARZOTI1994}. Sigmund Freud (1856-1939) abrange o conceito, retirando-o dessas conotações. Para ele, sexualidade vai além do ato sexual e da reprodução: trata-se de energia.

A Psicanálise desenvolvida por Freud no século XIX, na Europa, enfatiza a infância e sua relação com a sexualidade. Para ele não se pode atribuir à criança, até mesmo ao bebê recém-nascido, as noções de inocência, pureza e ausência de vício, porque ela é repleta de desejos os quais busca saciar constantemente. O reservatório a que Freud atribuiu os impulsos biológicos básicos do bebê, bem como toda sua energia como ser humano, é o ``id'' \cite{GALLATIN1978}. 

Em sua obra ``Os três ensaios sobre a teoria da sexualidade'', de 1905, o infantil é associado diretamente ao desenvolvimento pulsional. Sobre esse desenvolvimento \citeonline[p. 68]{ZAVARONI2007} afirmam: 

 \begin{citacao}
	Na elaboração de sua hipótese, sobre o desenvolvimento pulsional, Freud (1905/1980) aponta para a marca da sobreposição que se constituirá como característica do processo de subjetivação, em que os modos mais arcaicos do desenvolvimento permanecem presentes, também, na sexualidade do adulto. Assim, o adulto portará para sempre o infantil que o constituiu.
\end{citacao}

Nessa obra, Freud divide o período pré-puberal de desenvolvimento da personalidade em estágios dominados por tendências sexuais, ``essas provenientes de impulsos instintivos e não aprendidos'' que objetivam o prazer'' \cite[p. 2]{SCHINDHELM2011}, proposição que vai de encontro ao que ele denominou de ``Princípio do Prazer'', responsável por regular os processos de desenvolvimento do ser humano.

Na infância, o objeto sexual do instinto sexual se constitui no próprio corpo da criança. Durante o seu desenvolvimento, os objetos se modificam, até que a busca do prazer atinja seu nível máximo e se complete no prazer do ato sexual.

De maneira sucinta, a sexualidade, de acordo com o conceito psicanalítico, 

\begin{citacao}
	\ldots é energia vital instintiva direcionada para o prazer, passível de variações quantitativas e qualitativas, vinculada à homeostase, à afetividade, às relações sociais, às fases do desenvolvimento da libido infantil, ao erotismo, à genitalidade, à relação sexual, à procriação e à sublimação \cite[p. 117]{BEARZOTI1994}.
\end{citacao}

O que permite concluir que para ele a sexualidade é um conceito mais complexo e abrangente do que se supõe ao imaginar o simples ato sexual ou prazer provindo dele.

Freud surpreendeu a sociedade da época com sua teoria de que a vida e o comportamento de um adulto eram influenciados pelas experiências e condutas sexuais infantis, desfazendo o pensamento predominante de que a criança era uma criatura pura e inocente \cite{SCHINDHELM2011}, que não se aproxima de maneira alguma às questões sexuais.

O ``infantil'', de acordo com o conceito psicanalítico, transcende o que é da ordem da cronologia e das experiências passíveis de narração, diferindo-se assim do que é chamado de ``infância cronológica'', no sentido de que é algo que não se dá a ver, mas surge apenas no modo como o indivíduo se põe em análise \cite[p. 66]{ZAVARONI2007}.

Freud e seus estudos, embora não aceitos de imediato por irem contra a ideia de inocência tida como inerente à criança, trouxeram a reflexão acerca da função sexual não apenas ligada à reprodução, mas ao desenvolvimento da vida como um todo. De certa forma pôde contribuir para ressaltar que a sexualidade infantil existe, abrindo caminhos para que se fossem discutidas melhores maneiras de cuidado no tratamento ou contato com as crianças.  

Outro autor que discute o tema da sexualidade infantil é \citeonline{FOUCAULT1988}. Sua pretensão não é negar que o sexo tenha sido ``proibido, restringido, bloqueado, mascarado'' (p. 17), nem afirmar que a interdição do sexo é uma ilusão, mas, sim, que a ilusão está em fazer dessa interdição o elemento fundamental do que se diz do sexo desde o início dos tempos modernos. Segundo ele, é fato que a antiga liberdade com que as crianças eram expostas a qualquer assunto ou prática sexual tenha desaparecido, no entanto, isso não significa um silêncio puro e simples:

\begin{citacao}
	Não se deve fazer divisão binária entre o que se diz e o que não se diz; é preciso tentar determinar as diferentes maneiras de não dizer, como são distribuídos os que podem e os que não podem falar, que tipo de discurso é autorizado ou que forma de discrição é exigida a uns e outros. Não existe um só, mas muitos silêncios e são parte integrante das estratégias que apoiam e atravessam os discursos \cite[p. 30]{FOUCAULT1988}.
\end{citacao}

\citeauthoronline{FOUCAULT1988} traz contribuição para a reflexão sobre a hipótese repressiva pela qual a sexualidade passou no sistema social sendo relacionada às questões morais. Para ele, o indivíduo tem o prazer como algo natural e ativo, que é buscado constantemente. Portanto, da mesma forma a criança em qualquer manifestação sexual está em busca do prazer \cite{DONIZETE2010}. Se a sociedade é permeada por relações de poder que marcam a conduta das pessoas, é possível discutir acerca do poder nas relações no âmbito da sexualidade.

Esses e outros estudos ou investigações puderam contribuir para que se buscasse continuamente observar a forma com que se efetivava o cuidado e o tratamento às crianças, permitindo uma atenção e uma cautela maiores nas formas de zelo e de educação das mesmas. Além disso, motivou-se que leis e concepções de proteção à criança e de controle fossem criados ou consolidados.

\section{A infância no Brasil}

Ao mesmo tempo que sofreu determinação por muitas das transformações ocorridas na Europa, o Brasil teve sua própria organização da infância, com características que se diferiram totalmente daquela sociedade.

A criança brasileira teve sua história contada a partir do século XVI, quando do interesse da Igreja Católica, representada pela Companhia de Jesus, pelas crianças indígenas, os ``Culumins'' e a salvação de suas almas. Conforme \apudonline{PRIORE1992}{SANTOS1994} afirma, esse interesse teve influência a partir das transformações que tomaram conta da Europa, que fizeram surgir o sentimento de valorização da infância, se refletindo na Igreja Católica e em sua forma de ideologizar a criança.

Esse período no Brasil foi marcado pela missão principal de conversão das almas, o que demandou uma mudança na pedagogia a partir de métodos de disciplina como castigos, ameaças e vigilância constante. Alguns desses métodos serviam de instrumento através dos quais os padres deixavam claras as ideias de ``inferno e paraíso'' \cite[p. 37]{SANTOS1994}.

Comparando-se o sentimento de família na Europa Ocidental com o que se deu no Brasil, observa-se que até o século XIX seu desenvolvimento ainda era incipiente e, a partir de seu início, ele estava diretamente ligado às regras higiênicas prescritas pela medicina. Quando o progresso nas relações familiares no Brasil se deu, de fato, atingiu de forma maciça apenas a família branca. A criança negra não foi inclusa nesse processo da modernização da família brasileira, não lhe sendo atribuídos status de pureza, inocência e felicidade \cite{SANTOS1994}. 

No âmbito da escolarização, a infância começa a adquirir importância no Brasil, aproximadamente, em 1875, com o surgimento dos primeiros jardins de Infância baseados na proposta de Froebel\footnotemark, nos Estados do Rio de Janeiro e São Paulo. Esses jardins de Infância foram introduzidos no sistema educacional de caráter privado com o objetivo de atender às crianças filhas da classe média industrial emergente. Em 1930, após reformas jurídico educacionais, o atendimento pré-escolar passou a contar com a participação direta do setor público \cite{AHMAD2009}.

\footnotetext{Friedrich Froebel (1782-1852), alemão, foi um dos primeiros educadores a considerar a importância do início da infância na formação das pessoas.}

Com o movimento da sociedade civil e de órgãos governamentais para que o atendimento às crianças de zero a seis anos fosse amplamente reconhecido na Constituição de 1988, atingiu-se o reconhecimento da Educação Infantil como um direito da criança. É justamente nessa década que a criança começa a surgir no cenário jurídico brasileiro, não apenas no tocante à educação, mas ao reconhecimento de seus direitos e da garantia de sua proteção de uma forma geral. 

O contexto de inserção da criança no âmbito do Direito envolve uma complexidade de fatores. A visibilidade à criança no Brasil esteve sempre bastante ligada a uma realidade de confrontos em que se observaram ``trabalhos forçados, extermínio, abandono, criminalidade, prostituição, analfabetismo, sobrevida nas ruas, nas instituições, no lixo, dependência de drogas, extirpação de órgãos'' dentre outros \cite[p. 40]{SANTOS1994}. Dessa forma, entidades começaram a se levantar a fim de proteger a infância e a adolescência.

A denominação da criança no sistema jurídico serviu de instrumento que confirmasse a relação com o preconceito, o abandono e a exploração que foi associada à criança de classes mais desprovidas. Observou-se a origem do termo ``menor''. De acordo com \citeonline{LONDONO1998}, no final do século XIX e começo do século XX, esse termo aparecia recorrentemente no contexto jurídico do Brasil. Anterior a esse período, a palavra fazia referência aos limites de idade, dizendo acerca dos direitos das pessoas à emancipação ou assunção de responsabilidades civis e relacionadas à Igreja. Após a proclamação da Independência o termo ``menor'' foi utilizado por juristas na definição da idade, ``como um dos critérios que definiam a responsabilidade penal do indivíduo pelos seus atos'' (p. 130).

Ao final do século XIX, juristas brasileiros puderam constatar a presença do que se teve por ``menor'', até então, na sociedade brasileira, em crianças e adolescentes desprovidos de condições financeiras adequadas e sem o cuidado dos pais, sendo chamadas também de ``abandonadas'' (p. 135). Elas podiam ser encontradas nas ruas do centro das cidades, nas praças, em mercados e, por vezes, cometiam delitos, sendo chamadas, assim, de ``menores criminosos'' (p. 135). \citeonline[p. 135]{LONDONO1998} analisa a situação:

\begin{citacao}
	Partindo dessa definição, através dos jornais, das revistas jurídicas, dos discursos e das conferências acadêmicas foi se definindo uma imagem do menor, que se caracterizava principalmente como criança pobre, totalmente desprotegida moral e materialmente pelos seus pais, seus tutores, o Estado e a sociedade. Relacionando a origem do abandono com as condições econômicas e sociais que a modernização trouxe, os juristas, tanto no começo do século como nos anos 20 e 30, não deixaram porém de apontar a decomposição da família e a dissolução do poder paterno, como os principais responsáveis de tal situação.
\end{citacao}

Assim, percebe-se que a atenção às crianças de classes baixas se deu fortemente ligada ao fato de ela poder ser um sujeito de delinquência e não de direitos.

Em 1941 foi criado o Serviço de Assistência ao Menor -- SAM -- como exemplo da expressão de políticas públicas direcionadas à infância e à adolescência pobres no Brasil. Tornou-se, contudo, uma política de exclusão-reclusão ao retirar o ``menor'' do convívio social e encaminhá-lo a ``espaços institucionais de reclusão''. Essa política perdurou até 1964, quando a Ditadura Militar instituiu a Fundação Nacional do Bem Estar do Menor -- FUNABEM \cite[p. 54]{VASCONCELOS2002}.

Os novos status atribuídos à criança, juntamente com uma série de movimentos sociais em defesa da criança e do adolescente, que antes eram tratados como ``menor'', também contribuíram para a efetivação de políticas públicas relacionadas à criança como um cidadão de direitos que merecia uma atenção especial. Dessa forma, em 1959, a Declaração Internacional dos Direitos da Criança determinou a responsabilidade do adulto perante a criança e criou mecanismos de controle do cumprimento dessa regra. No Brasil, a Lei 8.069 do dia 13 de julho de 1990 regulamenta os direitos e a primazia de proteção a crianças e adolescentes no Estatuto da Criança e do Adolescente. Consoante a essa Lei, é caracterizado como criança aquele que possui a idade de até doze anos incompletos (ECA, Art. 2\textordmasculine).

Ressalta-se, por fim, que a inclusão da criança no âmbito do Direito não se deu somente porque ela necessitava de um cuidado especial devido à sua fragilidade, mas também porque o problema da criminalidade infanto-juvenil apresentou um aumento significativo no século XX. Segundo \citeonline{SOUZA2001}, esse é um problema que atinge a criança -- de família que vive na faixa de pobreza e miséria -- logo em seu início de vida, devido à carência de recursos básicos à sobrevivência, tornando-se assim vítima da negligência no que se refere à garantia de seus direitos fundamentais. 

Através do aprofundamento no estudo sobre a infância, é possível perceber que as crianças não receberam o mesmo tratamento nas sociedades, ou tiveram seu espaço garantido e direitos consolidados em todo tempo, ainda que se tenham revelado como sujeitos de cuidados e atenção específica. A legislação não foi sempre efetiva e os casos de violação aos seus direitos fundamentais eram constatados continuamente. 

O sistema jurídico brasileiro nem sempre abrangeu e priorizou os direitos das crianças e dos adolescentes. Muitas mudanças, infelizmente, ficam apenas no âmbito da palavra, sem passarem à ação e, embora algumas modificações já tenham se efetivado, ainda há muito que se conseguir. A aquisição dos direitos das crianças não se deu de forma imediata tampouco se darão os próximos êxitos. 

% ----------------------------------------------------------
% Capítulo II
% ----------------------------------------------------------
% ==================================
% Capítulo II
% ==================================

\chapter{Aspectos gerais do abuso sexual infantil}

Em tempos recentes na História a criança passou a possuir direitos e ser estudada como indivíduo em situação especial de desenvolvimento. Esse movimento foi também acompanhado pela preocupação com a garantia permanente e legal de sua proteção. No entanto, esse contexto não foi suficiente para que se extinguisse a violência à população infantil tampouco para impedir que novos dispositivos fossem empregados em atos de crueldade \cite{CAMPOS2002}. 

A violência contra crianças não é uma questão que remonta apenas ao século XX, mas acompanha a própria história da humanidade, ainda que não se tenha registro de todos os casos desde seu início \apud[n.p.]{ASSIS1994}{ALGERISOUZA2006}. No Brasil, é possível encontrar referências da prática de violência doméstica contra crianças já no período colonial (1500 -- 1822) \cite{LONGO2002}.

A violência infantil se revela de múltiplas maneiras, mas se configura, normalmente, em algum dos tipos já definidos, como negligência, abandono, violência física, psicológica,  sexual, dentre outros. 

Conforme \apudonline{FALEIROS2000}{JUNG2006} aponta, a violência sexual contra crianças se trata de uma  construção histórica e cultural nos processos sociais, que se articula ao nível de desenvolvimento e de civilização das sociedades em que acontece. 

Considerando a sociedade europeia dos séculos XIII ao XVII, em que os adultos submetiam crianças a práticas sexuais -- o que era relativamente comum diante de todos -- é possível refletir que essas práticas não se equiparam as que hoje são classificadas como formas de abuso sexual infantil. Isso porque a sociedade dessa época tinha uma maneira diferente de tratar e de se relacionar com as crianças e possuia concepções e costumes diferentes dos que predominam na maioria das sociedades atuais. À criança ainda não se havia atribuído os aspectos de inocência e de fragilidade tampouco lhes eram destinadas direitos específicos e leis de proteção. Aliás, ela era vista como um ``homem em miniatura'' \cite{ARIES2011} e não como um ser diferenciado que necessitasse de cuidados e de atenção especial, o que pode, em partes, explicar o tratamento indiferenciado que recebia. 

Diante disso, é possível refletir que houve um processo complexo responsável por caracterizar essas práticas como um crime e por afastar a criança de questões relacionadas à sexualidade como um todo. Diante dessas ponderações, cabe refletir e questionar: por que a criança? O que está por detrás do abuso sexual infantil que o torna um crime tão recorrente e, ao mesmo tempo, tão velado?

A violência sexual, de acordo com dados do Ministério da Saúde, é o segundo tipo de violência mais praticado contra crianças de zero a nove anos de idade. Como um crime de natureza sexual, torna-se um problema social de extrema importância e de saúde pública que provoca sérias consequências.

\begin{citacao}
	A prevalência do abuso sexual na população geral foi foco de um estudo realizado pela OMS (1999). Os resultados apontaram que, aproximadamente, 20\% das mulheres e 5 a 10\% dos homens sofreram abuso sexual na infância. Além disso, em todo o mundo, as taxas de prevalência de intercurso sexual forçado e outras formas de violência sexual que envolvem toques entre adolescentes com idade inferior a 18 anos é de 73 milhões (ou 7\%) entre os meninos e 150 milhões (ou 14\%) entre as meninas. \cite[p. 69]{HABIGZANG2012}
\end{citacao}

Outras estimativas demonstram que uma em cada quatro garotas e um em cada seis garotos experienciaram alguma forma de abuso na infância \apud{SANDERSON2005}{HABIGZANG2012}.

A visibilidade do fenômeno no Brasil se deu a partir da década de 1980, período em que também se travavam lutas pela defesa e garantia dos direitos da criança e do adolescente, presentes na Constituição Federal de 1988 (artigo 227)\footnotemark e no Estatuto da Criança e do Adolescente, de 1990 \apud{FERRARI2002}{ESBER2007}. De acordo com \citeonline{ESBER2009}, nesse contexto de transformação e de consolidação de leis voltadas à criança, destaca-se uma sequência de marcos históricos, a níveis internacional e nacional, que culminaram no enfrentamento da violência sexual por meio de políticas públicas. O primeiro marco se constitui na garantia dos direitos de crianças e adolescentes respaldadas legalmente com a  Declaração de Genebra \citeyear{ISTCA1924}, a Declaração Universal dos Direitos da Criança \citeyear{ONU1959} e o próprio ECA. O segundo marco se refere aos movimentos sociais como o Feminista, o Movimento Nacional de Meninos e Meninas de Rua e o Comitê de Enfrentamento à Violência Sexual contra Crianças e Adolescentes. O terceiro é a produção de estudos sobre a violência sexual contra crianças e adolescentes que muito têm contribuido para o aprofundamento e a elucidação sobre o fenômeno. Por fim, o quarto marco, como uma resposta aos marcos anteriores, se refere às políticas sociais de enfrentamento. Dentre elas está o Plano Nacional de Enfrentamento da Violência Sexual Infanto-Juvenil \citeyear{MJ2001}, criado pelo \citeauthor{MJ2001}, juntamente com a sociedade civil. 

\footnotetext{É dever da família, da sociedade e do Estado assegurar à criança, ao adolescente e ao jovem, com absoluta prioridade, o direito à vida, à saúde, à alimentação, à educação, ao lazer, à profissionalização, à cultura, à dignidade, ao respeito, à liberdade e à convivência familiar e comunitária, além de colocá-los a salvo de toda forma de negligência, discriminação, exploração, violência, crueldade e opressão.}

Embora a luta contra a violência sexual no Brasil tenha sido travada há décadas, percebe-se que o problema está longe de ser resolvido. As notícias das diferentes expressões de abuso sexual em crianças nos levam ao espanto, ao desconforto e à constatação do horror desses fatos. De acordo com pesquisas, somente 10\% dos casos são relatados ou conseguem chegar ao sistema judiciário criminal'' \apud{SANDERSON2005}{HABIGZANG2012}. Isso significa que o abuso acontece mais do que temos conhecimento, ou seja,  ainda que sejam notificadas algumas ocorrências, muito ainda permanece oculto.
	
As situações do abuso, em geral, são permeadas pelo silêncio da criança.  Muitas vezes, ele pode ser decorrente do medo do que venha ocorrer caso a denúncia  seja feita. \citeonline[p. 17]{BASS1983} discute a respeito das causas desse silêncio:

\begin{citacao}
	Mesmo que não haja o envolvimento de força física, toda vez que uma criança é usada sexualmente [\ldots], há coerção. Uma criança se submete a isso por diversas razões: tem medo de magoar os sentimentos [\ldots]; quer e precisa de afeto e esta é a única maneira que lhe é oferecida; teme que, se resistir, o homem\footnotemark a machucará, ou irá se vingar em alguém que ela ama; ou então que irá dizer que ela é quem estava querendo e, assim, lhe causará problemas; a criança é pega de surpresa e não tem a menor ideia do que fazer; o homem lhe diz que aquilo é certo, que está ensinando-a, que todo mundo também faz; ela aprendeu a obedecer aos adultos; acha que não tem outra escolha.
\end{citacao}

\footnotetext{Embora a autora priorize os casos de abuso sexual infantil em que o homem é o autor, o Código Penal já abrange a autoria de mulheres nesse crime.}

Com relação à diferenciação dos casos de abuso sexual infantil, de acordo com o âmbito de sua ocorrência, eles podem ser classificados em intrafamiliares\footnotemark -- aqueles que ocorrêm no próprio contexto familiar -- e extrafamiliares, que ocorrem externamente a esse meio. No primeiro contexto, 

\footnotetext{Também conhecido como abuso sexual incestuoso ou abuso sexual doméstico.}

\begin{citacao}
	por se tratar de uma esfera privada, [\ldots] encontra-se envolvido por esta atmosfera de segredo, podendo ter a complacência de outros membros da família; muitas vezes o abusador é, inclusive, o provedor econômico da casa \cite[p. 9]{JUNG2006}
\end{citacao}

Esta noção de espaço privado que a família assumiu na modernidade pode dificultar qualquer tipo de exposição e de possíveis avaliações externas. Normalmente, da família se espera proteção e cuidado aos filhos, não se aceitando assim contradições. Para \apudonline{OLIVEIRA1989}{JUNG, 2006}, é no âmbito familiar que o abuso sexual se inicia e se mantém mais facilmente, justamente pelo silêncio da vítima, decorrente da cumplicidade e da autoridade dos pais que lhe são impostas. 

Nas ocorrências extrafamiliares, os abusadores podem ser pessoas conhecidas da vítima e próximas de sua família ou totalmente desconhecidas e, nesses casos, a criança está sujeita a sofrer uma série de ameaças diretas a ela e/ou à sua família, o que facilita para que se permaneça o silêncio diante do ocorrido \cite{JUNG2006}. Outros fatores que podem colaborar para esse silenciamento são a culpa e a vergonha estimuladas pelo abusador na criança e o descrédito de alguns adultos diante do relato por parte dela. De acordo com \apudonline[p. 15]{GABEL1997}{JUNG2006}, ``a criança tem medo de falar e, quando o faz, o adulto tem medo de ouvi-la''. 

Esses fatos nos permitem refletir, dentre outras coisas, sobre a dificuldade em aproximar a criança de discussões que envolvam a sexualidade. À criança foi atribuído lugar de ser histórico e sujeito de direitos ao longo do tempo, o que surtiu e vem surtindo efeitos práticos em diversas áreas, contudo, não totalmente no âmbito da sexualidade. Debates revelam que, sob uma perspectiva dos direitos humanos, os direitos sexuais da criança continuam marcados pela ideia de proteção tutelar, dominação e excepcionalidade. Isso significa que seus direitos sexuais não são reconhecidos como direitos de fato, o que contribui para a manutenção do status ``castrador e adultocêntrico'' das discussões e intervenções públicas no campo da sexualidade. Para \citeonline[p. 74]{NETO2009}:

\begin{citacao}
	A garantia dos direitos sexuais de crianças e adolescentes deve ser considerada como uma proteção a seu direito à vida, competindo aos Estados partes assegurarem ao máximo a `sobrevivência e o desenvolvimento da criança' (CDC\footnotemark, Artigo 6, 1-2) e adotarem medidas apropriadas para `protegê-las contra todas as formas de abuso e exploração sexual' (CDC, Artigo 34, 1).
\end{citacao}

\footnotetext{Convenção das Nações Unidas sobre os Direitos da Criança.}

Discussões no sentido de se tentar garantir os direitos sexuais das crianças e adolescentes passam pela tentativa de superar a visão e a invocação da ``inocência da criança'', que têm contribuído para manter sua proteção tutelar como instrumento de intervenção  estatal reificadora e castradora \apud[p. 74]{ENNEW1986}{NETO2009}.

O abuso sexual infantil refere-se a práticas que ferem a dignidade e os direitos da criança, principalmente no tocante à sua sexualidade. Para o Ministério da Saúde, violência sexual é ``toda ação na qual uma pessoa, em situação de poder, obriga outra à realização de práticas sexuais contra a vontade, através da força física, da influência psicológica ou do uso de armas ou drogas''. Segundo a Organização Mundial da Saúde, o abuso sexual infantil acontece quando há ``o envolvimento de uma criança em atividade sexual que ele ou ela não compreende completamente, é incapaz de consentir, ou para a qual, em função de seu desenvolvimento [\ldots] não está preparada e não pode consentir'' \cite{OMS1999}. Além disso, a situação de abuso sexual infantil envolve a violação às leis ou tabus da sociedade. 

Atualmente, o abuso sexual infantil é avaliado no Código Penal, no Estatuto da Criança e do Adolescente, dentre outros (vide anexo). O ECA dispõe que a criança e o adolescente gozam de ``todos os direitos fundamentais inerentes à pessoa humana''  (art. 3), sendo que nenhuma delas ``será objeto de qualquer forma de negligência, discriminação, exploração, violência, crueldade e opressão'' (art. 5), estando sujeito à punição o descumprimento desses preceitos.

Para \apudonline{HOLMES1997}{TRINDADEBREIER2007}, o pano de fundo para o abuso sexual infantil é a desproporção de poder e conhecimento que existe entre adulto e criança. A criança espera de um adulto proteção, cuidado e condições mínimas para seu desenvolvimento, seja físico ou psicológico, o que caracteriza a infância como etapa peculiar do ciclo vital em que necessidades temporárias se fazem presentes \cite{TRINDADEBREIER2007}, não podendo essas, no entanto, se tornar fatores que justifiquem ou legitimem qualquer forma de abuso por parte dos mais velhos.

O \citeonline{MJ1997} aponta que, devido à sua complexidade, a compreensão do abuso sexual de crianças precisa se dar através da análise de todo um contexto histórico, cultural, jurídico, econômico e político. A sociedade brasileira é marcada por desigualdades constituídas não apenas pela dominação de classes, mas também de gênero e raça, e é marcada por um autoritarismo presente nas relações entre adulto e criança. Para \citeonline[p. 198]{PFEIFFERSALVAGNI2005}: 

\begin{citacao}
	Em todos os tempos, o domínio do mais forte sobre o mais fraco foi exercido sob as diversas formas de poder, nas diferentes esferas da sociedade, desde as políticas estatais, às sociais e familiares. A essa relação de poder, de busca dos excessos, do diferente e até mesmo do anormal, soma-se a pouca importância dada às crianças e aos adolescentes e às consequências dos maus-tratos dos adultos sobre eles. Dessa forma, mesmo com a evolução dos princípios morais e legais em defesa das crianças e adolescentes, os casos de abuso sexual não deixaram de acontecer, nem passaram a ser vistos de maneira uniforme pela sociedade como um crime que deixa sequelas, muitas vezes irreparáveis.
\end{citacao}

A problematização do abuso sexual infantil está cada vez mais presente em jornais e pesquisas acadêmicas, o que não o torna um tema esgotado, tampouco isento de desatenção e da necessidade de maior aprofundamento. Além disso, o abuso sexual ainda é repleto de mitos e situações que ao invés de o desvelarem   apenas reforçam a discriminação e sua naturalização. Aguns desses mitos são: a crença de que o abuso sexual infantil ocorre apenas em classes mais pobres ou somente em meninas, ou, ainda, que todos que o praticam são pedófilos e que vítimas de abuso sexual são violadores sexuais em potencial \cite[p. 177]{ALENCAR2009}. \citeonline[p. 80]{CAMPOS2002} destaca mais alguns, tais como a ideia de que o ``abusador é um psicopata'' com características facilmente reconhecíveis ou que o abuso sexual é, normalmente, uma ação isolada, ``contemplada num único ato violento e envolvendo somente a conjunção carnal'' (p. 81).

O abuso sexual pode ser classificado como verbal, que é aquele em que o contato físico não ocorre, e como abuso sexual com contato físico. O primeiro tipo abrange práticas como o exibicionismo, o assédio sexual, telefonemas obscenos, voyeurismo e manuseio de material pornográfico infantil. O segundo tipo pode se dar por atentado violento ao pudor, que consiste em forçar a criança a praticar algum ato, por estupro ou violação, por incesto, por prostituição infantil dentre outros \cite{VIEIRA2006}.

Opiniões quanto à capacidade de a criança se defender em situações de iminência do abuso sexual são divergentes. Alguns acreditam que por aprenderem desde cedo o que é socialmente aceitável com relação a seus genitais e de outrem, as crianças podem ser capazes de perceber as ações de risco \cite{BRINOWILLIAMS2008}. Dessa forma, é de extrema importância considerar que se instituições como escola e família fizerem seu papel nessa tarefa de instrução e orientação, a possibilidade de a criança poder discernir o perigo é maior. \citeonline[n.p.]{BRINOWILLIAMS2008} afirmam que ``o abuso sexual pode ser prevenido se as crianças forem capazes de reconhecer o comportamento inapropriado do adulto, reagir rapidamente, deixar a situação e relatar para alguém o ocorrido''.  

Em se tratando de instituições escolares, seus profissionais possuem outra importante tarefa diante da suspeita de abuso: a identificação e notificação das ocorrências. O grande contato e o vínculo a que as crianças estão submetidas com seus educadores permitem o fácil reconhecimento de sinais de violência que ela possa estar sofrendo. Para tanto, é fundamental que os profissionais da educação estejam familiarizados com o assunto do abuso sexual para que mais facilmente reconheçam e contribuam na interrupção do ciclo de violência \cite{CHILDHOODBRASIL2009}.

Os efeitos do abuso sexual podem atingir muitos aspectos da vida da vítima, tais como ``psicológicos, físicos, comportamentais, acadêmicos, sexuais e interpessoais'' \apud[p. 70]{DAY2003}{HABIGZANG2012}.  Para \citeonline{PFEIFFERSALVAGNI2005} é indiscutível seu grande impacto na saúde física e/ou mental da criança, com marcas no desenvolvimento e danos que podem perdurar por toda a vida. Para \apudonline{SANCHEZ1997}{TRINDADEBREIER2007} e \apudonline{MATTOS2002}{TRINDADEBREIER2007}, as consequências traumáticas decorrentes do abuso sexual podem vir a curto ou a longo prazo e se relacionam a múltiplos fatores, tais como:

\begin{citacao}
	idade da criança na época do abuso sexual; duração e frequência; grau de violência ou ameaça; diferença de idade entre a pessoa que cometeu o abuso e a vítima; proximidade da relação entre abusador e vítima; ausência de figuras parentais protetoras e o grau de segredo e de ameaças contra a criança; reação dos outros; dissolução da família depois da revelação; criança se responsabilizando pela interação sexual; perpetrador negando que o abuso aconteceu; considerados agravantes para o desenvolvimento de reações negativas à experiência de abuso sexual \cite[p. 70]{FURNISS1993, KAPLANSADOCKGREBB1997, SANDERSON2005}{HABIGZANG2012}.
\end{citacao}

Embora existam argumentos a respeito da influência direta do abuso sexual no comportamento da criança, tais como ``a apresentação de condutas sexualizadas'', ``sentimentos de culpa, fracasso ou dificuldades escolares'' dentre muitas outras, \apudonline[p. 79]{SANCHEZ1997}{TRINDADEBREIER2007} afirma que nem sempre se pode estabelecer relação direta de causa-efeito entre a ocorrência do abuso e comportamentos posteriores da criança. Ainda assim, é possível e necessário se atentar para alguns indícios que a criança venha apresentar, decorrentes de uma ou mais situações de abuso. Para \apudonline{IPPOLITO2003}{JUNG2006}, as crianças dão aviso de que estão sendo vítimas de abuso sexual de diversas formas, em sua maioria não-verbais. Salienta-se que não se deve considerar apenas um sinal isoladamente, mas por meio de um cruzamento com outros dados.

A criança vítima de abuso sexual pode apresentar apenas sintomas psicológicos e, além disso, o abuso nem sempre ocorre de forma agressiva ou violenta, o que inclui atitudes de leves toques, beijos, promessas de presente e de atenção, o que pode contribuir para o surgimento de um sentimento dúbio na criança e a impressão de que ela possa ter consentido com o ato \cite{RAMOS2009}. 

Para \apudonline{KAPLANSADOCKGREBB1997}{HABIGZANG2012}, não se associa diretamente nenhum sintoma psiquiátrico específico decorrente do abuso sexual, no entanto há maior risco de crianças que sofreram abuso sexual desenvolverem problemas interpessoais e psicológicos, a curto ou a longo prazo, em comparação com outras crianças da mesma idade que não tenham passado por isso. Em uma análise de \apudonline{CARMENMILLS}{JUNG2006}, constatou-se que 43\% dos pacientes psiquiátricos por ele observados apresentaram história anterior de abuso sexual na infância. Portanto, é pertinente se atentar para a possibilidade de correlação existente entre o abuso e tais consequências.

As consequências físicas do abuso sexual variam de  pequenas cicatrizes, traumas físicos na região genital, doenças sexualmente transmissíveis, até danos cerebrais permanentes e morte \apud{FERREIRASCHARAMM2000}{HABIGZANG2012}. Outras consequências físicas podem ser os distúrbios de sono, da alimentação, baixo controle dos esfíncteres e até aumento ou perda de peso afim de parecer menos atraente aos olhos do abusador \apud{IPPOLITO2003, SANTOS1991, VITIELLO1989}{JUNG2006}.

Os efeitos psicológicos vão desde a ``baixa autoestima até desordens psíquicas severas'' \apud[p. 71]{FERREIRASCHARAMM2000}{HABIGZANG2012}, tais como depressão, sentimento de culpa e de vergonha, ansiedade social, transtorno do pânico, distúrbios de conduta, transtorno dissociativo, transtorno de estresse pós-traumático (TEPT) etc \apud{HETZELRIGGINBRAUSCHMONTGOMERY2007}{HABIGZANG2012}. Podem também ser observados sintomas de ``déficit de atenção, hipervigilância e distúrbios de aprendizado'' \apud[p. 71]{SANDERSON2005}{HABIGZANG2012}. É importante destacar que, a longo prazo, muitos desses efeitos podem levar a vítima a possuir ideias de suicídio, à tentativa ou à própria consumação do ato \apud{SANCHEZ1991}{JUNG2006}.

No âmbito afetivo, conforme \citeonline{JUNG2006} afirma, as crianças vítimas de abuso sexual costumam experienciar sentimentos de auto-desvalorização, culpa e depressão. Na esfera sexual, uma das principais afetadas pelo abuso sexual, os efeitos se dão em forma de ``medo da intimidade'', se manifestando pela recusa de qualquer relacionamento sexual ou dificuldade em manter relacionamento sexual satisfatório \apud{AZEVEDO1989}{JUNG2006}. 

Sejam notificadas ou não, as situações de abuso sexual infantil ocorrem cotidianamente, a despeito da idade, do sexo, da raça e dos níveis social, cultural e econômico da criança. Atualmente se tem contado com a mídia na denúncia e notícia a respeito de tais casos. Embora se tenha noção da importância do papel que ela exerce, é importante se atentar para a forma como isso vem sendo efetivado, muitas vezes sensacionalista, causando assim ``revitimização'' e danos secundários às vítimas desse tipo de violência \cite[p. 95]{KUNG2009}. 

Diante da suspeita do abuso sexual contra crianças, todos têm como dever a denúncia às autoridades policiais, que daí em diante serão responsáveis por investigar o caso. A notícia de um abuso sexual deve ser sempre investigada, o que significa considerar que se precisa cumprir uma série de protocolos que vão desde a necessidade de a vítima se submeter a exames médicos para obtenção de evidências até entrevistas com profissionais psicólogos, assistentes sociais, do Conselho Tutelar, da Polícia, do Ministério Público e da Justiça. Ainda, em casos mais graves, pode se fazer necessário o afastamento da criança de seu convívio familiar devido à impossibilidade de permanência próxima ao abusador. Depreende-se então que:

\begin{citacao}
	\ldots as consequências do abuso sexual contra a criança se estendem para além dos efeitos do abuso em si, conduzindo a variadas experiências estressoras capazes de provocar uma segunda vitimização. Por isso deve-se procurar reduzir a necessidade de múltiplas entrevistas, diminuir as formalidades legais e minorar a frieza dos ambientes por onde a criança precisará transitar, bem como disponibilizar um quadro de profissionais -- tanto pelo lado do direito quanto do serviço social e de saúde física e mental -- especialmente treinado e preparado para acolher a criança e evitar a sua revitimização \cite[p. 81]{TRINDADEBREIER2007}.
\end{citacao}

Reforça-se então que, diante da possível revitimização a que a criança está exposta quando é submetida a procedimentos interrogativos que verifiquem a ocorrência do abuso sexual, é de fundamental importância que profissionais que atuam diretamente com ela estejam alertas para o cuidado em suas intervenções \apud{FURNISS1993}{JUNG2006}.

Entre os psicólogos, há que se preocupar com o cuidado no questionamento dos fatos à criança, de forma que não se façam perguntas diretas sobre o abuso em si, mas questões que permitam que ela relate como está se sentindo; ou que sejam utilizados métodos lúdicos através dos quais ela poderá se expressar, testes verbais que poderão dar informação do abuso de maneira simbólica etc \cite{JUNG2006}, até visitas domiciliares, quando necessário. De acordo com \apudonline[p. 42]{MILLER1987}{JUNG2006}, ``as crianças abusadas sexualmente precisam de meios apropriados para expressar sua raiva, medo, hostilidade e outros sentimentos que possam estar inibidos ou reprimidos''. 

Apesar de todo o avanço obtido na história concernente ao lugar de criança, ela ainda é, com frequência, considerada como inferior, inábil e dependente, tendo por referência o adulto'' \cite[p. 6]{JUNG2006}. É fato que essa relação de referência e de autoridade está posta e socialmente aceita. No entanto, o aspecto de adultocentrismo valida historicamente os adultos a exercitarem seu poder sobre os mais novos num sentido assimétrico ou desigual de poder, com atitudes danosas e desregradas. 

Outro aspecto que permite analisar as relações de poder existentes em nossa sociedade é o machismo, crença de que o homem se encontra em posição superior à mulher. Considerando-o, é possível refletir sobre o porquê de ser mais frequente  a vitimização de crianças do sexo feminino em situações de abuso sexual ser mais frequente.

De forma geral, as causas do abuso sexual infantil não são precisamente descritas. Na verdade, elas compõem um conjunto de fatores sociais, econômicos, culturais, psicológicos e situacionais \cite{JUNG2006}. Diante disso, cabe a profissionais o constante estudo, além do tratamento, tanto às vítimas quanto aos abusadores, de maneira que suas ações possam contribuir para diversas mudanças tanto na esfera relacional da vítima e do abusador quanto na própria estrutura social.

% ----------------------------------------------------------
% Capítulo III
% ----------------------------------------------------------
% ==================================
% Capítulo III
% ==================================

\chapter{A Psicologia Jurídica Brasileira Frente ao Abuso Sexual Infantil: intervenções, possibilidades e limites}
\addcontentsline{toc}{chapter}{A Psicologia Jurídica Brasileira Frente ao Abuso Sexual Infantil: intervenções, possibilidades e limites}

\section{Breve histórico da constituição da Psicologia Jurídica}
\addcontentsline{toc}{section}{Breve histórico da constituição da Psicologia Jurídica}

A Psicologia possui relação com o Direito por meio de um importante ponto: a atenção ao comportamento humano. No entanto, se diferencia dele na medida em que está voltada para o mundo do ``ser'', não para o do ``dever ser'' \apud[p. 13]{RIVEROS1995}{ROVINSKI2004}.

O encontro da Psicologia com o Direito em algum momento da história se daria, por mais que eles tentassem se conservar distantes. Conforme \apudonline[p.34]{SOBRALARCEPRIETO1994}{JESUS2001},

\begin{citacao}
	a Psicologia, por um lado, procurando compreender e explicar o comportamento humano, e o Direito, por outro, possuindo um conjunto de preocupações sobre como regular e prever determinados tipos de comportamento, com o objetivo de estabelecer um contrato social de convivência comunitária. Podemos perceber, então, a complementaridade que a Psicologia pode fornecer ao Direito, sem desejar ir além do que lhe compete.
\end{citacao}

Indícios do início do estabelecimento da relação entre Psicologia e Direito estão na busca, dentro do âmbito jurídico, de se estabelecer normas para convívio entre as pessoas, conforme os preceitos de conduta. É ao século XVIII que se remontam os primeiros sinais do que viria ser a Psicologia Jurídica \cite{JESUS2001}. No final do século XIX, análises sobre o Direito e ``sua função na vida social'' foram elaboradas a partir da Psicologia e de saberes próximos a ela (p. 27). Foi nesse século que surgiu a necessidade de se aplicar a Psicologia ao Direito e isso pôde ser verificado através da publicação de várias obras.\footnotemark

\footnotetext{Dentre eles: ``A Psicologia em suas principais aplicações à administração da Justiça'' (Hoffbauer, 1808); ``Exposição Documentada de Crimes Célebres'' (Feuerbach); ``A Doutrina da Prova'' (Mittermaier, 1834); ``Manual Sistemático de Psicologia Judicial'' (1835) etc. \cite[p. 28]{JESUS2001}}

Em 1868 foi publicado o livro \emph{Psychologie Naturelle} -- de Prosper Despine, médico francês -- em que estudos de casos envolvendo grandes criminosos da época eram apresentados trazendo uma perspectiva de características psicológicas de cada um deles. A conclusão de Despine era a de que os criminosos -- com algumas exceções -- não demonstravam nenhuma enfermidade física ou mental \cite{LEAL2008}. Ainda segundo ele, as irregularidades nos comportamentos dos que praticaram os crimes estavam ``em suas tendências e seu comportamento moral'' (p. 172). Essas tendências eram frequentemente nocivas, tais como o ódio, a vingança etc. Iniciava-se aí a ``Psicologia Criminal'', denominação dada à época para as práticas da Psicologia que estudavam os aspectos psicológicos de quem cometia crimes (p. 173). 

A Psicologia Criminal passou a se ocupar principalmente com a compreensão do agir e da personalidade do criminoso, permitindo que o crime fosse tomado como um problema não apenas do criminoso, mas também de outros agentes como o Juiz, o advogado, o psicólogo, o psiquiatra e o sociólogo \apud{DOURADO1965}{LEAL2008}.

Outro autor importante na constituição do conhecimento sobre a criminalidade foi Cesare Lombroso. Psiquiatra e criador da Antropologia Criminal, ele ocupou-se da chamada ``Psicologia do Delinquente'', defendendo em seus estudos a ligação entre a criminalidade e as características físicas \cite{JESUS2001}. Ao considerar características fisiológicas e anatômicas na análise de criminosos, em detrimento de toda a complexidade de contexto e fatores, pode-se cair no simplismo e no reducionismo, beirando uma visão determinista que dá abertura à intensificação do preconceito e da estigmatização social. De acordo com \citeonline{FARIA2008}, Lombroso não conseguiu provar a relação entre as características hereditárias e a criminalidade. Prosseguindo com seus estudos, em sua maioria com homens e mulheres que já sofriam estigmas e críticas, Lombroso continuou não se atentando à importância da questão social e, por fim, suas teses acabaram somente por reforçar o preconceito, sendo que a questão racial foi a principal nesse cenário de categorização do ser humano. Segundo \citeonline[n.p.]{FARIA2008}, ``os negros eram sempre considerados menos evoluídos e mais perigosos socialmente''.

Apesar dessas condições, desde o momento em que a criminologia surgiu no cenário das Ciências Humanas para estudar os fatores determinantes da criminalidade, a personalidade e a conduta do delinquente e a maneira de ressocializá-lo, a Psicologia Criminal passa a ocupar uma posição de destaque como ciência capaz de contribuir para a compreensão da conduta e da personalidade do criminoso. Considerando que no estudo criminológico busca-se esclarecer o ato humano antissocial com o objetivo de preveni-lo, a Psicologia pôde contribuir para uma maior excelência no julgamento de cada caso, com a compreensão ou a tentativa de compreender o delinquente e as forças psicológicas que o levaram ao crime.

O século XX foi marcado por trabalhos empíricos e experimentais de psicólogos europeus envolvendo o testemunho bem como a sua participação em processos judiciais, o que contribuiu para que a Psicologia do Testemunho fosse impulsionada e os psicólogos se aprofundassem em aspectos de investigação. Concomitantemente, os estudos psicométricos utilizados em Laudos Psicológicos foram se desenvolvendo a partir da colaboração de psicólogos clínicos junto a psiquiatras, em exames psicológicos legais e nos sistemas de justiça juvenil \cite{JESUS2001}. No começo a psicologia foi muito vinculada apenas à aplicação de testes, visão que acabou limitando, de certa forma, a grande contribuição que ela poderia oferecer ao contexto jurídico.

Ainda no século XX, a influência da Psicanálise, teoria desenvolvida por Freud, colaborou para que a doença mental fosse revista e o sujeito fosse estudado de maneira mais compreensiva. Consequentemente, o psicodiagnóstico adquiriu mais respeito, abandonando um enfoque notavelmente médico e obtendo acréscimos de aspectos psicológicos \apud{CUNHA1993}{LAGO2009}. O psicodiagnóstico configurou-se como um importante instrumento na Psicologia Jurídica por sua perspectiva de orientação aos operadores do Direito, devido ao seu caráter de objetividade. Esse foi um período em que a contribuição da Psicologia esteve bastante voltada à avaliação psicológica, através da inauguração dos testes psicológicos para exames e avaliações \cite{LAGO2009}.

De uma forma geral, a Psicologia aplicada ao Direito no século XX ganhou várias determinações, a depender de seu objeto de estudo, tais como ``Psicologia Criminal'', ``Psicologia Judiciária'' e ``Psicologia Penal'' \cite[p. 234]{COSTA2009}. De acordo com \citeonline[p. 35]{JESUS2001}, a Psicologia Jurídica estabelece-se, então, como área de investigação psicológica especializada, que estuda ``o comportamento dos atores jurídicos no âmbito do Direito, da lei e da justiça''. Esse campo vem crescendo nas últimas décadas a nível nacional e internacional \cite{LEAL2008}.

Segundo \citeonline[p. 182]{LEAL2008}, as possíveis áreas de atuação na Psicologia Jurídica são:

\begin{citacao}
	Psicologia Jurídica e as Questões da Infância e Juventude (adoção, conselho tutelar, criança e adolescente em situação de risco, intervenção junto a crianças abrigadas, infração e medidas sócioeducativas); Psicologia Jurídica e o Direito de Família (separação, paternidade, disputa de guarda, acompanhamento de visitas); Psicologia Jurídica e Direito Civil (interdições, indenizações, dano psíquico); Psicologia Jurídica do Trabalho (acidente de trabalho, indenizações, dano psíquico); Psicologia Jurídica e o Direito Penal (perícia, insanidade mental e crime, delinqüência); Psicologia Judicial ou do Testemunho (estudo do testemunho, falsas memórias); Psicologia Penitenciária (penas alternativas, intervenção junto ao recluso, egressos, trabalho com agentes de segurança); Psicologia Policial e das Forças Armadas (seleção e formação da polícia civil e militar, atendimento psicológico); Mediação (mediador nas questões de Direito de Família e Penal); Psicologia Jurídica e Direitos Humanos (defesa e promoção dos Direitos Humanos); Proteção a Testemunhas (existem no Brasil programas de Apoio e Proteção a Testemunhas); Formação e Atendimento aos Juízes e Promotores (avaliação psicológica na seleção de juízes e promotores, consultoria e atendimento psicológico aos juízes e promotores); Vitimologia (violência doméstica, atendimento a vítimas de violência e seus familiares) e Autópsia Psicológica (avaliação de características psicológicas mediante informações de terceiros).
\end{citacao}

Embora o crescimento das possibilidades de atuação em Psicologia Jurídica -- a nível não apenas nacional -- e da própria importância em si dessa área, vale destacar que a carência de profissionais especializados ainda é alta. É de extrema importância que quem deseje seguir nessa direção busque qualificação, a fim de desenvolver uma reflexão crítica que colabore para a construção de uma sociedade que entenda a complexidade do indivíduo e dos determinantes que o formam, dentre muitas outras contribuições.

\section{A Psicologia Jurídica no Brasil}
\addcontentsline{toc}{section}{A Psicologia Jurídica no Brasil}

A Psicologia Jurídica no Brasil, de acordo com o Conselho Federal de Psicologia, é definida como ``uma das especialidades do psicólogo'' \cite[p. 234]{COSTA2009}. É uma área com ampla possibilidade de atuação, mas que ainda se encontra bastante associada aos processos jurídicos \apud{BONFIM1994}{COSTA2009}, realidade que profissionais buscam transformar, a fim de ampliar suas práticas e contribuir de maneira crítica.

No Brasil, a Psicologia Jurídica é uma área relativamente recente de atuação, especialmente relacionada ao século XX. Seu início não é definido por um marco histórico específico, mas afirma-se que os psicólogos jurídicos começam sua atuação em meados da década de 60, quando do reconhecimento da profissão em Psicologia \cite{LAGO2009}. Anterior a esse marco, a atuação foi gradual, informal e marcada pela ação em trabalhos voluntários.

Os psicólogos iniciaram sua atuação no contexto jurídico, prioritariamente, em Varas de Família, Cível, Criminal, da Criança e do Adolescente, elaborando laudos sob o modelo pericial \cite{COSTA2009}.

A atividade do Psicólogo na Justiça se determina por legislações específicas e previsões nos regimentos internos dos Tribunais de Justiça \cite{COSTA2009}. Na lei n{\textordmasculine} 7.210, de 17 de julho de 1984, está prevista a atuação desse profissional para o Sistema Penal Brasileiro:

\begin{citacao}
	Art. 6{\textordmasculine} A classificação será feita por Comissão Técnica de Classificação que elaborará o programa individualizador e acompanhará a execução das penas privativas de liberdade e restritivas de direitos, devendo propor, à autoridade competente, as progressões e regressões dos regimes, bem como as conversões.
	
	Art 7{\textordmasculine} A Comissão Técnica de Classificação, existente em cada estabelecimento, será presidida pelo diretor e composta, no mínimo, por dois chefes de serviço, um psiquiatra, um psicólogo e um assistente social, quando se tratar de condenado à pena privativa da liberdade \cite{BRASIL1984}.
\end{citacao}

Desde 2003, a redação do art. 6{\textordmasculine} sofreu alteração pela Lei n{\textordmasculine} 10.792 de 1{\textordmasculine} de dezembro de 2003, e a nova composição passou a ser: ``A classificação será feita por Comissão Técnica de Classificação que elaborará o programa individualizador da pena privativa de liberdade adequada ao condenado ou preso provisório'' \cite{BRASIL2003}. 

As ações do psicólogo na justiça encontram embasamento no Código de Ética Profissional de 27 de agosto de 2005. O código indica, em \emph{Princípios Fundamentais}, que o trabalho do psicólogo será baseado ``no respeito e na promoção da liberdade, da dignidade, da igualdade e da integridade do ser humano, apoiando nos valores que embasam a Declaração Universal dos Direitos Humanos'' e visará a promoção de saúde e de qualidade de vida das pessoas e das coletividades, contribuindo para a eliminação de quaisquer formas de ``negligência, discriminação, exploração, violência, crueldade e opressão'' \cite[p. 27]{LIMA2008}. 

Atualmente um campo de atuação em ascensão para os psicólogos é o Direito Civil. Esse direito refere-se ao Direito Privado Comum, que disciplina o estado e a capacidade das pessoas e suas relações, de caráter privado, relativos à ``família'', às ``coisas'', às ``obrigações'' e à ``transmissão hereditária dos patrimônios'' \apud[p. 485]{RAO1952}{MONTORO2005}. Os trabalhos podem ser realizados com famílias, em perícias psicológicas e também no Juizado de Menores. A partir de 1990, quando o Juizado de Menores passou a ser denominado ``Juizado da Infância e Juventude'', o psicólogo teve seu trabalho ampliado -- por exemplo, na área pericial; no acompanhamento e aplicação das medidas de proteção ou socioeducativas etc. \apud[p. 485]{TABAJASKIGAIGERRODRIGUES1998}{LAGO2009}.

\citeonline[p. 78]{FRANCA2004} destaca os principais setores de atuação do psicólogo jurídico no Brasil, a saber, Psicologia Criminal, Psicologia Penitenciária ou Carcerária, Psicologia Jurídica e as questões da infância e juventude, Psicologia Jurídica: investigação, formação e ética, Psicologia Jurídica e Direito de Família, Psicologia do Testemunho, Psicologia Jurídica e Direito Civil e Psicologia Policial/Militar. 

Como emprego do saber psicológico às questões relacionadas ao saber do Direito e ao exercício desse \cite{LEAL2008}, a Psicologia Jurídica muito tem auxiliado, sobretudo devido aos seus conhecimentos colocados à disposição do juiz a respeito da esfera psicológica dos agentes envolvidos em processos, análises estas que vão além da literalidade restritiva da lei e que, de outra maneira, podiam não chegar ao conhecimento do juiz \apud{SILVA2007}{LEAL2008}.

\citeonline[p. 35]{JESUS2001} divide sinteticamente algumas atribuições do psicólogo jurídico, tais como ``avaliação e diagnóstico'' no que tange às condutas psicológicas dos atores jurídicos; ``assessoramento e/ou orientação'' nas perícias em órgãos judiciais, nos casos próprios da Psicologia; ``intervenção'' junto a atores jurídicos na comunidade e no âmbito penitenciário, individual e coletivamente; ``formação e educação'', como em trabalhos de seleção e treinamento de profissionais do sistema legal; ``campanhas'' de prevenção social contra a criminalidade em meios de comunicação; pesquisa no que concerne à Psicologia Jurídica; contribuição para a melhoria da situação da vítima e sua interação com o sistema legal e ações de mediação, permitindo que, por meio de soluções negociadas dos conflitos jurídicos, o dano emocional e social seja diminuído e prevenido, e contribuindo para que se ofereça uma alternativa à via legal em que as partes tenham um papel predominante.

Dentro da Psicologia Jurídica é possível especificar três subconjuntos de áreas de atuação: A Psicologia Forense, a Psicologia Criminal e a Psicologia Judiciária. Na Psicologia Forense estão as práticas psicológicas ligadas aos procedimentos forenses, correspondendo a toda aplicação do saber psicológico a um processo ou procedimento em andamento no Foro. A Psicologia Criminal, também subconjunto da Psicologia Forense, faz o estudo das condições psíquicas do criminoso e como se origina e se processa nele a ação criminosa \apud{BRUNO1967}{LEAL2008}. A Psicologia Criminal abrange a Psicologia do delinquente, a Psicologia do delito e a Psicologia das testemunhas. A Psicologia Judiciária, também subconjunto da Forense, se relaciona a toda prática psicológica realizada ``a mando e a serviço da justiça'', ou seja, acontece sob subordinação imediata à autoridade judiciária. É na Psicologia Judiciária que está inclusa a função pericial \cite{LEAL2008}.

A perícia, do latim ``\emph{peritia}: destreza, habilidade'', refere-se ao exame de fatos ou situações, realizado por especialista na matéria que lhe é submetida, e tem por objetivo elucidar determinados aspectos técnicos \apud[p. 21]{BRANDIMILLER1996}{ROVINSKI2004}.

Na área judicial, a investigação pericial tem por objetivo clarificar situações e fatos controversos provindos de conflitos de interesses relacionados a um direito contestado. Ressalta-se que a perícia, como meio de prova, não se constituiu em uma verdade única, mas, reafirma-se, uma elucidação acerca dos fatos a fim de dar fundamento e auxílio em uma posterior conclusão que cabe somente ao Juiz \cite{ROVINSKI2004}.

O artigo 145 do Código de Processo Civil especifica quem pode exercer as atividades de perito. Tendo-o por base, afirma-se que está apto ao papel de perito o psicólogo devidamente regulamentado ao CRP\footnotemark e que tenha capacidade técnica para responder, concernente à matéria de psicologia, às questões formuladas em juízo. Através de seu órgão de classe, a função de perito pelo psicólogo também fica regulamentada. No Decreto 53.964 (21.01.64), que regulamenta a Lei 4.112 --  responsável pela criação da profissão de psicólogo -- a situação de realizar perícia e emitir pareceres sobre a matéria de Psicologia é prevista \cite{ROVINSKI2004}.

\footnotetext{ Embora no Brasil já exista o reconhecimento da área de Especialização em Psicologia Jurídica, não se exige esse título para a atuação em perícias judiciais. Basta o psicólogo estar regulamentado pelo CRP do qual faz parte. Tal fator, no entanto, não impede que o psicólogo busque conhecimento acerca do que for investigar e do sistema em que vai operar \cite{ROVINSKI2004}.}

\section{Práticas, possibilidades e limites frente ao abuso sexual infantil}
\addcontentsline{toc}{section}{Práticas, possibilidades e limites frente ao abuso sexual infantil}

Considerando a forte aproximação que se deu nos últimos anos entre Psicologia e Direito, observa-se que a Psicologia Jurídica, um resultado interdisciplinar, aos poucos vem se constituindo essencial na garantia da justiça. Na medida em que o reconhecimento de sua contribuição a partir de suas avaliações substanciais e cientificamente embasadas continuar ocorrendo, a possibilidade de mudanças mais efetivas ocorrerão \apud{STEIN2009}{PELISOLIGAVADELLAGLIO2011}. Mas para isso é importante que o profissional exerça constante reflexão acerca de sua prática, para não incorrer simplesmente em realizações rotineiras de avaliação e elaboração de laudos, ocultando assim determinações dos acontecimentos, avaliando as pessoas de forma superficial ou taxando-as através dos instrumentos e laudos científicos.

O abuso sexual como um problema complexo de ordem mundial exige a interlocução de diversos saberes, dentre eles a Psicologia e o Direito. De acordo com \citeonline[n.p.]{GRIGOLATTOROVERATTI2009}, 

\begin{citacao}
	A interação entre a Psicologia e o Direito na área de Abuso sexual de crianças e adolescentes é de extrema importância, uma vez que a psicologia como mediadora entre a criança e o contexto judiciário participa da trajetória sócio-histórica da infância brasileira, exigindo do profissional um compromisso ético e uma boa qualidade de escuta, trazendo à realidade toda complexidade da criança.
\end{citacao}

Com frequência os casos de abuso sexual que chegam ao sistema jurídico precisam ser avaliados e verificados. A partir da alegação de abuso sexual infantil, é necessário que algumas etapas sejam cumpridas como ``investigação da suspeita, medidas de proteção para a criança e ação legal'' etc \apud[n.p.]{OATES2000}{PELISOLIGAVADELLAGLIO2011}. Diante disso, profissionais de diversas categorias, atuantes no contexto jurídico, têm importantes contribuições. 

Por sua vez, o psicólogo jurídico, pode intervir nessas etapas de algumas maneiras, a depender do objetivo que lhe é proposto. De uma forma geral e sucinta, ele tem o papel de auxiliar o juiz durante o processo judicial, participando da produção de conhecimento acerca dos fatos psicólogos das pessoas envolvidas -- sejam estas a vítima, seus pais/responsáveis ou o abusador, que pode inclusive ser um dos pais ou responsável pela criança. Isso porque em grande parte das denúncias de abuso sexual infantil, não se dispõe de provas materiais e de testemunhas que confirmem  sua ocorrência, sem falar da possível ausência de vestígios físicos no corpo da criança. De acordo com \apudonline{IPPOLITO2003}{JUNG2006}, em apenas 30\% dos casos se observam evidências físicas do abuso. Além disso, o agressor está sempre disposto a negar seu crime. 

Diante disso, um dos principais papeis do psicólogo jurídico nessa área é o de perito. A perícia psicológica forense

\begin{citacao}
	pode ser definida como o exame ou avaliação do estado psíquico de um indivíduo com o objetivo de elucidar determinados aspectos psicológicos deste; este objetivo se presta à finalidade de fornecer ao juiz ou a outro agente judicial que solicitou a perícia, informações técnicas que escapam ao senso comum e ultrapassam o conhecimento jurídico \cite[p. 36]{JUNG2006}.
\end{citacao}

No processo de avaliação de uma criança possivelmente abusada sexualmente, realizam-se a descrição de personalidade, uma análise da repercussão dos fatos no psiquismo e a compreensão dos acontecimentos. A partir disso, é possível traçar um esboço do retrato psicológico da criança e refletir sobre a necessidade e a possibilidade de encaminhamento a um tratamento psicoterapêutico \apud{VIAUX1997}{JUNG2006}, afinal, a ajuda às crianças vítimas de abuso sexual não inclui somente o diagnóstico e a punição ao agressor. 

O psicólogo jurídico, além da entrevista clínica, pode se utilizar de instrumentos psicológicos, como os testes -- especialmente os projetivos -- e um conjunto de técnicas lúdicas a fim de conseguir dados por parte da criança. De acordo com \apudonline[p. 37]{SILVA2003}{JUNG2006}, os instrumentos utilizados devem abranger ``métodos e materiais adequados, destinados a analisar e avaliar aspectos referentes à estrutura da personalidade, à cognição, à dinâmica e à afetividade das pessoas envolvidas''. Em entrevistas com os pais ou responsáveis é possível detectar aspectos que auxiliam na suposição ou não da vitimização bem como na possibilidade de avaliar gravidade e frequência. 	Através do trabalho de avaliação pericial, o psicólogo pode levantar tanto a confirmação da ocorrência do abuso quanto da gravidade das alterações psicológicas e do dano psíquico \cite{JUNG2006}. 

Como um meio de prova no contexto forense, a materialização da perícia se dá através do Laudo Pericial. Segundo \citeonline[p. 37]{JUNG2006},

\begin{citacao}
	O laudo pericial, que será apreciado pelo agente jurídico que o solicitou, deve ser redigido em linguagem clara e objetiva, para que possa efetivamente fornecer elementos que auxiliem a decisão judicial, devendo responder ao quesito solicitado, que, neste caso, concretiza-se numa pergunta do tipo: `há indícios de que esta criança foi vítima de abuso sexual?'
\end{citacao}

É importante ressaltar que, no contexto jurídico, como o foco é determinado pelo sistema legal, o objetivo final da avaliação forense será o de responder a uma questão legal expressa pelo agente jurídico, através da compreensão psicológica do caso. Portanto, o psicólogo está, de certa forma, sujeito e limitado ao que lhe foi solicitado. Considerando isso, para \apudonline[p. 43]{MELTON1997}{ROVINSKI2004}, aspectos clínicos como ``diagnóstico'' ou ``necessidade de tratamento'' ficam para segundo plano, em relação a outros de relevância legal no caso. A atitude do psicólogo perito deve ser de maior objetividade, ``afastamento'', e neutralidade, atentando-se para que o processo de avaliação forense não seja transformado em um ambiente terapêutico \cite[p. 44]{ROVINSKI2004}.	

Dessa forma, por mais que o processo de avaliação psicológica no marco legal não se diferencie substancialmente daquele que ocorre no contexto da clínica, é fundamental que adequações dos procedimentos às normas sejam feitos, já que os contextos não são os mesmos e a forma como o ``cliente'' se apresenta a esse momento é diferente \cite[p. 42]{ROVINSKI2004}.

Embora esteja pautado em uma demanda do juiz, é fundamental que o psicólogo jurídico reflita constantemente sobre a possibilidade de, através de sua atuação, não reproduzir apenas técnicas a fim de se chegar a um conhecimento e elucidação maior dos fatos. Mas que atue também por meio de um olhar amplificado, que considere o sujeito dentro de um contexto social complexo ao qual se vincula a violência por ele sofrida, contribuindo por fim para facilitar ou promover mudanças na esfera individual do mesmo. De acordo com uma pesquisa de \apudonline{SANTOS2004}{SANTOS2009}, o papel do psicólogo no contexto jurídico pode ir além da tecnicidade de uma perícia, atingindo múltiplas possibilidades e gerando mudanças, entendendo estas não como ``o controle social que assujeita o indivíduo a uma norma ou padrão estabelecido, mas a mudança indicada pela própria família em seu desejo de resolução de conflitos'', mudança por meio de autoconhecimento e da competência própria para transformar e dar significado ao seu ato, tornando-se sujeito e autor de si mesmo e da própria história. 

\citeonline[p. 8]{SANTOS2009} realiza reflexões que apontam para a necessidade de que o Psicólogo Jurídico mantenha constantemente uma postura crítico-reflexiva e que se atente  

\begin{citacao}
	para o risco de que a intervenção do sistema de proteção à criança e ao adolescente reproduza o mesmo modelo de controle social que vigorava na época do Código de Menores. Por outro lado, indicam também que o psicólogo ao se isentar e se ausentar de cenas significativas no estabelecimento de um Estado de direito e justo, estará permitindo a continuidade de relações sociais desiguais, eticamente condenadas. Pois, se uma intervenção pode seguir modelos vigentes, promovendo e corroborando injustiças, a \emph{não intervenção} também não deixa de ser uma resposta que da mesma forma pode manter, favorecer e legitimar injustiças sociais.
\end{citacao}

Dessa forma, é fundamental que o psicólogo desenvolva uma escuta diferenciada, capaz de promover acolhimento devido e necessário à criança vitimizada e a suas famílias. Para isso, ``é preciso favorecer uma relação de empatia e solidariedade onde o profissional possa se reconhecer nela, na mesma condição humana de vulnerabilidade e desproteção'' \cite[p. 25]{SANTOS2009}. É a partir daí que a criança estará apta não somente a compartilhar informações sobre os fatos, mas a transmitir seus sentimentos. 

Essa diferenciação na forma de atuação em perícias não é permanente entre os psicólogos, até porque se constitui em um exercício e uma busca de atenção diferenciada constantes, a que nem todos os profissionais psicólogos no contexto jurídico estão acostumados, seja pela carência de orientação anterior sobre esse contexto e suas possibilidades diante do social, seja pelo engessamento que o sistema judiciário aos poucos promove no profissional etc. \apudonline[p. 3]{ARANTES2007}{SANTOS2009} realiza discussões nesse sentido, apontando algumas reflexões de psicólogos jurídicos insatisfeitos ``com sua atuação restrita às avaliações, com a fragilidade epistemológica desse campo de conhecimento e com a falta de autonomia profissional''.

\citeonline[p. 19]{SANTOS2009} reflete sobre a importância de um atendimento diferenciado e propõe ao Psicólogo Jurídico, mesmo submetido ao limite da determinação judicial:

\begin{citacao}
	um modo de intervenção [\ldots] que possibilite não apenas uma avaliação pericial, mas a construção de um espaço conversacional, no qual possam emergir os significados construídos e constituir novos significados. Mais do que apenas ouvir a criança, sem contextualizar sua narrativa como uma oportunidade para o resgate do direito restitutivo e protetor em complementaridade ao direito da regulação. 

	Acredita-se que, ao buscar uma compreensão das condições emocionais, relacionais e sociais dos envolvidos, com o objetivo de se conhecerem quais direitos deverão ser reparados, o estudo psicossocial tem o potencial de oferecer às crianças e adolescentes e a seus familiares as devidas condições de emancipação e de mudança do contexto de risco.
\end{citacao}

Por fim, ressalta-se que para que o contexto jurídico possa ser propício para transformações, e não somente decisões, mudanças precisam ocorrer tanto na formação do psicólogo quanto na dos operados do Direito. São necessárias também modificações nas concepções da Justiça através das quais não se preocupe prioritariamente em regular as relações entre os cidadãos, mas que se voltem para o cuidado e cidadania das pessoas \cite{COSTA2009}. 

Embora ainda haja muito que ser feito e aperfeiçoado entre os psicólogos no meio Jurídico, é importante também considerar que os limites em sua intervenção existem -- como em qualquer outra profissão -- a fim de que o ``psicologismo'' seja evitado, comprometendo o espaço já consolidado do profissional psicólogo \cite[p. 34]{JESUS2001}.

% ----------------------------------------------------------
% Considerações Finais
% ----------------------------------------------------------
% ==================================
% Considerações Finais
% ==================================

\chapter*[Considerações Finais]{Considerações Finais}
\addcontentsline{toc}{chapter}{Considerações Finais}

O objetivo do presente trabalho foi realizar um breve estudo sobre a maneira como a Psicologia Jurídica brasileira vem atuando frente aos casos de abuso sexual infantil bem como sobre alguns limites e possibilidades com os quais esse campo pode se deparar.

Para tanto foi fundamental a anterior compreensão sobre alguns elementos. O primeiro deles envolveu saber como a criança moderna se constituiu e que fatores engendraram mudanças no modo de ver e de considerá-la ao longo dos séculos. Abrangeu-se também a constituição do interesse pela criança no Brasil que, embora tenha sofrido influência do contexto europeu, teve suas próprias características. 

É importante ressaltar sobre as mudanças na forma de tratar a criança, decorrentes do processo de cristianização dos costumes e, ainda, o seu afastamento de questões sexuais -- inclusive, e principalmente, das próprias práticas sexuais a que eram submetidas na sociedade europeia --, o que, ao longo dos séculos e relacionados a outros fatores, se traduziu num contexto repleto de contradições, tabus e controvérsias, com desdobramentos cruéis, sobretudo à criança.

Quanto ao caminho histórico percorrido no primeiro capítulo a fim de contextualizar e demarcar o lugar que a infância ocupa nos tempos modernos, ressalta-se que sua complexidade é bem maior, tendo sido exposto aqui somente um resumo sobre os fatos e não o aprofundamento deles. Uma sistematização e um detalhamento maiores permitiriam realizar questionamentos e reflexões mais críticas diante dos acontecimentos históricos. No entanto, objetivava-se apenas uma noção das mudanças pelas quais a concepção de infância e/ou a própria criança sofreu.

Com relação à infância no Brasil, focou-se na questão do contexto jurídico e do tratamento da criança nesse contexto a fim de ressaltar a diferença social na sociedade brasileira que marginaliza e explora crianças, conferindo-lhes e reforçando lugares que não lhes cabem como ser em desenvolvimento, detentor de direitos humanos e proteção garantida por lei. A exposição realizada permitirá um profundo estudo futuro sobre os aspectos políticos, sociais, econômicos e culturais que envolvem essa realidade do “menor” -- a criança marginalizada no Brasil --, objetivando repensar as leis voltadas à criança e como podia ser dar uma mudança efetiva em nossa estrutura social.

Referente à sexualidade infantil, buscou-se a contribuição das teorias de Freud e de Foucault, embora elas estejam envoltas de debates e controvérsias, por terem colaborado para a ideia de que a sexualidade é inerente aos indivíduos, portanto, separar os dois é cindir com o próprio sujeito, dando vazão ao exercício do domínio externo sobre ele de maneira funesta.

O segundo aspecto que foi necessário compreender se refere ao abuso sexual infantil. Por meio do estudo realizado foi possível depreender que ele não se constitui em um fato isolado tampouco decorre de um fator apenas, mas se relaciona aos processos sociais ao longo da História como um todo, chegando a atingir níveis consideráveis de ocorrência, desprezando os direitos da criança e contribuindo para que se perpetue um estigma de subjulgo da mesma ao poder do adulto. Foi possível refletir ainda sobre a força com que esse fenômeno se mantém, não conseguindo as punições previstas em lei conter a atuação dos abusadores. Diante disso, cabe analisar de forma mais profunda os aspectos que engendram essa perpetuação da violência infantil.

É importante ressaltar que no que concerne às consequências do abuso priorizou-se por não apresentar uma visão determinista, embora se tenha clara a noção de que as consequências para a vida da criança abusada inegavelmente existem e lhe sobrevêm. Além disso, é necessário pontuar que não foram abrangidas todas as consequências do abuso sexual infantil.

Uma provável discussão a ser feita se refere à possibilidade de, mesmo diante de uma denúncia, o abuso sexual não ter ocorrido; são os casos de falsa denúncia do abuso sexual infantil. A literatura já traz algumas discussões nesse sentido, que podem ser interessantes para profissionais que desejam atuar diretamente com a denúncia desse tipo de violência infantil.

De maneira geral, observou-se que é extensa a literatura sobre o abuso sexual infantil, o que se constituiu em uma importante contribuição para profissionais e para a própria população como um todo que deseje conhecer ou se aprofundar no tema. Em contrapartida, verificou-se que a bibliografia brasileira que trata especificamente do abusador -- embora ele não tenha sido o foco do capítulo deste trabalho -- é escassa, sendo, portanto, necessário que mais pesquisas nessa área sejam realizadas.

Por fim, o terceiro ponto  refere-se às práticas, limites e possibilidades atuais da Psicologia Jurídica brasileira frente ao abuso sexual infantil. Antes de qualquer consideração, há que se ressaltar sobre a carência de estudos aprofundados sobre o histórico da Psicologia Jurídica, sobretudo a brasileira. 

Observou-se que existem divergências de opiniões quanto a consolidação das práticas do psicólogo jurídico, o que indica a necessidade de se configurar melhor o lugar desse profissional nas várias áreas dentro do Sistema. Notou-se ainda frequentes questionamentos a respeito da carência de preparo durante a graduação para a atuação do psicólogo no âmbito jurídico. Raros são os cursos de graduação em Psicologia que oferecem a disciplina de Psicologia Jurídica, seja de forma obrigatória ou optativa \cite{ROVINSKI2004}.

Sobre a atuação específica do psicólogo jurídico em casos de abuso sexual infantil que chegam ao Sistema Judiciário, percebeu-se que se encontra de maneira quase completa permeada pelo exercício da perícia psicológica ou estudo psicossocial, portanto, constituída das práticas de avaliações psicológicas e psicodiagnósticos. Há que se perceber a contribuição que essa atuação tem trazido, no sentido de poder colaborar para a proteção da vítima e para seu desenvolvimento psicossocial. 
 
É importante ressaltar que se faz extremamente necessária a qualificação profissional específica dos psicólogos jurídicos em sua área de interesse bem como a busca de conhecimento sobre o próprio sistema em que atua: a Justiça. Além da qualificação específica, é absolutamente fundamental o desenvolvimento da escuta diferenciada, caso se pretenda ir além de uma simples perícia, percebendo e acolhendo o indivíduo em sua totalidade, possibilitando, a partir dos contatos periciais, a mudança de significados e a autonomia deste. Essa forma de atuar é um desafio atual para a Psicologia Jurídica em casos de abuso sexual infantil. Isso porque o que é mais comum é a simples conclusão de perícias psicológicas, materializadas por meio do Laudo Pericial, após as quais não se observa mudanças na vida dos envolvidos. Constitui-se em um desafio por ir contra o que está posto e estabelecido ao próprio cargo de psicólogo jurídico como perito. É claro que aqui não se fala de se desconsiderar o âmbito de trabalho ou de fugir às atribuições do próprio cargo -- inclusive porque isso se estabeleceria como irresponsabilidade profissional --, mas se trata de ir além de barreiras muitas vezes impostas não pelo sistema judiciário, mas pela própria categoria profissional; trata-se também de vencer certo conformismo e pessimismo presente na sociedade quanto à (in)eficiência de qualquer atuação no Jurídico e, por fim, trata-se de relembrar o aspecto e a condição humana que vêm junto com a vítima no momento da perícia.

Referente à perícia psicológica de crianças abusadas sexualmente existe uma gama de instrumental psicológico específico, além de entrevista clínica estruturada, que aqui não se abordou de forma minuciosa. Seria interessante a produção de mais estudos abrangendo a especificidade e a descrição desse material a fim de auxiliar no conhecimento de psicólogos jurídicos que estejam iniciando sua prática nesse âmbito.

Um ponto importante que precisa de atenção é a forma como se dá a atuação do psicólogo jurídico, assim como de qualquer profissional nessa área, tendo em vista que a criança possivelmente passou ou passará por uma série de procedimentos de verificação do abuso sexual. Muitas vezes esses procedimentos sequer possuem conexão entre si, só reforçando ainda mais o sofrimento e a revitimização. Há casos em que o tempo entre a ocorrência do abuso sexual e sua denúncia é extenso, portanto, retomar todos os fatos e trazer à tona os sentimentos da criança pode ser mais árduo ainda para ela, daí a importância de um trabalho integrado, associado e minimizador do sofrimento e da revitimização.

É importante mencionar que não foi abordado nesse capítulo o chamado \emph{Depoimento sem Dano}, técnica recente que objetiva a tomada especial do depoimento da criança vítima de abuso sexual infantil. Apenas salienta-se que ao psicólogo atualmente essa prática está vedada, conforme decisão do Conselho Federal de Psicologia. Debates sobre esse tema foram e ainda vêm sendo realizados e as controvérsias são grandes. Em virtude dessa suspensão da atuação do psicólogo e da considerável discordância de opiniões, priorizou-se por não torná-lo ponto de discussão aqui. No entanto, isso não significa que as discussões devam cessar, pelo contrário, é preciso que se chegue num argumento definitivo sobre a contribuição ou não do psicólogo jurídico nessa técnica, em benefício final da criança.

De tudo, depreende-se que a Psicologia Jurídica ainda tem muito a contribuir e a desenvolver em sua atuação como um todo e em casos específicos de abuso sexual infantil. O profissional desse contexto, não apenas psicólogo, não pode se esquecer de que o cliente que lhe chega não está isolado, mas inserido em um contexto complexo, em uma rede de relações sociais sujeitas a todo instante a alterações; é um ser particular, com sua individualidade, detentor de vontades e também de direitos e permeado por valores e crenças. Sendo assim, é possível pensar em medidas de reorganização, atribuição de novos significados à experiência, prevenção e outros conjuntos de ações que possam contribuir para restabelecer não somente o sentido à criança, mas à relação familiar e até as próprias relações entre adultos e crianças.

% ----------------------------------------------------------
% ELEMENTOS PÓS-TEXTUAIS
% ----------------------------------------------------------
\postextual


% ----------------------------------------------------------
% Referências bibliográficas
% ----------------------------------------------------------
\bibliography{bibliografia}

% ----------------------------------------------------------
% Glossário
% ----------------------------------------------------------
%
% Consulte o manual da classe abntex2 para orientações sobre o glossário.
%
%\glossary

% ----------------------------------------------------------
% Apêndices
% ----------------------------------------------------------

% ---
% Inicia os apêndices
% ---
%\begin{apendicesenv}

% Imprime uma página indicando o início dos apêndices
%\partapendices

%\end{apendicesenv}
% ---


% ----------------------------------------------------------
% Anexos
% ----------------------------------------------------------

% ---
% Inicia os anexos
% ---
\begin{anexosenv}

% Imprime uma página indicando o início dos anexos
\partanexos

\chapter{Estatuto da Criança e do Adolescente}

Art. 3{\textordmasculine} A criança e o adolescente gozam de todos os direitos fundamentais inerentes à pessoa humana sem prejuízo da proteção integral de que trata esta Lei, assegurando-se-lhes, por lei ou por outros meios, todas as oportunidades e facilidades, a fim de lhes facultar o desenvolvimento físico, mental, moral, espiritual e social, em condições de liberdade e de dignidade.

Art 4{\textordmasculine} É dever da família, da comunidade, da sociedade em geral e do Poder Público assegurar, com absoluta prioridade, a efetivação dos direitos referentes à vida, à saúde, à alimentação, à educação, ao esporte, ao lazer, à profissionalização, à cultura, à dignidade, ao respeito, à liberdade e à convivência familiar e comunitária. 

Art. 5{\textordmasculine} Nenhuma criança ou adolescente será objeto de qualquer forma de negligência, discriminação, exploração, violência, crueldade e opressão, punido na forma da lei qualquer atenção, por ação ou omissão, aos seus diretos fundamentais.

\chapter{Convenção Internacional dos Direitos da Criança}

Artigo 16. Nenhuma criança será objeto de interferências arbitrárias ou ilegais em sua vida particular, sua família, seu domicílio ou sua correspondência, nem de atentados ilegais a sua honra e reputação. [\ldots] A criança tem direito à proteção da lei contra essas interferências ou atentados.

Artigo 19. 1. Os Estados-partes adotarão todas as medidas legislativas, administrativas, sociais e educacionais apropriadas para proteger a criança contra todas as formas de violências física ou mental abuso ou tratamento negligente, maus tratos ou exploração, inclusive abuso sexual, enquanto a criança estiver sob a custódia dos pais, do representante legal ou de qualquer outra pessoa responsável por ela.

\chapter{Código Penal}

Art. 217- A. Ter conjunção carnal ou praticar outro ato libidinoso com menor de 14 (catorze anos):
Pena – reclusão, de 8 (oito) a 15 (quinze) anos. 

Art. 218. Induzir alguém menor de 14 (catorze) anos a satisfazer a lascívia de outrem:
Pena – reclusão, de 2 (dois) a 5 (cinco) anos. 

Art. 241-D. Aliciar, assediar, instigar ou constranger, por qualquer meio de comunicação, crianças, com o fim de com ela praticar ato libidinoso:
Pena – reclusão, de 1 (um) a 3 (três) anos, e multa. 

\chapter{Constituição Federal de 1988}

Art. 227. É dever da família, da sociedade e do Estado assegurar à criança e ao adolescente, com absoluta prioridade, o direito à vida, à saúde, à alimentação, à educação, ao lazer, à profissionalização, à cultura, à dignidade ao respeito, à liberdade e à convivência familiar e comunitária, além de colocá-los a salvo de toda forma de negligência, discriminação, exploração, violência, crueldade e opressão. 

§ 4{\textordmasculine}. A lei punirá severamente o abuso, a violência e a exploração sexual da criança e do adolescente.

\end{anexosenv}

%---------------------------------------------------------------------
% INDICE REMISSIVO
%---------------------------------------------------------------------

\printindex

\end{document}
