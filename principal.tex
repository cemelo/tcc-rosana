% ------------------------------------------------------------------------
% ------------------------------------------------------------------------
% Trabalho de Conclusão de Curso
% Bacharelado em Psicologia
% Faculdade de Educação
% Universidade Federal de Goiás
% ------------------------------------------------------------------------
% ------------------------------------------------------------------------

% verso e anverso:
%\documentclass[12pt,openright,twoside,a4paper,english,french,spanish]{abntex2}

% apenas verso:	
\documentclass[12pt,oneside,a4paper,english,french,spanish]{abntex2} 


% ---
% PACOTES
% ---

% ---
% Pacotes fundamentais 
% ---
\usepackage{cmap}				% Mapear caracteres especiais no PDF
\usepackage{lmodern}			% Usa a fonte Latin Modern	
%\usepackage{fourier}		
\usepackage[T1]{fontenc}		% Seleção de códigos de fonte.
\usepackage[utf8]{inputenc}		% Determina a codificação utiizada (conversão automática dos acentos)
\usepackage{makeidx}            % Cria o indice
\usepackage{hyperref}  			% Controla a formação do índice
\usepackage{lastpage}			% Usado pela Ficha catalográfica
\usepackage{indentfirst}		% Indenta o primeiro parágrafo de cada seção.
\usepackage{color}				% Controle das cores
%\usepackage{graphicx}			% Inclusão de gráficos
% ---
	
% ---
% Pacotes adicionais, usados apenas no âmbito do Modelo Canônico do abnteX2
% ---
\usepackage{lipsum}				% para geração de dummy text
% ---

% ---
% Pacotes de citações
% ---
\usepackage[brazilian,hyperpageref]{backref}	 % Paginas com as citações na bibl
\usepackage[alf]{abntex2cite}	% Citações padrão ABNT

% --- 
% CONFIGURAÇÕES DE PACOTES
% --- 

% ---
% Configurações do pacote backref
% Usado sem a opção hyperpageref de backref
\renewcommand{\backrefpagesname}{Citado na(s) página(s):~}
% Texto padrão antes do número das páginas
\renewcommand{\backref}{}
% Define os textos da citação
\renewcommand*{\backrefalt}[4]{
	\ifcase #1 %
		%Nenhuma citação no texto.%
	\or
		%Citado na página #2.%
	\else
		%Citado #1 vezes nas páginas #2.%
	\fi}%
% ---

% ---
% Informações de dados para CAPA e FOLHA DE ROSTO
% ---
\titulo{A Psicologia Jurídica Frente ao Abuso Sexual Infantil:
  Intervenções, Possibilidades e Limites}
\autor{Rosana Rabelo}
\local{Goiânia}
\data{Março 2013}
\orientador{Juliana de Castro Chaves}
\instituicao{%
  Universidade Federal de Goiás - UFG
  \par
  Faculdade de Educação
  \par
  Bacharelado em Psicologia}
\tipotrabalho{Trabalho de Conclusão de Curso}
% O preambulo deve conter o tipo do trabalho, o objetivo, 
% o nome da instituição e a área de concentração 
\preambulo{Trabalho apresentado à banca examinadora da
  Faculdade de Educação da Universidade Federal de Goiás
  - UFG como exigência parcial para obtenção do grau de
  Bacharel em Psicologia, sob orientação da Prof. Dra.
  Juliana de Castro Chaves.}
% ---

% alterando o aspecto da cor azul
\definecolor{blue}{RGB}{41,5,195}

% informações do PDF
\hypersetup{
     	%pagebackref=true,
		pdftitle={\imprimirtitulo}, 
		pdfauthor={\imprimirautor},
    	pdfsubject={\imprimirpreambulo},
		pdfkeywords={psicologia}{juridica}{praticas}{abuso}{sexual}{infantil},
	    pdfproducer={LaTeX}, 	% producer of the document
	    pdfcreator={\imprimirautor},
    	colorlinks=false,       		% false: boxed links; true: colored links
    	linkcolor=blue,          	% color of internal links
    	citecolor=blue,        		% color of links to bibliography
    	filecolor=magenta,      		% color of file links
		urlcolor=blue,
		bookmarksdepth=4
}
% --- 

% --- 
% ESPAÇAMENTOS ENTRE LINHAS E PARÁGRAFOS
% --- 

% O tamanho do parágrafo é dado por:
\setlength{\parindent}{1.3cm}

% Controle do espaçamento entre um parágrafo e outro:
\setlength{\parskip}{0.2cm}  % tente também \onelineskip
\makeindex

% ----
% INÍCIO DO DOCUMENTO
% ----
\begin{document}
\frenchspacing 

% ----------------------------------------------------------
% ELEMENTOS PRÉ-TEXTUAIS
% ----------------------------------------------------------
% \pretextual

\imprimircapa
\imprimirfolhaderosto*

% ---
% FICHA CATALOGRÁFICA
% ---

% Isto é um exemplo de Ficha Catalográfica, ou ``Dados internacionais de
% catalogação-na-publicação''. Você pode utilizar este modelo como referência. 
% Porém, provavelmente a biblioteca da sua universidade lhe fornecerá um PDF
% com a ficha catalográfica definitiva após a defesa do trabalho. Quando estiver
% com o documento, salve-o como PDF no diretório do seu projeto e substitua todo
% o conteúdo de implementação deste arquivo pelo comando abaixo:
%
% \begin{fichacatalografica}
%     \includepdf{fig_ficha_catalografica.pdf}
% \end{fichacatalografica}
\begin{fichacatalografica}
	\vspace*{\fill}					% Posição vertical
	\hrule							% Linha horizontal
	\begin{center}					% Minipage Centralizado
	\begin{minipage}[c]{12.5cm}		% Largura
	
	\imprimirautor
	
	\hspace{0.5cm} \imprimirtitulo  / \imprimirautor. --
	\imprimirlocal, \imprimirdata-
	
	\hspace{0.5cm} \pageref{LastPage} p. ; 30 cm.\\
	
	\hspace{0.5cm} \imprimirorientadorRotulo~\imprimirorientador\\
	
	\hspace{0.5cm}
	\parbox[t]{\textwidth}{\imprimirtipotrabalho~--~\imprimirinstituicao,
	\imprimirdata.}\\
	
	\hspace{0.5cm}
		1. Psicologia Jurídica.
		2. Abuso sexual infantil.
    3. Práticas.
		I. Juliana de Castro Chaves.
		II. Universidade Federal de Goiás.
		III. Faculdade de Educação.
		IV. Título\\
	
	\hspace{8.75cm} CDU 02:141:005.7\\
	
	\end{minipage}
	\end{center}
	\hrule
\end{fichacatalografica}
% ---

% ---
% FOLHA DE APROVAÇÃO
% ---

% Isto é um exemplo de Folha de aprovação, elemento obrigatório da NBR
% 14724/2011 (seção 4.2.1.3). Você pode utilizar este modelo até a aprovação
% do trabalho. Após isso, substitua todo o conteúdo deste arquivo por uma
% imagem da página assinada pela banca com o comando abaixo:
%
% \includepdf{folhadeaprovacao_final.pdf}
%
\begin{folhadeaprovacao}

  \begin{center}
    {\ABNTEXchapterfont\large\imprimirautor}

    \vspace*{\fill}\vspace*{\fill}
    {\ABNTEXchapterfont\bfseries\Large\imprimirtitulo}
    \vspace*{\fill}
    
    \hspace{.45\textwidth}
    \begin{minipage}{.5\textwidth}
        \imprimirpreambulo
    \end{minipage}%
    \vspace*{\fill}
  \end{center}
    
  Trabalho aprovado. \imprimirlocal, 05 de março de 2013:

  \assinatura{\textbf{\imprimirorientador} \\ Orientadora}
  \assinatura{\textbf{Liliane Domingos Martins} \\ Avaliadora}
      
  \begin{center}
    \vspace*{0.5cm}
    {\large\imprimirlocal}
    \par
    {\large\imprimirdata}
    \vspace*{1cm}
  \end{center}
  
\end{folhadeaprovacao}
% ---

% ---
% Dedicatória
% ---
\begin{dedicatoria}
    \vspace*{\fill}
    \centering
    \noindent
    \textit{A fazer.} \vspace*{\fill}
\end{dedicatoria}
% ---

% ---
% Agradecimentos
% ---
\begin{agradecimentos}
  A fazer.
\end{agradecimentos}
% ---

% ---
% Epígrafe
% ---
\begin{epigrafe}
    \vspace*{\fill}
	\begin{flushright}
		\textit{A pesquisar.}
	\end{flushright}
\end{epigrafe}
% ---

% ---
% RESUMOS
% ---

% ---
% Resumo
% ---
\begin{resumo}
 %Segundo a \citeonline[3.1-3.2]{NBR6028:2003}, o resumo deve ressaltar o
 %objetivo, o método, os resultados e as conclusões do documento. A ordem e a extensão
 %destes itens dependem do tipo de resumo (informativo ou indicativo) e do
 %tratamento que cada item recebe no documento original. O resumo deve ser
 %precedido da referência do documento, com exceção do resumo inserido no
 %próprio documento. (\ldots) As palavras-chave devem figurar logo abaixo do
 %resumo, antecedidas da expressão Palavras-chave:, separadas entre si por
 %ponto e finalizadas também por ponto.
 A fazer.

 \vspace{\onelineskip}
    
 \noindent
 \textbf{Palavras-chaves}: psicologia jurídica. abuso sexual infantil. práticas.
\end{resumo}

% ---
% Abstract
% ---
\begin{resumo}[Abstract]
  \begin{otherlanguage*}{english}
    This is the english abstract.

    \vspace{\onelineskip}
 
    \noindent 
    \textbf{Keywords}: juridical psychology. child sexual abuse.
  \end{otherlanguage*}
\end{resumo}
% ---

% ---
% Criação do Sumário
% ---
\pdfbookmark[0]{\contentsname}{toc}
\tableofcontents*
\cleardoublepage
% ---

\mainmatter

% ----------------------------------------------------------
% Introdução
% ----------------------------------------------------------
% ==================================
% Introdução
% ==================================

\chapter*[Introdução]{Introdução}
\addcontentsline{toc}{chapter}{Introdução}

A Psicologia Jurídica -- campo em que os conhecimentos de Psicologia são diretamente aplicados aos assuntos relacionados ao Direito -- é campo relativamente recente de atuação dos profissionais psicólogos, sobretudo no Brasil. A nível internacional, os primeiros indícios da relação entre a Psicologia e o Direito se deram por volta do século XVIII através da busca, no âmbito jurídico, de se estabelecer normas para o convívio entre as pessoas conforme os preceitos de conduta. Foi no final do século XIX que a necessidade da Psicologia aplicada ao Direito se firmou, sendo consolidada com a publicação de diversas obras sobre o assunto \cite{jesus01}.

Em termos gerais, essas duas disciplinas se aproximam pela atenção ao comportamento humano e se diferenciam na medida emque uma se volta para o mundo do ``ser'' e a outra para o do ``dever ser'' \apud{riveros95}{rovinski04}.

No Brasil, a constiuição da Psicologia Jurídica se deu mais tardamente, aproximadamente em meados do século XX, no contexto do reconhecimento da profissão em Psicologia, na década de sessenta. Os psicólogos iniciaram sua atuação de forma prioritária em Varas de Família, Cível, Criminal, da Criança e do Adolescente, elaborando laudos sob o modelo pericial \cite{costa09}. Primeiros  registros de trabalhos de psicólogos em instituições de Justiça no Brasil datam das décadas de setenta e oitenta.

A atuação do psicólogo jurídico brasileiro é regulamentada em legislações específicas e reconhecida pelo Conselho Federal de Psicologia. Muitos são os setores possíveis de intervenção do psicólogo no contexto jurídico, sendo alguns deles a Psicologia Criminal, a Psicologia Jurídica e Direitos de Família, a Psicologia do Testemunho etc. 

Observa-se que uma importante contribuição do psicológo à justiça é no sentido de, através de seus estudos, formular conhecimento acerca da esfera psicológica de agentes envolvidos em processos, colocando-o à disposição do juiz como forma de auxílio às futuras decisões ou sentenças do mesmo \apud{silva07}{leal08}.

Atualmente um trabalho que com frequência tem sido solicitado ao psicólogo jurídico é o de perícia psicológica. A perícia refere-se ao exame de fatos ou situações, praticado por especialista na matéria que lhe é submetida, e tem por objetivo elucidar determinados aspectos técnicos \apud{brandimiller96}{rovinski04}. As solicitações podem ser por motivos de guarda de crianças, interdições, verificação da ocorrência de abuso sexual infantil etc.

O abuso sexual infantil constitui-se em uma das diversas formas de violência contra crianças. Várias são as maneiras como ele pode ser praticado, portanto sua descrição é extensa. No entanto, de uma forma geral, é definido, segundo a Organização Mundial da Saúde, como a participação de crianças em atividades não compatíveis com a sua idade e com as quais ela não está apta a consentir ou compreender completamente.  O fenômeno atinge meninas e meninos de todas as classes econômicas, seja no contexto familiar ou externo a ele. Baseado nisso, ele pode ser classificado como intrafamiliar ou extrafamiliar.
	
O abuso é permeado pelo aspecto do poder, ou seja, o abusador sempre se encontra em uma posição de autoridade com relação à criança, o que lhe incentiva a utilizar-se dela para atingir seu objetivo de gratificação sexual. 
	
As consequências do abuso sexual são múltiplas na vida da criança, podendo atingir as áreas física, psicológica, afetiva, relacional e comportamental, a curto ou a longo prazo. 

Um componente recorrente nas situações de abuso sexual infantil é o silêncio da vítima, por detrás do qual está, principalmente, o medo da criança do que possa ocorrer se o fato vier à tona. 

Pensar o abuso sexual infantil, sua recorrência, as formas crueis de que ocorrem e as relações de poder nele enredadas permite constatar que os direitos humanos da criança não vêm sendo efetivados, mas negligenciados, sobretudo os direitos sexuais.  

Nem sempre na historia esses direitos existiram. Aliás, nem sempre houve infância, tal como a concebemos na modernidade. O sentimento de infância, com as noções de ingenuidade e inocência, surgiu, foi modificado e determinado por uma complexidade de fatores sociais, culturais, políticos e econômicos e por movimentos culturais iluministas e religiosos protestantes da sociedade europeia dos séculos XVII e XVIII \cite{aries81}. Até o século XVII, as crianças eram consideradas como miniaturas do homem, sendo misturadas no meio de adultos e com eles se envolvendo em todos os tipos de práticas, inclusive as sexuais \cite{aries11}. 
	
A família moderna, que teve o início do seu estabelecimento na burguesia do século XVIII, instalou padrões de intimidade e de vida privada, que culminava na união sentimental entre casal e entre os pais e filhos \cite{aries11} e uma nova maneira de agir frente a questões sexuais, implementando uma série de fatores de controle e de educação da criança. 

No Brasil a infância adquire importância a partir do século XVI, sofrendo a influência do processo que ocorria no contexto europeu, mas diferenciando-se desse na medida em que atribuía à criança outros status, provindos, principalmente, de um contexto de preconceito, exploração, abandono e pobreza no Brasil da época – contexto esse que propiciou para que os direitos e o reconhecimento à criança fossem pensados e elaborados. 

Este trabalho pretendeu produzir conhecimentos acerca das práticas da Psicologia Jurídica brasileira frente ao abuso sexual infantil, abrangendo também suas possibilidades e limites. Para tanto, se fez necessário compreender alguns aspectos do fenômeno. Antes disso, foi preciso ainda entender e contextualizar a ``criança'' de que se fala. Isso porque nem sempre ela existiu, foi vista e tratada como é hoje. 

Compreendendo então que o conceito de criança da modernidade foi construído e moldado historicamente a partir de uma gama de mudanças nas sociedades, até que a ela foram atribuídos status de sujeito dependente e em condição especial de desenvolvimento, é possível realizar reflexões sobre a transformação na forma das relações entre crianças e adultos que legitima o abuso de poder e de autoridade por parte desses.

Entendendo também que a violência sexual é um problema social multifatorial e que cada vez mais tem sido noticiado, é possível ponderar algumas práticas diante dela atualmente.

Com base no interesse na Psicologia Jurídica como uma área crescente de atuação do psicólogo, no estágio realizado nesse contexto durante o último ano do curso, bem como pesquisas e o próprio contato com uma perícia psicológica em que uma criança supostamente havia sido abusada sexualmente por um membro de sua família, pretendeu se aprofundar no assunto, por meio de revisão da literatura, a fim de levantar dados sobre a contribuição da Psicologia Jurídica brasileira diante da suspeita e/ou da ocorrência de abuso sexual infantil. Além disso, algumas possibilidades e limites com os quais ela se depara nessa atuação. 

O trabalho está estruturado em três capítulos. O primeiro delineia o caminho que a criança percorreu na história, a partir do século XIII, que lhe concedeu o status que possui na modernidade. Esse capítulo se subdivide em dois pontos: A Constituição da Infância e A Infância no Brasil. O segundo capítulo abrange os aspectos gerais do Abuso Sexual Infantil e o terceiro das práticas da Psicologia Jurídica brasileira diante desse fenômeno, estando subdivido em três pontos: Breve Histórico da Constituição da Psicologia Jurídica, A Psicologia Jurídica no Brasil e Práticas e Limites Frente ao Abuso Sexual Infantil.

A título de esclarecimento, utilizou-se aqui as nomenclaturas \emph{violência sexual infantil} e \emph{abuso sexual infantil} como equivalentes. 


% ----------------------------------------------------------
% Capítulo I
% ----------------------------------------------------------
% ==================================
% Capítulo I
% ==================================

\chapter{Infância e Suas Transformações}

\section{Constituição da infância}

Muitas vezes a infância é naturalizada, sendo abordada como um estágio da vida que ``sempre existiu'', concebida da mesma maneira em tempos históricos diferenciados e tratada, portanto, de forma homogênea pela sociedade. Alguns delimitam essa fase da vida pela idade, cronologia ou agem como se o fenômeno fosse suficientemente conhecido, o que indica a necessidade de aprofundamento de estudos e de esclarecimentos. É por existirem visões generalistas e naturalizadas sobre a infância, como analisa \citeonline{FROTA2007}, que se ressalta a importância da problematização sobre esse período da vida no sentido de se tentar compreender dilemas novos ou que se recriam ao longo do tempo. Antes disso, é fundamental buscar as raízes do que atualmente chamamos de Infância. Ela sempre existiu como tal? Se não, como era abordada? Quais elementos da materialidade histórica que podem nos auxiliar a compreender a infância na atualidade?

\citeonline{ALMEIDA2004} apontam que as delimitações das fases da vida se apoiam em visões de homem, de mundo e em valores e normas vigentes da sociedade. Embora em cada sociedade existam concepções predominantes de criança, ainda é possível encontrar diferenciações na forma como ela é tratada e, até certo ponto, na maneira como seus direitos são efetivados, a depender da classe social a que pertence.

A criança, por volta do século XIII na sociedade europeia, não era concebida como um ser completo, mas dotada de propriedades limitadas, chegando a ser considerada algumas vezes como uma forma reduzida de homem \cite{ARIES2011}. A partir de determinantes concretos e históricos, baseados em novas exigências sociais e em estudos, essa visão foi se modificando e o tratamento e o cuidado com as crianças se efetivando de novas maneiras. 

Anterior ao século XII não se conhecia a infância como um período de vida diferenciado dos demais ou não se tentava representá-la nas obras, pelo fato provável de que seu espaço na sociedade ainda não estivesse estabelecido\footnotemark, ou ainda devido ao fato de ela ser considerada um período de passagem na vida da pessoa, uma fase que rapidamente seria transposta e esquecida \cite{ARIES2011}.

\footnotetext{Uma ``miniatura otoniana'' do século XI traz a ideia da modificação imposta aos corpos da criança. Ela retrata a cena do Evangelho em que Jesus pede que se deixe vir a ele as criancinhas. Foram agrupadas em torno de Jesus oito ``verdadeiros homens'', apenas retratados numa escala menor, sem nenhuma das particularidades da infância \cite[p. 17]{ARIES2011}.}

Do século XIV ao século XVII, essencialmente nas artes, houve a predominância da ilustração da infância de forma religiosa. Nesses séculos constam, inicialmente, obras da figura do menino Jesus, da Nossa Senhora menina e, posteriormente, outras infâncias sagradas, com as figuras de São João, São Tiago, Maria Zebedeu e Maria Salomé. Segundo \citeonline[p. 19]{ARIES2011}, ``com a maternidade da Virgem a tenra infância ingressou no mundo das representações'', o que influenciou na representação de cenas de famílias em que as crianças já apareciam desenhadas com traços ternos e ingênuos, ainda que em forma de homens em miniatura. 

No século XV surgiu a representação da criança nua -- o \emph{putto} -- que perdurou mais fortemente no século XVII. Anterior a esse período, quase nunca se representava a criança de forma nua, especialmente o Menino Jesus, que só apareceu desnudado no final da Idade Média, próximo ao século XV. O gosto por esse tipo de representação, segundo \citeonline[p. 26]{ARIES2011}, ia além do gosto pela nudez clássica e se relacionava com um ``amplo movimento de interesse em favor da infância''. Esse movimento nas artes de representação da criança de diversas formas\footnotemark eram indícios de seu reconhecimento e da saída de sua condição de anonimato \cite{ARIES1981}.

\footnotetext{Além das representações comuns, surgiu também o retrato da criança morta, denotando um sentimento que começava a surgir para com a criança, cujo falecimento já não poderia ser considerado tão tolerável, como tinha sido até então \cite{ARIES2011}.}

No século XVII, embora tenha se dado a percepção de que criança era um ser diferenciado dos demais, em alguns aspectos o tratamento dirigido a elas permanecia ainda indiferenciado com relação aos adultos. Não se omitia ou restringia-lhes assuntos e práticas relacionadas ao sexo, coisas que viriam a ser desrespeitosas e reprováveis à inocência infanto-juvenil em sociedades vindouras \cite{MOTT1998}. Pelo contrário, a iniciação sexual infantil em algumas sociedades, como a Grécia Antiga, era ``conduta normal, método pedagógico ou ritual de iniciação no mundo adulto'' \apud[p. 45]{DOVER1978}{MOTT1998}. Segundo \citeonline{MOTT1998}, alguns historiadores discutem que a necessidade do afastamento da criança das questões sexuais é recente na história ocidental. Esse afastamento pode ter relação, dentre outros fatores, com a ideia de inocência que se tornou diretamente ligada à infância à medida que a conjuntura das sociedades se alterava. A não censura de questões sexuais às crianças, existente nos séculos passados, se dava, na opinião geral:

\begin{citacao}
	Primeiro, porque se acreditava que a criança impúbere fosse alheia e indiferente à sexualidade. Portanto, os gestos e as alusões não tinham consequência sobre a criança, tornavam-se gratuitos e perdiam sua especificidade sexual -- neutralizavam-se. Segundo, porque ainda não existia o sentimento de que as referências aos assuntos sexuais, mesmo que despojadas na prática de segundas intenções equívocas, pudessem macular a inocência infantil -- de fato ou segundo a opinião que se tinha da inocência. Na realidade, não se acreditava que essa inocência realmente existisse \cite[p. 132]{ARIES1981}.
\end{citacao}

Embora essa fosse a opinião predominante, alguns educadores e moralistas da época já haviam iniciado um movimento de reconhecimento das particularidades das crianças dois séculos antes, no século XV. Esse interesse específico pela infância se pautava principalmente na preocupação com a naturalidade das práticas sexuais perante os menores. A ideia de que certos comportamentos deviam ser associados à culpa -- embora os pequenos não tivessem noção dela -- e que poderiam beirar a ``sodomia'', começou a ser difundida, principalmente pelo moralista Gerson \cite[p. 80]{ARIES2011}. Ele foi responsável por estudar o comportamento das crianças no tocante à sexualidade, a fim de que o sentimento de culpa fosse despertado nos pequenos -- de 10 a 12 anos de idade -- através dos confessores. Além disso, Gerson considerava o ato da masturbação grave, ainda que fosse uma prática comum e da qual nenhuma criança se sentisse culpada, mas acreditava que a educação deveria exercer o papel de conservá-la dos riscos da sexualidade \cite{ARIES2011}.

A preocupação de Gerson e de outros educadores e moralistas do século XV contribuiu para uma determinação do conhecimento acerca do comportamento da criança, estabelecendo a necessidade de modificações nas rotinas educacionais, em que uma nova forma de relação com as crianças se daria \cite{ARIES2011}.

É a partir do final século XVII que o significado das crianças para os adultos passa a se transformar: de objeto de paparicação e risos para objeto de preocupações disciplinares. Segundo \citeonline{ARIES2011},

\begin{citacao}
	É entre os moralistas e os educadores do século XVII que vemos formar-se esse outro sentimento de infância (…) que inspirou toda a educação até o século XX, tanto na cidade como no campo, na burguesia como no povo. O apego à infância e à sua particularidade não se exprimia mais através da distração e da brincadeira, mas através do interesse psicológico e da preocupação moral (p. 104).
\end{citacao}

Esse contexto europeu específico do século XVII foi permeado pela percepção dos adultos, através da influência dos moralistas, de que as crianças não eram capazes, por conta própria, de enfrentar a vida. Sendo assim, aos pais, gradativamente, foi incumbida a função de constituição moral e espiritual da criança enquanto que à Escolarização atribuiu-se a formação restante necessária e apropriada a uma vida adulta, conforme o pensamento vigente da sociedade \cite{MIRANDA1989}.  

É importante ressaltar, contudo, que até o século XVII, a escolarização não foi geral, ou seja, nem todas as crianças passavam pela escola. Segundo \citeonline{ARIES2011}, a separação entre criança e vida adulta, pautada no método da escolarização, estava intimamente ligada ao movimento de moralização promovida pelos reformadores católicos ou protestantes, vinculados à Igreja, às leis ou ao Estado, momento também de ascendência da burguesia. Sendo assim, às meninas de famílias não burguesas ainda eram repassados ensinamentos domésticos por suas mães e ensinamentos religiosos em conventos. Esses fatores colaboraram para que, entre estas, se mantivessem inalteradas as características de infância curta e a conduta precoce \cite{ARIES2011}. 

Essas mudanças que se deram mediante uma reorganização social em que a burguesia ascendia e impunha seus direitos determinaram a formação de uma ideia de infância que, progressivamente, adquiriu status cristalizado, absoluto e global, encobrindo todo o aspecto social embutido nas relações anteriores entre criança e adulto ou sociedade \cite{MIRANDA1989}. A nova imagem que ia se atribuindo à criança contribuiu para o estabelecimento de um modelo, que seria internalizado ou rejeitado por ela. Conforme \citeonline{MIRANDA1989}, 

\begin{citacao}
	Tanto a assimilação do modelo quanto a sua recusa são plenamente justificadas pela idéia de natureza infantil. Ideologicamente, fica legitimada a necessidade de se auxiliar a criança no seu processo de assimilação das normas e penalizar aquelas que as recusam, em nome de uma condição natural na criança.
\end{citacao}

A constituição da mentalidade moderna de infância como algo que exige cuidados diferenciados em aspectos específicos se fortaleceu por volta do final do século XVII e pôde ser manifesta em âmbitos como o das artes\footnotemark, com a representação da criança nos retratos de forma centralizada ou separada dos demais membros da família, e na educação -- cujas mudanças já remontavam do final do século XVI --, com a intolerância a livros de linguagem ou conteúdo impróprios às crianças, marcando o início do pudor na linguagem escrita \cite{ARIES2011}. Além da preocupação com o conteúdo dos livros acadêmicos, passou-se a orientar mudanças na forma com que os castigos eram dados nas escolas. Sendo assim, punições físicas que anteriormente expunham o corpo dos pequenos ocorreriam de maneira que a exposição fosse evitada \cite{ARIES1981}.

\footnotetext{Nas artes, a partir do início do século XVII, a infância retratada se forma sagrada ganhou maior importância que nos séculos anteriores, além de o Menino Jesus passar a ser retratado isoladamente em pinturas, gravuras e esculturas religiosas. \citeonline[p. 93]{ARIES2011} destaca a relação imediata que foi estabelecida ``entre essa devoção da santa infância e o grande movimento de interesse pela infância, de criação de pequenas escolas e colégios de preocupação pedagógica''.}

Além das transformações iniciadas na educação no final do século XVII e início do século XVIII, também surgia o sentimento de afeição pela criança no seio familiar burguês. De uma forma geral, a burguesia\footnotemark dos séculos XVIII e XIX experienciou a reforma moralizante, iniciada no século XVII, por meio de modificações nos costumes das famílias em diversos níveis e a instauração de ``escrúpulos'' e ``decência'' \cite[p. 77]{ARIES2011}. Sobre o modelo de família existente anterior à reforma, \citeonline[p. x]{ARIES2011} esboça:

\begin{citacao}
	\ldots tinha por missão -- sentida por todos -- a conservação dos bens, a prática comum de um ofício, a ajuda mútua quotidiana num mundo em que um homem, e mais ainda uma mulher isolados não podiam sobreviver, e ainda, nos casos de crise, a proteção da honra e das vidas. Ela não tinha função afetiva. Isso não quer dizer que o amor estivesse sempre ausente: ao contrário, ele é muitas vezes reconhecível, em alguns casos desde o noivado, mais geralmente depois do casamento, criado e alimentado pela vida em comum, como na família do Duque de Saint-Simon. Mas (e é isso o que importa), o sentimento entre os cônjuges, entre os pais e os filhos, não era necessário à existência nem ao equilíbrio da família: se ele existisse, tanto melhor.
\end{citacao}

\footnotetext{Sobretudo a inglesa e a francesa.}

Após o século XVII, o que se observa é o início de uma organização familiar em que a criança adquire importância e lugar central, saindo de seu estado de anonimato. Dessa forma, tornou-se ``impossível perdê-la ou substituí-la sem uma enorme dor'' \cite[p. xi]{ARIES2011}, o que com o passar dos tempos se traduziu em uma redução voluntária da natalidade por parte dos adultos, além da preocupação com a higiene e com a saúde física dos pequenos. Além disso, a prática do infanticídio existente até esse século sofreu redução por consequência da maior sensibilidade atribuída aos pequenos, decorrente do contexto de profunda cristianização dos costumes \cite{ARIES2011}. No combate ao infanticídio, os religiosos iniciaram um movimento de ``Sacralização da Infância'', que se baseava na relação da infância com a ideia de ``Pureza''. A elas foi atribuída a existência de alma e, ainda, iniciou-se o costume de batizá-las, antes que a morte lhes sobreviesse \cite[p. 34]{SANTOS1994}.

Outras transformações puderam ser constatadas, como a consolidação do desaparecimento da liberdade com relação às brincadeiras sexuais, marcando a consideração da infância como fase da vida ``ingênua, pura e angelical'' e ainda o prolongamento desse período da vida, ou seja, o retardamento da difusão dos pequenos no meio de adultos \cite{ARIES2011}. 

\begin{citacao}
	O sentido da inocência infantil resultou portanto numa dupla atitude moral com relação à infância: preservá-la da sujeira da vida, e especialmente da sexualidade tolerada -- quando não aprovada -- entre os adultos; e fortalecê-la, desenvolvendo o caráter e a razão. Pode parecer que existe aí uma contradição, pois de um lado a infância é conservada, e de outro é tornada mais velha do que realmente é. Mas essa contradição só existe para nós, homens do século XX (p. 91).
\end{citacao}

Além disso, o próprio ambiente doméstico sofreu alterações: antes aberto para o exterior, agora retraído e preparado para a intimidade da família, longe da rua e da vida coletiva. Em casa os pais passaram a ser modelos de identificação para as crianças. A sexualidade se restringiu ao âmbito do lar e um novo padrão para ela foi instaurado: o fim de reprodução. O que fugisse disso seria coibido. \citeonline[p. 9]{FOUCAULT1988} faz referência a esse momento na história:

\begin{citacao}
	A sexualidade é, então, cuidadosamente encerrada. Muda-se para dentro de casa. A família conjugal a confisca. E absorve-a, inteiramente, na seriedade da função de reproduzir. Em torno do sexo, se cala. O casal, legítimo e procriador, dita a lei. Impõe-se como modelo, faz reinar a norma, detém a verdade, guarda o direito de falar, reservando-se o princípio do segredo.
\end{citacao}

O que \citeonline{FOUCAULT1988} afirma, no entanto, não é que se tenha cessado o assunto do sexo a partir dessas mudanças observadas no final do século XVII. Ao contrário, passou-se a discutir muito sobre ele, mas discursos acerca do que se podia ou não fazer. O sexo não deveria mais ``ser mencionado sem prudência'' (p. 23), segundo a nova pastoral, ``mas seus aspectos, suas correlações, seus efeitos devem ser seguidos até às mais finas ramificações \ldots'' (p. 23). Assim, o sexo torna-se sitiado por uma fala que não lhe admite nenhuma opacidade:

\begin{citacao}
	A pastoral cristã inscreveu, como dever fundamental, a tarefa de fazer passar tudo o que se relaciona com o sexo pelo crivo interminável da palavra. A interdição de certas palavras, a decência das expressões, todas as censuras do vocabulário poderiam muito bem ser apenas dispositivos secundários com relação a essa grande sujeição: maneiras de torná-la moralmente aceitável e tecnicamente útil \cite[p. 24]{FOUCAULT1988}.
\end{citacao}

É importante destacar novamente que as transformações iniciadas no final do século XVII que envolveram as crianças, tais como as mudanças nos trajes, no linguajar, nas brincadeiras, nos comportamentos, nos papeis sociais e nos aspectos sexuais não abrangeram a princípio todas elas. Elas puderam ser mais facilmente percebidas nas famílias burguesas em que, além das melhores condições financeiras, os princípios religiosos e morais se faziam mais prementes, sendo, portanto, mantidos e reforçados, principalmente pela Igreja Católica. Crianças de famílias pobres, como as de camponeses e artesãos, continuavam convivendo com adultos, vestindo o mesmo tipo de roupas que eles e, no caso das meninas, aprendendo e desempenhando afazeres domésticos juntamente com suas mães \cite{ARIES2011}. Os meninos, por sua vez, aprendendo e ajudando nas atividades laborais diárias.

Considerando essa diferenciação entre as classes sociais das famílias, depreende-se que, embora a ideia de cuidado e de proteção à criança devido à atribuição de uma fragilidade fosse difundida nos mais diversos âmbitos, ela não era exercida em todas as classes sociais, já que na classe baixa as famílias não podiam deixar de contar com a ajuda dessa mão de obra na agricultura e nos trabalhos familiares\footnotemark.

\footnotetext{Desde o século XX, dada a industrialização incipiente no Brasil e agudização de certo conflito social, a presença de crianças e adolescentes em fábricas e oficinas de São Paulo tornou-se predominante, principalmente no setor têxtil \cite{MOURA1998}. Constatou-se ainda que grande parte dos trabalhadores menores se acidentava durante as atividades. Verifica-se, portanto, a diferenciação nas condições a que elas são submetidas de acordo com sua classe social, além do cuidado não exercido ou exercido de forma inadequada.}

É necessário salientar também que essa família do século XVII ainda não é a família moderna, conforme destaca \citeonline{ARIES2011}. Porque, até então, ela conserva a sociabilidade e mantém-se como um ``centro de relações sociais, a capital de uma pequena sociedade complexa e hierarquizada, comandada pelo chefe da família'' (p. 189). A família moderna, que começou a se formar a partir do século XVIII, se afasta da sociedade e se volta, principalmente, para o cuidado com o desenvolvimento de suas crianças, não se mantendo mais como um espaço aberto tal como era.

Foi no século XVIII que uma sistematização marcante da infância se deu. O filósofo suíço Jean Jacques Rousseau foi um dos que contribuíram por especificar uma diferença decisiva entre a infância e a maturidade -- denominada por ele ``idade da razão'' --, afirmando sobre a inabilidade da criança para pensar em abstrações e a limitação de seu pensamento àquilo que se pode ver ou manejar. Dentre outros trabalhos de Rousseau, encontra-se a divisão da infância em estágios, ideia que se assemelhava a Jean Piaget\footnotemark, no século XX \cite[p. 9]{GALLATIN1978}. Embora os estudos de Rousseau tenham sido, a princípio, conhecidos somente por uma camada mais favorecida, interessada em educar seus filhos e em conhecer as diferenças intelectuais entre os pequenos e os adultos, foram fundamentais para que se solidificasse posteriormente o conceito de infância na cultura ocidental.  

\footnotetext{Suiço (1896-1980) que muito contribuiu à Psicologia com estudos sobre a infância.}

No século XIX, grandes mudanças na cultura, na economia e na política marcaram as sociedades. Em muitos países -- inclusive no Brasil -- o capitalismo foi firmado e expandido, de forma que a burguesia pôde se consolidar no poder político. Nesse contexto, a concepção de infância também sofreu alterações: a criança ganhou lugar central das atenções, demandando proteção e cuidados \apud{SANTOS2007}{KULLER2009}. A instalação da vida privada e com ela a intimidade e o sentimento de união afetiva entre os membros da família finalmente se consolidou graças ao arranjo em função das necessidades do modelo capitalista, em detrimento das formas comunitárias tradicionais \cite{MIRANDA1989}.

Foi nesse século que se tornaram comuns o conceito e a sistematização dos estágios da infância iniciados no século anterior por alguns teóricos, dentre os quais Rousseau \cite{GALLATIN1978}. Dessa forma, o século XIX foi marcado pela difusão da necessidade de se compreender essa fase da vida, no que concerne ao seu desenvolvimento, demarcando melhor suas fronteiras. 

No século XX a criança foi alvo de estudos de outras ordens, dentre elas, a Psicologia. Esse interesse já remonta aos séculos anteriores quando da preocupação de Gerson com a correta educação das crianças. Observações sobre a psicologia infantil já constavam em textos do final do século XVI e do século XVII, revelando a tentativa da época de se adentrar na mentalidade das crianças para o planejamento da melhor forma de educá-las. Sobre esse assunto, \citeonline[p. 104]{ARIES2011} complementa:

\begin{citacao}
	\ldots as pessoas se preocupavam muito com as crianças, consideradas testemunhos da inocência batismal, semelhantes aos anjos e próximas de Cristo, que as havia amado. Mas esse interesse impunha que se desenvolvesse nas crianças uma razão ainda frágil e que se fizesse delas homens racionais e cristãos.
\end{citacao}

O século XX foi espaço propício para o estudo científico sobre a criança, que não era mais vista com base somente em princípios morais, mas em sua constituição como ser diferenciado. Estudos da Psicologia infantil do século XX -- denominada também ``psicologia do desenvolvimento'', ``psicologia da criança'' e ``psicologia da aprendizagem'' -- enfatizavam linguagem, motricidade, pensamento etc. \cite[p. 643]{ALMEIDA2004} e demonstravam que a infância tinha um desenvolvimento específico, diferente do desenvolvimento adulto e merecia atenção e pesquisas.

O que pôde ser observado mais fortemente com os anos e os estudos em psicologia do desenvolvimento foram a sistematização e a descrição do chamado ``ciclo vital'' \cite[p. 20]{CASTRO1998}. Modos de ser e viver da criança se tornaram determinados à medida que a ciência moderna, a Psicologia e ramos dela -- inclusive a do Desenvolvimento -- ditaram concepções de infância a partir dos estudos realizados.

Em meio às transformações que envolveram a criança no período que compreende os séculos XVII ao XIX em diante, passando pelo surgimento do que \citeonline{ARIES2011} chama de ``sentimento da infância'' até a necessidade da sistematização de seus estágios e a preocupação com a educação e a saúde nessa fase da vida, percebe-se um afastamento do mundo infantil ao que se relaciona com o mundo da sexualidade. A ideia de inocência que se tornou gradativamente associada à criança dificultou e ainda dificulta reflexões sobre qualquer assunto em que ambos -- infância e sexualidade -- se aproximem. Conforme \citeonline[p. 2]{NUNES2000} afirmam, o que ainda predomina concernente à ``sexualidade infantil'' é um contexto de ``incompreensão, a improvisação do senso comum, o repetir de preconceitos e quase sempre o descaso''.

Outra mudança fundamental -- decorrente de uma complexidade de fatores -- que também permeou esse período foi o abandono da visão de que o sexo é somente para procriação \cite[n.p.]{REZENDE2008}, portanto restrito apenas aos adultos. Dessa forma, ao assuntos sexuais foi relacionada novamente a ideia do prazer. 

Comumente a sexualidade é relacionada à genitalidade, assim como vida sexual equivalente a relação sexual \cite{BEARZOTI1994}. Sigmund Freud (1856-1939) abrange o conceito, retirando-o dessas conotações. Para ele, sexualidade vai além do ato sexual e da reprodução: trata-se de energia.

A Psicanálise desenvolvida por Freud no século XIX, na Europa, enfatiza a infância e sua relação com a sexualidade. Para ele não se pode atribuir à criança, até mesmo ao bebê recém-nascido, as noções de inocência, pureza e ausência de vício, porque ela é repleta de desejos os quais busca saciar constantemente. O reservatório a que Freud atribuiu os impulsos biológicos básicos do bebê, bem como toda sua energia como ser humano, é o ``id'' \cite{GALLATIN1978}. 

Em sua obra ``Os três ensaios sobre a teoria da sexualidade'', de 1905, o infantil é associado diretamente ao desenvolvimento pulsional. Sobre esse desenvolvimento \citeonline[p. 68]{ZAVARONI2007} afirmam: 

 \begin{citacao}
	Na elaboração de sua hipótese, sobre o desenvolvimento pulsional, Freud (1905/1980) aponta para a marca da sobreposição que se constituirá como característica do processo de subjetivação, em que os modos mais arcaicos do desenvolvimento permanecem presentes, também, na sexualidade do adulto. Assim, o adulto portará para sempre o infantil que o constituiu.
\end{citacao}

Nessa obra, Freud divide o período pré-puberal de desenvolvimento da personalidade em estágios dominados por tendências sexuais, ``essas provenientes de impulsos instintivos e não aprendidos'' que objetivam o prazer'' \cite[p. 2]{SCHINDHELM2011}, proposição que vai de encontro ao que ele denominou de ``Princípio do Prazer'', responsável por regular os processos de desenvolvimento do ser humano.

Na infância, o objeto sexual do instinto sexual se constitui no próprio corpo da criança. Durante o seu desenvolvimento, os objetos se modificam, até que a busca do prazer atinja seu nível máximo e se complete no prazer do ato sexual.

De maneira sucinta, a sexualidade, de acordo com o conceito psicanalítico, 

\begin{citacao}
	\ldots é energia vital instintiva direcionada para o prazer, passível de variações quantitativas e qualitativas, vinculada à homeostase, à afetividade, às relações sociais, às fases do desenvolvimento da libido infantil, ao erotismo, à genitalidade, à relação sexual, à procriação e à sublimação \cite[p. 117]{BEARZOTI1994}.
\end{citacao}

O que permite concluir que para ele a sexualidade é um conceito mais complexo e abrangente do que se supõe ao imaginar o simples ato sexual ou prazer provindo dele.

Freud surpreendeu a sociedade da época com sua teoria de que a vida e o comportamento de um adulto eram influenciados pelas experiências e condutas sexuais infantis, desfazendo o pensamento predominante de que a criança era uma criatura pura e inocente \cite{SCHINDHELM2011}, que não se aproxima de maneira alguma às questões sexuais.

O ``infantil'', de acordo com o conceito psicanalítico, transcende o que é da ordem da cronologia e das experiências passíveis de narração, diferindo-se assim do que é chamado de ``infância cronológica'', no sentido de que é algo que não se dá a ver, mas surge apenas no modo como o indivíduo se põe em análise \cite[p. 66]{ZAVARONI2007}.

Freud e seus estudos, embora não aceitos de imediato por irem contra a ideia de inocência tida como inerente à criança, trouxeram a reflexão acerca da função sexual não apenas ligada à reprodução, mas ao desenvolvimento da vida como um todo. De certa forma pôde contribuir para ressaltar que a sexualidade infantil existe, abrindo caminhos para que se fossem discutidas melhores maneiras de cuidado no tratamento ou contato com as crianças.  

Outro autor que discute o tema da sexualidade infantil é \citeonline{FOUCAULT1988}. Sua pretensão não é negar que o sexo tenha sido ``proibido, restringido, bloqueado, mascarado'' (p. 17), nem afirmar que a interdição do sexo é uma ilusão, mas, sim, que a ilusão está em fazer dessa interdição o elemento fundamental do que se diz do sexo desde o início dos tempos modernos. Segundo ele, é fato que a antiga liberdade com que as crianças eram expostas a qualquer assunto ou prática sexual tenha desaparecido, no entanto, isso não significa um silêncio puro e simples:

\begin{citacao}
	Não se deve fazer divisão binária entre o que se diz e o que não se diz; é preciso tentar determinar as diferentes maneiras de não dizer, como são distribuídos os que podem e os que não podem falar, que tipo de discurso é autorizado ou que forma de discrição é exigida a uns e outros. Não existe um só, mas muitos silêncios e são parte integrante das estratégias que apoiam e atravessam os discursos \cite[p. 30]{FOUCAULT1988}.
\end{citacao}

\citeauthoronline{FOUCAULT1988} traz contribuição para a reflexão sobre a hipótese repressiva pela qual a sexualidade passou no sistema social sendo relacionada às questões morais. Para ele, o indivíduo tem o prazer como algo natural e ativo, que é buscado constantemente. Portanto, da mesma forma a criança em qualquer manifestação sexual está em busca do prazer \cite{DONIZETE2010}. Se a sociedade é permeada por relações de poder que marcam a conduta das pessoas, é possível discutir acerca do poder nas relações no âmbito da sexualidade.

Esses e outros estudos ou investigações puderam contribuir para que se buscasse continuamente observar a forma com que se efetivava o cuidado e o tratamento às crianças, permitindo uma atenção e uma cautela maiores nas formas de zelo e de educação das mesmas. Além disso, motivou-se que leis e concepções de proteção à criança e de controle fossem criados ou consolidados.

\section{A infância no Brasil}

Ao mesmo tempo que sofreu determinação por muitas das transformações ocorridas na Europa, o Brasil teve sua própria organização da infância, com características que se diferiram totalmente daquela sociedade.

A criança brasileira teve sua história contada a partir do século XVI, quando do interesse da Igreja Católica, representada pela Companhia de Jesus, pelas crianças indígenas, os ``Culumins'', e a salvação de suas almas. Conforme \apudonline{PRIORE1992}{SANTOS1994} afirma, esse interesse teve influência a partir das transformações que tomaram conta da Europa, que fizeram surgir o sentimento de valorização da infância, se refletindo na Igreja Católica e em sua forma de ideologizar a criança.

Esse período no Brasil foi marcado pela missão principal de conversão das almas, o que demandou uma mudança na pedagogia a partir de métodos de disciplina como castigos, ameaças e vigilância constante. Alguns desses métodos serviam de instrumento através dos quais os padres deixavam claras as ideias de ``inferno e paraíso'' \cite[p. 37]{SANTOS1994}.

Comparando-se o sentimento de família na Europa Ocidental com o que se deu no Brasil, observa-se que até o século XIX seu desenvolvimento ainda era incipiente e, a partir de seu início, ele estava diretamente ligado às regras higiênicas prescritas pela medicina. Quando o progresso nas relações familiares no Brasil se deu, de fato, atingiu de forma maciça apenas a família branca. A criança negra não foi inclusa nesse processo da modernização da família brasileira, não lhe sendo atribuídos status de pureza, inocência e felicidade \cite{SANTOS1994}. 

No âmbito da escolarização, a infância começa a adquirir importância no Brasil, aproximadamente, em 1875, com o surgimento dos primeiros jardins de Infância baseados na proposta de Froebel\footnotemark, nos Estados do Rio de Janeiro e São Paulo. Esses jardins de Infância foram introduzidos no sistema educacional de caráter privado com o objetivo de atender às crianças filhas da classe média industrial emergente. Em 1930, após reformas jurídico educacionais, o atendimento pré-escolar passou a contar com a participação direta do setor público \cite{AHMAD2009}.

\footnotetext{Friedrich Froebel (1782-1852), alemão, foi um dos primeiros educadores a considerar a importância do início da infância na formação das pessoas.}

Com o movimento da sociedade civil e de órgãos governamentais para que o atendimento às crianças de zero a seis anos fosse amplamente reconhecido na Constituição de 1988, atingiu-se o reconhecimento da Educação Infantil como um direito da criança. É justamente nessa década que a criança começa a surgir no cenário jurídico brasileiro, não apenas no tocante à educação, mas ao reconhecimento de seus direitos e da garantia de sua proteção de uma forma geral. 

O contexto de inserção da criança no âmbito do Direito envolve uma complexidade de fatores. A visibilidade à criança no Brasil esteve sempre bastante ligada a uma realidade de confrontos em que se observaram ``trabalhos forçados, extermínio, abandono, criminalidade, prostituição, analfabetismo, sobrevida nas ruas, nas instituições, no lixo, dependência de drogas, extirpação de órgãos'' dentre outros \cite[p. 40]{SANTOS1994}. Dessa forma, entidades começaram a se levantar a fim de proteger a infância e a adolescência.

A denominação da criança no sistema jurídico serviu de instrumento que confirmasse a relação com o preconceito, o abandono e a exploração que foi associada à criança de classes mais desprovidas. Observou-se a origem do termo ``menor''. De acordo com \citeonline{LONDONO1998}, no final do século XIX e começo do século XX, esse termo aparecia recorrentemente no contexto jurídico do Brasil. Anterior a esse período, a palavra fazia referência aos limites de idade, dizendo acerca dos direitos das pessoas à emancipação ou assunção de responsabilidades civis e relacionadas à Igreja. Após a proclamação da Independência o termo ``menor'' foi utilizado por juristas na definição da idade, ``como um dos critérios que definiam a responsabilidade penal do indivíduo pelos seus atos'' (p. 130).

Ao final do século XIX, juristas brasileiros puderam constatar a presença do que se teve por ``menor'', até então, na sociedade brasileira, em crianças e adolescentes desprovidos de condições financeiras adequadas e sem o cuidado dos pais, sendo chamadas também de ``abandonadas'' (p. 135). Elas podiam ser encontradas nas ruas do centro das cidades, nas praças, em mercados e, por vezes, cometiam delitos, sendo chamadas, assim, de ``menores criminosos'' (p. 135). \citeonline[p. 135]{LONDONO1998} analisa a situação:

\begin{citacao}
	Partindo dessa definição, através dos jornais, das revistas jurídicas, dos discursos e das conferências acadêmicas foi se definindo uma imagem do menor, que se caracterizava principalmente como criança pobre, totalmente desprotegida moral e materialmente pelos seus pais, seus tutores, o Estado e a sociedade. Relacionando a origem do abandono com as condições econômicas e sociais que a modernização trouxe, os juristas, tanto no começo do século como nos anos 20 e 30, não deixaram porém de apontar a decomposição da família e a dissolução do poder paterno, como os principais responsáveis de tal situação.
\end{citacao}

Assim, percebe-se que a atenção às crianças de classes baixas se deu fortemente ligada ao fato de ela poder ser um sujeito de delinquência e não de direitos.

Em 1941 foi criado o Serviço de Assistência ao Menor -- SAM -- como exemplo da expressão de políticas públicas direcionadas à infância e à adolescência pobres no Brasil. Tornou-se, contudo, uma política de exclusão-reclusão ao retirar o ``menor'' do convívio social e encaminhá-lo a ``espaços institucionais de reclusão''. Essa política perdurou até 1964, quando a Ditadura Militar instituiu a Fundação Nacional do Bem Estar do Menor -- FUNABEM \cite[p. 54]{VASCONCELOS2002}.

Os novos status atribuídos à criança, juntamente com uma série de movimentos sociais em defesa da criança e do adolescente, que antes eram tratados como ``menor'', também contribuíram para a efetivação de políticas públicas relacionadas à criança como um cidadão de direitos que merecia uma atenção especial. Dessa forma, em 1959, a Declaração Internacional dos Direitos da Criança determinou a responsabilidade do adulto perante a criança e criou mecanismos de controle do cumprimento dessa regra. No Brasil, a Lei 8.069 do dia 13 de julho de 1990 regulamenta os direitos e a primazia de proteção a crianças e adolescentes no Estatuto da Criança e do Adolescente. Consoante a essa Lei, é caracterizado como criança aquele que possui a idade de até doze anos incompletos (ECA, Art. 2\textordmasculine).

Ressalta-se, por fim, que a inclusão da criança no âmbito do Direito não se deu somente porque ela necessitava de um cuidado especial devido à sua fragilidade, mas também porque o problema da criminalidade infanto-juvenil apresentou um aumento significativo no século XX. Segundo \citeonline{SOUZA2001}, esse é um problema que atinge a criança -- de família que vive na faixa de pobreza e miséria -- logo em seu início de vida, devido à carência de recursos básicos à sobrevivência, tornando-se assim vítima da negligência no que se refere à garantia de seus direitos fundamentais. 

Através do aprofundamento no estudo sobre a infância, é possível perceber que as crianças não receberam o mesmo tratamento nas sociedades, ou tiveram seu espaço garantido e direitos consolidados em todo tempo, ainda que se tenham revelado como sujeitos de cuidados e atenção específica. A legislação não foi sempre efetiva e os casos de violação aos seus direitos fundamentais eram constatados continuamente. 

O sistema jurídico brasileiro nem sempre abrangeu e priorizou os direitos das crianças e dos adolescentes. Muitas mudanças, infelizmente, ficam apenas no âmbito da palavra, sem passarem à ação e, embora algumas modificações já tenham se efetivado, ainda há muito que se conseguir. A aquisição dos direitos das crianças não se deu de forma imediata tampouco se darão os próximos êxitos. 

% ----------------------------------------------------------
% ELEMENTOS PÓS-TEXTUAIS
% ----------------------------------------------------------
\postextual


% ----------------------------------------------------------
% Referências bibliográficas
% ----------------------------------------------------------
\bibliography{bibliografia}

% ----------------------------------------------------------
% Glossário
% ----------------------------------------------------------
%
% Consulte o manual da classe abntex2 para orientações sobre o glossário.
%
%\glossary

% ----------------------------------------------------------
% Apêndices
% ----------------------------------------------------------

% ---
% Inicia os apêndices
% ---
%\begin{apendicesenv}

% Imprime uma página indicando o início dos apêndices
%\partapendices

%\end{apendicesenv}
% ---


% ----------------------------------------------------------
% Anexos
% ----------------------------------------------------------

% ---
% Inicia os anexos
% ---
\begin{anexosenv}

% Imprime uma página indicando o início dos anexos
\partanexos

\end{anexosenv}

%---------------------------------------------------------------------
% INDICE REMISSIVO
%---------------------------------------------------------------------

\printindex

\end{document}
