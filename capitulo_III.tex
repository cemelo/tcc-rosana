% ==================================
% Capítulo III
% ==================================

\chapter{A Psicologia Jurídica Brasileira Frente ao Abuso Sexual Infantil -- intervenções, possibilidades e limites}

\section{Breve histórico da constituição da Psicologia Jurídica}

A Psicologia possui relação com o Direito por meio de um importante ponto: a atenção ao comportamento humano. No entanto, se diferencia dele na medida em que está voltada para o mundo do ``ser'', não para o do ``dever ser'' \apud[p. 13]{RIVEROS1995}{ROVINSKI2004}.

O encontro da Psicologia com o Direito em algum momento da história se daria, por mais que eles tentassem se conservar distantes. Conforme \apudonline[p.34]{SOBRALARCEPRIETO1994}{JESUS2001},

\begin{citacao}
	a Psicologia, por um lado, procurando compreender e explicar o comportamento humano, e o Direito, por outro, possuindo um conjunto de preocupações sobre como regular e prever determinados tipos de comportamento, com o objetivo de estabelecer um contrato social de convivência comunitária. Podemos perceber, então, a complementaridade que a Psicologia pode fornecer ao Direito, sem desejar ir além do que lhe compete.
\end{citacao}

Indícios do início do estabelecimento da relação entre Psicologia e Direito estão na busca, dentro do âmbito jurídico, de se estabelecer normas para convívio entre as pessoas, conforme os preceitos de conduta. É ao século XVIII que se remontam os primeiros sinais do que viria ser a Psicologia Jurídica \cite{JESUS2001}. No final do século XIX, análises sobre o Direito e ``sua função na vida social'' foram elaboradas a partir da Psicologia e de saberes próximos a ela (p. 27). Foi nesse século que surgiu a necessidade de se aplicar a Psicologia ao Direito e isso pôde ser verificado através da publicação de várias obras.\footnotemark

\footnotetext{Dentre eles: ``A Psicologia em suas principais aplicações à administração da Justiça'' (Hoffbauer, 1808); ``Exposição Documentada de Crimes Célebres'' (Feuerbach); ``A Doutrina da Prova'' (Mittermaier, 1834); ``Manual Sistemático de Psicologia Judicial'' (1835) etc. \cite[p. 28]{JESUS2001}}

Em 1868 foi publicado o livro \emph{Psychologie Naturelle} -- de Prosper Despine, médico francês -- em que estudos de casos envolvendo grandes criminosos da época eram apresentados trazendo uma perspectiva de características psicológicas de cada um deles. A conclusão de Despine era a de que os criminosos -- com algumas exceções -- não demonstravam nenhuma enfermidade física ou mental \cite{LEAL2008}. Ainda segundo ele, as irregularidades nos comportamentos dos que praticaram os crimes estavam ``em suas tendências e seu comportamento moral'' (p. 172). Essas tendências eram frequentemente nocivas, tais como o ódio, a vingança etc. Iniciava-se aí a ``Psicologia Criminal'', denominação dada à época para as práticas da Psicologia que estudavam os aspectos psicológicos de quem cometia crimes (p. 173). 

A Psicologia Criminal passou a se ocupar principalmente com a compreensão do agir e da personalidade do criminoso, permitindo que o crime fosse tomado como um problema não apenas do criminoso, mas também de outros agentes como o Juiz, o advogado, o psicólogo, o psiquiatra e o sociólogo \apud{DOURADO1965}{LEAL2008}.

Outro autor importante na constituição do conhecimento sobre a criminalidade foi Cesare Lombroso. Psiquiatra e criador da Antropologia Criminal, ele ocupou-se da chamada ``Psicologia do Delinquente'', defendendo em seus estudos a ligação entre a criminalidade e as características físicas \cite{JESUS2001}. Ao considerar características fisiológicas e anatômicas na análise de criminosos, em detrimento de toda a complexidade de contexto e fatores, pode-se cair no simplismo e no reducionismo, beirando uma visão determinista que dá abertura à intensificação do preconceito e da estigmatização social. De acordo com \citeonline{FARIA2008}, Lombroso não conseguiu provar a relação entre as características hereditárias e a criminalidade. Prosseguindo com seus estudos, em sua maioria com homens e mulheres que já sofriam estigmas e críticas, Lombroso continuou não se atentando à importância da questão social e, por fim, suas teses acabaram somente por reforçar o preconceito, sendo que a questão racial foi a principal nesse cenário de categorização do ser humano. Segundo \citeonline[n.p.]{FARIA2008}, ``os negros eram sempre considerados menos evoluídos e mais perigosos socialmente''.

Apesar dessas condições, desde o momento em que a criminologia surgiu no cenário das Ciências Humanas para estudar os fatores determinantes da criminalidade, a personalidade e a conduta do delinquente e a maneira de ressocializá-lo, a Psicologia Criminal passa a ocupar uma posição de destaque como ciência capaz de contribuir para a compreensão da conduta e da personalidade do criminoso. Considerando que no estudo criminológico busca-se esclarecer o ato humano antissocial com o objetivo de preveni-lo, a Psicologia pôde contribuir para uma maior excelência no julgamento de cada caso, com a compreensão ou a tentativa de compreender o delinquente e as forças psicológicas que o levaram ao crime.

O século XX foi marcado por trabalhos empíricos e experimentais de psicólogos europeus envolvendo o testemunho bem como a sua participação em processos judiciais, o que contribuiu para que a Psicologia do Testemunho fosse impulsionada e os psicólogos se aprofundassem em aspectos de investigação. Concomitantemente, os estudos psicométricos utilizados em Laudos Psicológicos foram se desenvolvendo a partir da colaboração de psicólogos clínicos junto a psiquiatras, em exames psicológicos legais e nos sistemas de justiça juvenil \cite{JESUS2001}. No começo a psicologia foi muito vinculada apenas à aplicação de testes, visão que acabou limitando, de certa forma, a grande contribuição que ela poderia oferecer ao contexto jurídico.

Ainda no século XX, a influência da Psicanálise, teoria desenvolvida por Freud, colaborou para que a doença mental fosse revista e o sujeito fosse estudado de maneira mais compreensiva. Consequentemente, o psicodiagnóstico adquiriu mais respeito, abandonando um enfoque notavelmente médico e obtendo acréscimos de aspectos psicológicos \apud{CUNHA1993}{LAGO2009}. O psicodiagnóstico configurou-se como um importante instrumento na Psicologia Jurídica por sua perspectiva de orientação aos operadores do Direito, devido ao seu caráter de objetividade. Esse foi um período em que a contribuição da Psicologia esteve bastante voltada à avaliação psicológica, através da inauguração dos testes psicológicos para exames e avaliações \cite{LAGO2009}.

De uma forma geral, a Psicologia aplicada ao Direito no século XX ganhou várias determinações, a depender de seu objeto de estudo, tais como ``Psicologia Criminal'', ``Psicologia Judiciária'' e ``Psicologia Penal'' \cite[p. 234]{COSTA2009}. De acordo com \citeonline[p. 35]{JESUS2001}, a Psicologia Jurídica estabelece-se, então, como área de investigação psicológica especializada, que estuda ``o comportamento dos atores jurídicos no âmbito do Direito, da lei e da justiça''. Esse campo vem crescendo nas últimas décadas a nível nacional e internacional \cite{LEAL2008}.

Segundo \citeonline[p. 182]{LEAL2008}, as possíveis áreas de atuação na Psicologia Jurídica são:

\begin{citacao}
	Psicologia Jurídica e as Questões da Infância e Juventude (adoção, conselho tutelar, criança e adolescente em situação de risco, intervenção junto a crianças abrigadas, infração e medidas sócioeducativas); Psicologia Jurídica e o Direito de Família (separação, paternidade, disputa de guarda, acompanhamento de visitas); Psicologia Jurídica e Direito Civil (interdições, indenizações, dano psíquico); Psicologia Jurídica do Trabalho (acidente de trabalho, indenizações, dano psíquico); Psicologia Jurídica e o Direito Penal (perícia, insanidade mental e crime, delinqüência); Psicologia Judicial ou do Testemunho (estudo do testemunho, falsas memórias); Psicologia Penitenciária (penas alternativas, intervenção junto ao recluso, egressos, trabalho com agentes de segurança); Psicologia Policial e das Forças Armadas (seleção e formação da polícia civil e militar, atendimento psicológico); Mediação (mediador nas questões de Direito de Família e Penal); Psicologia Jurídica e Direitos Humanos (defesa e promoção dos Direitos Humanos); Proteção a Testemunhas (existem no Brasil programas de Apoio e Proteção a Testemunhas); Formação e Atendimento aos Juízes e Promotores (avaliação psicológica na seleção de juízes e promotores, consultoria e atendimento psicológico aos juízes e promotores); Vitimologia (violência doméstica, atendimento a vítimas de violência e seus familiares) e Autópsia Psicológica (avaliação de características psicológicas mediante informações de terceiros).
\end{citacao}

Embora o crescimento das possibilidades de atuação em Psicologia Jurídica -- a nível não apenas nacional -- e da própria importância em si dessa área, vale destacar que a carência de profissionais especializados ainda é alta. É de extrema importância que quem deseje seguir nessa direção busque qualificação, a fim de desenvolver uma reflexão crítica que colabore para a construção de uma sociedade que entenda a complexidade do indivíduo e dos determinantes que o formam, dentre muitas outras contribuições.

\section{A Psicologia Jurídica no Brasil}

A Psicologia Jurídica no Brasil, de acordo com o Conselho Federal de Psicologia, é definida como ``uma das especialidades do psicólogo'' \cite[p. 234]{COSTA2009}. É uma área com ampla possibilidade de atuação, mas que ainda se encontra bastante associada aos processos jurídicos \apud{BONFIM1994}{COSTA2009}, realidade que profissionais buscam transformar, a fim de ampliar suas práticas e contribuir de maneira crítica.

No Brasil, a Psicologia Jurídica é uma área relativamente recente de atuação, especialmente relacionada ao século XX. Seu início não é definido por um marco histórico específico, mas afirma-se que os psicólogos jurídicos começam sua atuação em meados da década de 60, quando do reconhecimento da profissão em Psicologia \cite{LAGO2009}. Anterior a esse marco, a atuação foi gradual, informal e marcada pela ação em trabalhos voluntários.

Os psicólogos iniciaram sua atuação no contexto jurídico, prioritariamente, em Varas de Família, Cível, Criminal, da Criança e do Adolescente, elaborando laudos sob o modelo pericial \cite{COSTA2009}.

A atividade do Psicólogo na Justiça se determina por legislações específicas e previsões nos regimentos internos dos Tribunais de Justiça \cite{COSTA2009}. Na lei n\textordmasculine 7.210, de 17 de julho de 1984, está prevista a atuação desse profissional para o Sistema Penal Brasileiro:

\begin{citacao}
	Art. 6\textordmasculine A classificação será feita por Comissão Técnica de Classificação que elaborará o programa individualizador e acompanhará a execução das penas privativas de liberdade e restritivas de direitos, devendo propor, à autoridade competente, as progressões e regressões dos regimes, bem como as conversões.
	
	Art 7\textordmasculine A Comissão Técnica de Classificação, existente em cada estabelecimento, será presidida pelo diretor e composta, no mínimo, por dois chefes de serviço, um psiquiatra, um psicólogo e um assistente social, quando se tratar de condenado à pena privativa da liberdade \cite{BRASIL1984}.
\end{citacao}

Desde 2003, a redação do art. 6\textordmasculine sofreu alteração pela Lei n\textordmasculine 10.792 de 1\textordmasculine de dezembro de 2003, e a nova composição passou a ser: ``A classificação será feita por Comissão Técnica de Classificação que elaborará o programa individualizador da pena privativa de liberdade adequada ao condenado ou preso provisório'' \cite{BRASIL2003}. 

As ações do psicólogo na justiça encontram embasamento no Código de Ética Profissional de 27 de agosto de 2005. O código indica, em \emph{Princípios Fundamentais}, que o trabalho do psicólogo será baseado ``no respeito e na promoção da liberdade, da dignidade, da igualdade e da integridade do ser humano, apoiando nos valores que embasam a Declaração Universal dos Direitos Humanos'' e visará a promoção de saúde e de qualidade de vida das pessoas e das coletividades, contribuindo para a eliminação de quaisquer formas de ``negligência, discriminação, exploração, violência, crueldade e opressão'' \cite[p. 27]{LIMA2008}. 

Atualmente um campo de atuação em ascensão para os psicólogos é o Direito Civil. Esse direito refere-se ao Direito Privado Comum, que disciplina o estado e a capacidade das pessoas e suas relações, de caráter privado, relativos à ``família'', às ``coisas'', às ``obrigações'' e à ``transmissão hereditária dos patrimônios'' \apud[p. 485]{RAO1952}{MONTORO2005}. Os trabalhos podem ser realizados com famílias, em perícias psicológicas e também no Juizado de Menores. A partir de 1990, quando o Juizado de Menores passou a ser denominado ``Juizado da Infância e Juventude'', o psicólogo teve seu trabalho ampliado -- por exemplo, na área pericial; no acompanhamento e aplicação das medidas de proteção ou socioeducativas etc. \apud[p. 485]{TABAJASKIGAIGERRODRIGUES1998}{LAGO2009}.

\citeonline[p. 78]{FRANCA2004} destaca os principais setores de atuação do psicólogo jurídico no Brasil, a saber, Psicologia Criminal, Psicologia Penitenciária ou Carcerária, Psicologia Jurídica e as questões da infância e juventude, Psicologia Jurídica: investigação, formação e ética, Psicologia Jurídica e Direito de Família, Psicologia do Testemunho, Psicologia Jurídica e Direito Civil e Psicologia Policial/Militar. 

Como emprego do saber psicológico às questões relacionadas ao saber do Direito e ao exercício desse \cite{LEAL2008}, a Psicologia Jurídica muito tem auxiliado, sobretudo devido aos seus conhecimentos colocados à disposição do juiz a respeito da esfera psicológica dos agentes envolvidos em processos, análises estas que vão além da literalidade restritiva da lei e que, de outra maneira, podiam não chegar ao conhecimento do juiz \apud{SILVA2007}{LEAL2008}.

\citeonline[p. 35]{JESUS2001} divide sinteticamente algumas atribuições do psicólogo jurídico, tais como ``avaliação e diagnóstico'' no que tange às condutas psicológicas dos atores jurídicos; ``assessoramento e/ou orientação'' nas perícias em órgãos judiciais, nos casos próprios da Psicologia; ``intervenção'' junto a atores jurídicos na comunidade e no âmbito penitenciário, individual e coletivamente; ``formação e educação'', como em trabalhos de seleção e treinamento de profissionais do sistema legal; ``campanhas'' de prevenção social contra a criminalidade em meios de comunicação; pesquisa no que concerne à Psicologia Jurídica; contribuição para a melhoria da situação da vítima e sua interação com o sistema legal e ações de mediação, permitindo que, por meio de soluções negociadas dos conflitos jurídicos, o dano emocional e social seja diminuído e prevenido, e contribuindo para que se ofereça uma alternativa à via legal em que as partes tenham um papel predominante.

Dentro da Psicologia Jurídica é possível especificar três subconjuntos de áreas de atuação: A Psicologia Forense, a Psicologia Criminal e a Psicologia Judiciária. Na Psicologia Forense estão as práticas psicológicas ligadas aos procedimentos forenses, correspondendo a toda aplicação do saber psicológico a um processo ou procedimento em andamento no Foro. A Psicologia Criminal, também subconjunto da Psicologia Forense, faz o estudo das condições psíquicas do criminoso e como se origina e se processa nele a ação criminosa \apud{BRUNO1997}{LEAL2008}. A Psicologia Criminal abrange a Psicologia do delinquente, a Psicologia do delito e a Psicologia das testemunhas. A Psicologia Judiciária, também subconjunto da Forense, se relaciona a toda prática psicológica realizada ``a mando e a serviço da justiça'', ou seja, acontece sob subordinação imediata à autoridade judiciária. É na Psicologia Judiciária que está inclusa a função pericial \cite{LEAL2008}.

A perícia, do latim ``\emph{peritia}: destreza, habilidade'', refere-se ao exame de fatos ou situações, realizado por especialista na matéria que lhe é submetida, e tem por objetivo elucidar determinados aspectos técnicos \apud[p. 21]{BRANDIMILLER1996}{ROVINSKI2004}.

Na área judicial, a investigação pericial tem por objetivo clarificar situações e fatos controversos provindos de conflitos de interesses relacionados a um direito contestado. Ressalta-se que a perícia, como meio de prova, não se constituiu em uma verdade única, mas, reafirma-se, uma elucidação acerca dos fatos a fim de dar fundamento e auxílio em uma posterior conclusão que cabe somente ao Juiz \cite{ROVINSKI2004}.

O artigo 145 do Código de Processo Civil especifica quem pode exercer as atividades de perito. Tendo-o por base, afirma-se que está apto ao papel de perito o psicólogo devidamente regulamentado ao CRP\footnotemark e que tenha capacidade técnica para responder, concernente à matéria de psicologia, às questões formuladas em juízo. Através de seu órgão de classe, a função de perito pelo psicólogo também fica regulamentada. No Decreto 53.964 (21.01.64), que regulamenta a Lei 4.112 --  responsável pela criação da profissão de psicólogo -- a situação de realizar perícia e emitir pareceres sobre a matéria de Psicologia é prevista \cite{ROVINSKI2004}.

\footnotetext{ Embora no Brasil já exista o reconhecimento da área de Especialização em Psicologia Jurídica, não se exige esse título para a atuação em perícias judiciais. Basta o psicólogo estar regulamentado pelo CRP do qual faz parte. Tal fator, no entanto, não impede que o psicólogo busque conhecimento acerca do que for investigar e do sistema em que vai operar \cite{ROVINSKI2004}.}

\section{Práticas, possibilidades e limites frente ao abuso sexual infantil}

Considerando a forte aproximação que se deu nos últimos anos entre Psicologia e Direito, observa-se que a Psicologia Jurídica, um resultado interdisciplinar, aos poucos vem se constituindo essencial na garantia da justiça. Na medida em que o reconhecimento de sua contribuição a partir de suas avaliações substanciais e cientificamente embasadas continuar ocorrendo, a possibilidade de mudanças mais efetivas ocorrerão \apud{STEIN2009}{PELISOLIGAVADELLAGLIO2011}. Mas para isso é importante que o profissional exerça constante reflexão acerca de sua prática, para não incorrer simplesmente em realizações rotineiras de avaliação e elaboração de laudos, ocultando assim determinações dos acontecimentos, avaliando as pessoas de forma superficial ou taxando-as através dos instrumentos e laudos científicos.

O abuso sexual como um problema complexo de ordem mundial exige a interlocução de diversos saberes, dentre eles a Psicologia e o Direito. De acordo com \citeonline[n.p.]{GRIGOLATTOROVERATTI2009}, 

\begin{citacao}
	A interação entre a Psicologia e o Direito na área de Abuso sexual de crianças e adolescentes é de extrema importância, uma vez que a psicologia como mediadora entre a criança e o contexto judiciário participa da trajetória sócio-histórica da infância brasileira, exigindo do profissional um compromisso ético e uma boa qualidade de escuta, trazendo à realidade toda complexidade da criança.
\end{citacao}

Com frequência os casos de abuso sexual que chegam ao sistema jurídico precisam ser avaliados e verificados. A partir da alegação de abuso sexual infantil, é necessário que algumas etapas sejam cumpridas como ``investigação da suspeita, medidas de proteção para a criança e ação legal'' etc \apud[n.p.]{OATES2000}{PELISOLIGAVADELLAGLIO2011}. Diante disso, profissionais de diversas categorias, atuantes no contexto jurídico, têm importantes contribuições. 

Por sua vez, o psicólogo jurídico, pode intervir nessas etapas de algumas maneiras, a depender do objetivo que lhe é proposto. De uma forma geral e sucinta, ele tem o papel de auxiliar o juiz durante o processo judicial, participando da produção de conhecimento acerca dos fatos psicólogos das pessoas envolvidas -- sejam estas a vítima, seus pais/responsáveis ou o abusador, que pode inclusive ser um dos pais ou responsável pela criança. Isso porque em grande parte das denúncias de abuso sexual infantil, não se dispõe de provas materiais e de testemunhas que confirmem  sua ocorrência, sem falar da possível ausência de vestígios físicos no corpo da criança. De acordo com \apudonline{IPPOLITO2003}{JUNG2006}, em apenas 30\% dos casos se observam evidências físicas do abuso. Além disso, o agressor está sempre disposto a negar seu crime. 

Diante disso, um dos principais papeis do psicólogo jurídico nessa área é o de perito. A perícia psicológica forense

\begin{citacao}
	pode ser definida como o exame ou avaliação do estado psíquico de um indivíduo com o objetivo de elucidar determinados aspectos psicológicos deste; este objetivo se presta à finalidade de fornecer ao juiz ou a outro agente judicial que solicitou a perícia, informações técnicas que escapam ao senso comum e ultrapassam o conhecimento jurídico \cite[p. 36]{JUNG2006}.
\end{citacao}

No processo de avaliação de uma criança possivelmente abusada sexualmente, realizam-se a descrição de personalidade, uma análise da repercussão dos fatos no psiquismo e a compreensão dos acontecimentos. A partir disso, é possível traçar um esboço do retrato psicológico da criança e refletir sobre a necessidade e a possibilidade de encaminhamento a um tratamento psicoterapêutico \apud{VIAUX1997}{JUNG2006}, afinal, a ajuda às crianças vítimas de abuso sexual não inclui somente o diagnóstico e a punição ao agressor. 

O psicólogo jurídico, além da entrevista clínica, pode se utilizar de instrumentos psicológicos, como os testes -- especialmente os projetivos -- e um conjunto de técnicas lúdicas a fim de conseguir dados por parte da criança. De acordo com \apudonline[p. 37]{SILVA2003}{JUNG2006}, os instrumentos utilizados devem abranger ``métodos e materiais adequados, destinados a analisar e avaliar aspectos referentes à estrutura da personalidade, à cognição, à dinâmica e à afetividade das pessoas envolvidas''. Em entrevistas com os pais ou responsáveis é possível detectar aspectos que auxiliam na suposição ou não da vitimização bem como na possibilidade de avaliar gravidade e frequência. 	Através do trabalho de avaliação pericial, o psicólogo pode levantar tanto a confirmação da ocorrência do abuso quanto da gravidade das alterações psicológicas e do dano psíquico \cite{JUNG2006}. 

Como um meio de prova no contexto forense, a materialização da perícia se dá através do Laudo Pericial. Segundo \citeonline[p. 37]{JUNG2006},

\begin{citacao}
	O laudo pericial, que será apreciado pelo agente jurídico que o solicitou, deve ser redigido em linguagem clara e objetiva, para que possa efetivamente fornecer elementos que auxiliem a decisão judicial, devendo responder ao quesito solicitado, que, neste caso, concretiza-se numa pergunta do tipo: `há indícios de que esta criança foi vítima de abuso sexual?'
\end{citacao}

É importante ressaltar que, no contexto jurídico, como o foco é determinado pelo sistema legal, o objetivo final da avaliação forense será o de responder a uma questão legal expressa pelo agente jurídico, através da compreensão psicológica do caso. Portanto, o psicólogo está, de certa forma, sujeito e limitado ao que lhe foi solicitado. Considerando isso, para \apudonline[p. 43]{MELTON1997}{ROVINSKI2004}, aspectos clínicos como ``diagnóstico'' ou ``necessidade de tratamento'' ficam para segundo plano, em relação a outros de relevância legal no caso. A atitude do psicólogo perito deve ser de maior objetividade, ``afastamento'', e neutralidade, atentando-se para que o processo de avaliação forense não seja transformado em um ambiente terapêutico \cite[p. 44]{ROVINSKI2004}.	

Dessa forma, por mais que o processo de avaliação psicológica no marco legal não se diferencie substancialmente daquele que ocorre no contexto da clínica, é fundamental que adequações dos procedimentos às normas sejam feitos, já que os contextos não são os mesmos e a forma como o ``cliente'' se apresenta a esse momento é diferente \cite[p. 42]{ROVINSKI2004}.

Embora esteja pautado em uma demanda do juiz, é fundamental que o psicólogo jurídico reflita constantemente sobre a possibilidade de, através de sua atuação, não reproduzir apenas técnicas a fim de se chegar a um conhecimento e elucidação maior dos fatos. Mas que atue também por meio de um olhar amplificado, que considere o sujeito dentro de um contexto social complexo ao qual se vincula a violência por ele sofrida, contribuindo por fim para facilitar ou promover mudanças na esfera individual do mesmo. De acordo com uma pesquisa de \apudonline{SANTOS2004}{SANTOS2009}, o papel do psicólogo no contexto jurídico pode ir além da tecnicidade de uma perícia, atingindo múltiplas possibilidades e gerando mudanças, entendendo estas não como ``o controle social que assujeita o indivíduo a uma norma ou padrão estabelecido, mas a mudança indicada pela própria família em seu desejo de resolução de conflitos'', mudança por meio de autoconhecimento e da competência própria para transformar e dar significado ao seu ato, tornando-se sujeito e autor de si mesmo e da própria história. 

\citeonline[p. 8]{SANTOS2009} realiza reflexões que apontam para a necessidade de que o Psicólogo Jurídico mantenha constantemente uma postura crítico-reflexiva e que se atente  

\begin{citacao}
	para o risco de que a intervenção do sistema de proteção à criança e ao adolescente reproduza o mesmo modelo de controle social que vigorava na época do Código de Menores. Por outro lado, indicam também que o psicólogo ao se isentar e se ausentar de cenas significativas no estabelecimento de um Estado de direito e justo, estará permitindo a continuidade de relações sociais desiguais, eticamente condenadas. Pois, se uma intervenção pode seguir modelos vigentes, promovendo e corroborando injustiças, a \emph{não intervenção} também não deixa de ser uma resposta que da mesma forma pode manter, favorecer e legitimar injustiças sociais.
\end{citacao}

Dessa forma, é fundamental que o psicólogo desenvolva uma escuta diferenciada, capaz de promover acolhimento devido e necessário à criança vitimizada e a suas famílias. Para isso, ``é preciso favorecer uma relação de empatia e solidariedade onde o profissional possa se reconhecer nela, na mesma condição humana de vulnerabilidade e desproteção'' \cite[p. 25]{SANTOS2009}. É a partir daí que a criança estará apta não somente a compartilhar informações sobre os fatos, mas a transmitir seus sentimentos. 

Essa diferenciação na forma de atuação em perícias não é permanente entre os psicólogos, até porque se constitui em um exercício e uma busca de atenção diferenciada constantes, a que nem todos os profissionais psicólogos no contexto jurídico estão acostumados, seja pela carência de orientação anterior sobre esse contexto e suas possibilidades diante do social, seja pelo engessamento que o sistema judiciário aos poucos promove no profissional etc. \apudonline[p. 3]{ARANTES2007}{SANTOS2009} realiza discussões nesse sentido, apontando algumas reflexões de psicólogos jurídicos insatisfeitos ``com sua atuação restrita às avaliações, com a fragilidade epistemológica desse campo de conhecimento e com a falta de autonomia profissional''.

\citeonline[p. 19]{SANTOS2009} reflete sobre a importância de um atendimento diferenciado e propõe ao Psicólogo Jurídico, mesmo submetido ao limite da determinação judicial:

\begin{citacao}
	um modo de intervenção [\ldots] que possibilite não apenas uma avaliação pericial, mas a construção de um espaço conversacional, no qual possam emergir os significados construídos e constituir novos significados. Mais do que apenas ouvir a criança, sem contextualizar sua narrativa como uma oportunidade para o resgate do direito restitutivo e protetor em complementaridade ao direito da regulação. 

	Acredita-se que, ao buscar uma compreensão das condições emocionais, relacionais e sociais dos envolvidos, com o objetivo de se conhecerem quais direitos deverão ser reparados, o estudo psicossocial tem o potencial de oferecer às crianças e adolescentes e a seus familiares as devidas condições de emancipação e de mudança do contexto de risco.
\end{citacao}

Por fim, ressalta-se que para que o contexto jurídico possa ser propício para transformações, e não somente decisões, mudanças precisam ocorrer tanto na formação do psicólogo quanto na dos operados do Direito. São necessárias também modificações nas concepções da Justiça através das quais não se preocupe prioritariamente em regular as relações entre os cidadãos, mas que se voltem para o cuidado e cidadania das pessoas \cite{COSTA2009}. 

Embora ainda haja muito que ser feito e aperfeiçoado entre os psicólogos no meio Jurídico, é importante também considerar que os limites em sua intervenção existem -- como em qualquer outra profissão -- a fim de que o ``psicologismo'' seja evitado, comprometendo o espaço já consolidado do profissional psicólogo \cite[p. 34]{JESUS2001}.