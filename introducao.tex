% ==================================
% Introdução
% ==================================

\chapter*[Introdução]{Introdução}
\addcontentsline{toc}{chapter}{Introdução}

A Psicologia Jurídica -- campo em que os conhecimentos de Psicologia são diretamente aplicados aos assuntos relacionados ao Direito -- é campo relativamente recente de atuação dos profissionais psicólogos, sobretudo no Brasil. A nível internacional, os primeiros indícios da relação entre a Psicologia e o Direito se deram por volta do século XVIII através da busca, no âmbito jurídico, de se estabelecer normas para o convívio entre as pessoas conforme os preceitos de conduta. Foi no final do século XIX que a necessidade da Psicologia aplicada ao Direito se firmou, sendo consolidada com a publicação de diversas obras sobre o assunto \cite{JESUS2001}.

Em termos gerais, essas duas disciplinas se aproximam pela atenção ao comportamento humano e se diferenciam na medida emque uma se volta para o mundo do ``ser'' e a outra para o do ``dever ser'' \apud{RIVEROS1995}{ROVINSKI2004}.

No Brasil, a constiuição da Psicologia Jurídica se deu mais tardamente, aproximadamente em meados do século XX, no contexto do reconhecimento da profissão em Psicologia, na década de sessenta. Os psicólogos iniciaram sua atuação de forma prioritária em Varas de Família, Cível, Criminal, da Criança e do Adolescente, elaborando laudos sob o modelo pericial \cite{COSTA2009}. Primeiros  registros de trabalhos de psicólogos em instituições de Justiça no Brasil datam das décadas de setenta e oitenta.

A atuação do psicólogo jurídico brasileiro é regulamentada em legislações específicas e reconhecida pelo Conselho Federal de Psicologia. Muitos são os setores possíveis de intervenção do psicólogo no contexto jurídico, sendo alguns deles a Psicologia Criminal, a Psicologia Jurídica e Direitos de Família, a Psicologia do Testemunho etc. 

Observa-se que uma importante contribuição do psicológo à justiça é no sentido de, através de seus estudos, formular conhecimento acerca da esfera psicológica de agentes envolvidos em processos, colocando-o à disposição do juiz como forma de auxílio às futuras decisões ou sentenças do mesmo \apud{SILVA2007}{LEAL2008}.

Atualmente um trabalho que com frequência tem sido solicitado ao psicólogo jurídico é o de perícia psicológica. A perícia refere-se ao exame de fatos ou situações, praticado por especialista na matéria que lhe é submetida, e tem por objetivo elucidar determinados aspectos técnicos \apud{BRANDIMILLER1996}{ROVINSKI2004}. As solicitações podem ser por motivos de guarda de crianças, interdições, verificação da ocorrência de abuso sexual infantil etc.

O abuso sexual infantil constitui-se em uma das diversas formas de violência contra crianças. Várias são as maneiras como ele pode ser praticado, portanto sua descrição é extensa. No entanto, de uma forma geral, é definido, segundo a Organização Mundial da Saúde, como a participação de crianças em atividades não compatíveis com a sua idade e com as quais ela não está apta a consentir ou compreender completamente.  O fenômeno atinge meninas e meninos de todas as classes econômicas, seja no contexto familiar ou externo a ele. Baseado nisso, ele pode ser classificado como intrafamiliar ou extrafamiliar.
	
O abuso é permeado pelo aspecto do poder, ou seja, o abusador sempre se encontra em uma posição de autoridade com relação à criança, o que lhe incentiva a utilizar-se dela para atingir seu objetivo de gratificação sexual. 
	
As consequências do abuso sexual são múltiplas na vida da criança, podendo atingir as áreas física, psicológica, afetiva, relacional e comportamental, a curto ou a longo prazo. 

Um componente recorrente nas situações de abuso sexual infantil é o silêncio da vítima, por detrás do qual está, principalmente, o medo da criança do que possa ocorrer se o fato vier à tona. 

Pensar o abuso sexual infantil, sua recorrência, as formas crueis de que ocorrem e as relações de poder nele enredadas permite constatar que os direitos humanos da criança não vêm sendo efetivados, mas negligenciados, sobretudo os direitos sexuais.  

Nem sempre na historia esses direitos existiram. Aliás, nem sempre houve infância, tal como a concebemos na modernidade. O sentimento de infância, com as noções de ingenuidade e inocência, surgiu, foi modificado e determinado por uma complexidade de fatores sociais, culturais, políticos e econômicos e por movimentos culturais iluministas e religiosos protestantes da sociedade europeia dos séculos XVII e XVIII \cite{ARIES1981}. Até o século XVII, as crianças eram consideradas como miniaturas do homem, sendo misturadas no meio de adultos e com eles se envolvendo em todos os tipos de práticas, inclusive as sexuais \cite{ARIES2011}. 
	
A família moderna, que teve o início do seu estabelecimento na burguesia do século XVIII, instalou padrões de intimidade e de vida privada, que culminava na união sentimental entre casal e entre os pais e filhos \cite{ARIES2011} e uma nova maneira de agir frente a questões sexuais, implementando uma série de fatores de controle e de educação da criança. 

No Brasil a infância adquire importância a partir do século XVI, sofrendo a influência do processo que ocorria no contexto europeu, mas diferenciando-se desse na medida em que atribuía à criança outros status, provindos, principalmente, de um contexto de preconceito, exploração, abandono e pobreza no Brasil da época – contexto esse que propiciou para que os direitos e o reconhecimento à criança fossem pensados e elaborados. 

Este trabalho pretendeu produzir conhecimentos acerca das práticas da Psicologia Jurídica brasileira frente ao abuso sexual infantil, abrangendo também suas possibilidades e limites. Para tanto, se fez necessário compreender alguns aspectos do fenômeno. Antes disso, foi preciso ainda entender e contextualizar a ``criança'' de que se fala. Isso porque nem sempre ela existiu, foi vista e tratada como é hoje. 

Compreendendo então que o conceito de criança da modernidade foi construído e moldado historicamente a partir de uma gama de mudanças nas sociedades, até que a ela foram atribuídos status de sujeito dependente e em condição especial de desenvolvimento, é possível realizar reflexões sobre a transformação na forma das relações entre crianças e adultos que legitima o abuso de poder e de autoridade por parte desses.

Entendendo também que a violência sexual é um problema social multifatorial e que cada vez mais tem sido noticiado, é possível ponderar algumas práticas diante dela atualmente.

Com base no interesse na Psicologia Jurídica como uma área crescente de atuação do psicólogo, no estágio realizado nesse contexto durante o último ano do curso, bem como pesquisas e o próprio contato com uma perícia psicológica em que uma criança supostamente havia sido abusada sexualmente por um membro de sua família, pretendeu se aprofundar no assunto, por meio de revisão da literatura, a fim de levantar dados sobre a contribuição da Psicologia Jurídica brasileira diante da suspeita e/ou da ocorrência de abuso sexual infantil. Além disso, algumas possibilidades e limites com os quais ela se depara nessa atuação. 

O trabalho está estruturado em três capítulos. O primeiro delineia o caminho que a criança percorreu na história, a partir do século XIII, que lhe concedeu o status que possui na modernidade. Esse capítulo se subdivide em dois pontos: A Constituição da Infância e A Infância no Brasil. O segundo capítulo abrange os aspectos gerais do Abuso Sexual Infantil e o terceiro das práticas da Psicologia Jurídica brasileira diante desse fenômeno, estando subdivido em três pontos: Breve Histórico da Constituição da Psicologia Jurídica, A Psicologia Jurídica no Brasil e Práticas e Limites Frente ao Abuso Sexual Infantil.

A título de esclarecimento, utilizou-se aqui as nomenclaturas \emph{violência sexual infantil} e \emph{abuso sexual infantil} como equivalentes. 
