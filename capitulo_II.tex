% ==================================
% Capítulo II
% ==================================

\chapter{Aspectos Gerais do Abuso Sexual Infantil}

Em tempos recentes na História a criança passou a possuir direitos e ser estudada como indivíduo em situação especial de desenvolvimento. Esse movimento foi também acompanhado pela preocupação com a garantia permanente e legal de sua proteção. No entanto, esse contexto não foi suficiente para que se extinguisse a violência à população infantil tampouco para impedir que novos dispositivos fossem empregados em atos de crueldade \cite{CAMPOS2002}. 

A violência contra crianças não é uma questão que remonta apenas ao século XX, mas acompanha a própria história da humanidade, ainda que não se tenha registro de todos os casos desde seu início \apud[n.p.]{ASSIS1994}{ALGERISOUZA2006}. No Brasil, é possível encontrar referências da prática de violência doméstica contra crianças já no período colonial (1500 -- 1822) \cite{LONGO2002}.

A violência infantil se revela de múltiplas maneiras, mas se configura, normalmente, em algum dos tipos já definidos, como negligência, abandono, violência física, psicológica,  sexual, dentre outros. 

Conforme \apudonline{FALEIROS2000}{JUNG2006} aponta, a violência sexual contra crianças se trata de uma  construção histórica e cultural nos processos sociais, que se articula ao nível de desenvolvimento e de civilização das sociedades em que acontece. 

Considerando a sociedade europeia dos séculos XIII ao XVII, em que os adultos submetiam crianças a práticas sexuais -- o que era relativamente comum diante de todos -- é possível refletir que essas práticas não se equiparam as que hoje são classificadas como formas de abuso sexual infantil. Isso porque a sociedade dessa época tinha uma maneira diferente de tratar e de se relacionar com as crianças e possuia concepções e costumes diferentes dos que predominam na maioria das sociedades atuais. À criança ainda não se havia atribuído os aspectos de inocência e de fragilidade tampouco lhes eram destinadas direitos específicos e leis de proteção. Aliás, ela era vista como um ``homem em miniatura'' \cite{ARIES2011} e não como um ser diferenciado que necessitasse de cuidados e de atenção especial, o que pode, em partes, explicar o tratamento indiferenciado que recebia. 

Diante disso, é possível refletir que houve um processo complexo responsável por caracterizar essas práticas como um crime e por afastar a criança de questões relacionadas à sexualidade como um todo. Diante dessas ponderações, cabe refletir e questionar: por que a criança? O que está por detrás do abuso sexual infantil que o torna um crime tão recorrente e, ao mesmo tempo, tão velado?

A violência sexual, de acordo com dados do Ministério da Saúde, é o segundo tipo de violência mais praticado contra crianças de zero a nove anos de idade. Como um crime de natureza sexual, torna-se um problema social de extrema importância e de saúde pública que provoca sérias consequências.

\begin{citacao}
	A prevalência do abuso sexual na população geral foi foco de um estudo realizado pela OMS (1999). Os resultados apontaram que, aproximadamente, 20\% das mulheres e 5 a 10\% dos homens sofreram abuso sexual na infância. Além disso, em todo o mundo, as taxas de prevalência de intercurso sexual forçado e outras formas de violência sexual que envolvem toques entre adolescentes com idade inferior a 18 anos é de 73 milhões (ou 7\%) entre os meninos e 150 milhões (ou 14\%) entre as meninas. \cite[p. 69]{HABIGZANG2012}
\end{citacao}

Outras estimativas demonstram que uma em cada quatro garotas e um em cada seis garotos experienciaram alguma forma de abuso na infância \apud{SANDERSON2005}{HABIGZANG2012}.

A visibilidade do fenômeno no Brasil se deu a partir da década de 1980, período em que também se travavam lutas pela defesa e garantia dos direitos da criança e do adolescente, presentes na Constituição Federal de 1988 (artigo 227)\footnotemark e no Estatuto da Criança e do Adolescente, de 1990 \apud{FERRARI2002}{ESBER2007}. De acordo com \citeonline{ESBER2009}, nesse contexto de transformação e de consolidação de leis voltadas à criança, destaca-se uma sequência de marcos históricos, a níveis internacional e nacional, que culminaram no enfrentamento da violência sexual por meio de políticas públicas. O primeiro marco se constitui na garantia dos direitos de crianças e adolescentes respaldadas legalmente com a  Declaração de Genebra, a Declaração Universal dos Direitos da Criança \citeyear{ONU1959} e o próprio ECA \cite{BRASIL1990}. O segundo marco se refere aos movimentos sociais como o Feminista, o Movimento Nacional de Meninos e Meninas de Rua e o Comitê de Enfrentamento à Violência Sexual contra Crianças e Adolescentes. O terceiro é a produção de estudos sobre a violência sexual contra crianças e adolescentes que muito têm contribuido para o aprofundamento e a elucidação sobre o fenômeno. Por fim, o quarto marco, como uma resposta aos marcos anteriores, se refere às políticas sociais de enfrentamento. Dentre elas está o Plano Nacional de Enfrentamento da Violência Sexual Infanto-Juvenil \citeyear{MJ2001}, criado pelo \citeauthor{MJ2001}, juntamente com a sociedade civil. 

\footnotetext{É dever da família, da sociedade e do Estado assegurar à criança, ao adolescente e ao jovem, com absoluta prioridade, o direito à vida, à saúde, à alimentação, à educação, ao lazer, à profissionalização, à cultura, à dignidade, ao respeito, à liberdade e à convivência familiar e comunitária, além de colocá-los a salvo de toda forma de negligência, discriminação, exploração, violência, crueldade e opressão.}

Embora a luta contra a violência sexual no Brasil tenha sido travada há décadas, percebe-se que o problema está longe de ser resolvido. As notícias das diferentes expressões de abuso sexual em crianças nos levam ao espanto, ao desconforto e à constatação do horror desses fatos. De acordo com pesquisas, somente 10\% dos casos são relatados ou conseguem chegar ao sistema judiciário criminal'' \apud{SANDERSON2005}{HABIGZANG2012}. Isso significa que o abuso acontece mais do que temos conhecimento, ou seja,  ainda que sejam notificadas algumas ocorrências, muito ainda permanece oculto.
	
As situações do abuso, em geral, são permeadas pelo silêncio da criança.  Muitas vezes, ele pode ser decorrente do medo do que venha ocorrer caso a denúncia  seja feita. \citeonline[p. 17]{BASS1983} discute a respeito das causas desse silêncio:

\begin{citacao}
	Mesmo que não haja o envolvimento de força física, toda vez que uma criança é usada sexualmente [\ldots], há coerção. Uma criança se submete a isso por diversas razões: tem medo de magoar os sentimentos [\ldots]; quer e precisa de afeto e esta é a única maneira que lhe é oferecida; teme que, se resistir, o homem\footnotemark a machucará, ou irá se vingar em alguém que ela ama; ou então que irá dizer que ela é quem estava querendo e, assim, lhe causará problemas; a criança é pega de surpresa e não tem a menor ideia do que fazer; o homem lhe diz que aquilo é certo, que está ensinando-a, que todo mundo também faz; ela aprendeu a obedecer aos adultos; acha que não tem outra escolha.
\end{citacao}

\footnotetext{Embora a autora priorize os casos de abuso sexual infantil em que o homem é o autor, o Código Penal já abrange a autoria de mulheres nesse crime.}

Com relação à diferenciação dos casos de abuso sexual infantil, de acordo com o âmbito de sua ocorrência, eles podem ser classificados em intrafamiliares\footnotemark -- aqueles que ocorrêm no próprio contexto familiar -- e extrafamiliares, que ocorrem externamente a esse meio. No primeiro contexto, 

\footnotetext{Também conhecido como abuso sexual incestuoso ou abuso sexual doméstico.}

\begin{citacao}
	por se tratar de uma esfera privada, [\ldots] encontra-se envolvido por esta atmosfera de segredo, podendo ter a complacência de outros membros da família; muitas vezes o abusador é, inclusive, o provedor econômico da casa \cite[p. 9]{JUNG2006}
\end{citacao}

Esta noção de espaço privado que a família assumiu na modernidade pode dificultar qualquer tipo de exposição e de possíveis avaliações externas. Normalmente, da família se espera proteção e cuidado aos filhos, não se aceitando assim contradições. Para \apudonline{OLIVEIRA1989}{JUNG2006}, é no âmbito familiar que o abuso sexual se inicia e se mantém mais facilmente, justamente pelo silêncio da vítima, decorrente da cumplicidade e da autoridade dos pais que lhe são impostas. 

Nas ocorrências extrafamiliares, os abusadores podem ser pessoas conhecidas da vítima e próximas de sua família ou totalmente desconhecidas e, nesses casos, a criança está sujeita a sofrer uma série de ameaças diretas a ela e/ou à sua família, o que facilita para que se permaneça o silêncio diante do ocorrido \cite{JUNG2006}. Outros fatores que podem colaborar para esse silenciamento são a culpa e a vergonha estimuladas pelo abusador na criança e o descrédito de alguns adultos diante do relato por parte dela. De acordo com \apudonline[p. 15]{GABEL1997}{JUNG2006}, ``a criança tem medo de falar e, quando o faz, o adulto tem medo de ouvi-la''. 

Esses fatos nos permitem refletir, dentre outras coisas, sobre a dificuldade em aproximar a criança de discussões que envolvam a sexualidade. À criança foi atribuído lugar de ser histórico e sujeito de direitos ao longo do tempo, o que surtiu e vem surtindo efeitos práticos em diversas áreas, contudo, não totalmente no âmbito da sexualidade. Debates revelam que, sob uma perspectiva dos direitos humanos, os direitos sexuais da criança continuam marcados pela ideia de proteção tutelar, dominação e excepcionalidade. Isso significa que seus direitos sexuais não são reconhecidos como direitos de fato, o que contribui para a manutenção do status ``castrador e adultocêntrico'' das discussões e intervenções públicas no campo da sexualidade. Para \citeonline[p. 74]{NETO2009}:

\begin{citacao}
	A garantia dos direitos sexuais de crianças e adolescentes deve ser considerada como uma proteção a seu direito à vida, competindo aos Estados partes assegurarem ao máximo a `sobrevivência e o desenvolvimento da criança' (CDC\footnotemark, Artigo 6, 1-2) e adotarem medidas apropriadas para `protegê-las contra todas as formas de abuso e exploração sexual' (CDC, Artigo 34, 1).
\end{citacao}

\footnotetext{Convenção das Nações Unidas sobre os Direitos da Criança.}

Discussões no sentido de se tentar garantir os direitos sexuais das crianças e adolescentes passam pela tentativa de superar a visão e a invocação da ``inocência da criança'', que têm contribuído para manter sua proteção tutelar como instrumento de intervenção  estatal reificadora e castradora \apud[p. 74]{ENNEW1986}{NETO2009}.

O abuso sexual infantil refere-se a práticas que ferem a dignidade e os direitos da criança, principalmente no tocante à sua sexualidade. Para o Ministério da Saúde, violência sexual é ``toda ação na qual uma pessoa, em situação de poder, obriga outra à realização de práticas sexuais contra a vontade, através da força física, da influência psicológica ou do uso de armas ou drogas''. Segundo a Organização Mundial da Saúde, o abuso sexual infantil acontece quando há ``o envolvimento de uma criança em atividade sexual que ele ou ela não compreende completamente, é incapaz de consentir, ou para a qual, em função de seu desenvolvimento [\ldots] não está preparada e não pode consentir'' (OMS 1999). Além disso, a situação de abuso sexual infantil envolve a violação às leis ou tabus da sociedade. 

Atualmente, o abuso sexual infantil é avaliado no Código Penal \cite{BRASIL1940}, no Estatuto da Criança e do Adolescente \cite{BRASIL1990}, dentre outros (vide anexo). O ECA dispõe que a criança e o adolescente gozam de ``todos os direitos fundamentais inerentes à pessoa humana''  (art. 3), sendo que nenhuma delas ``será objeto de qualquer forma de negligência, discriminação, exploração, violência, crueldade e opressão'' (art. 5), estando sujeito à punição o descumprimento desses preceitos.

Para \apudonline{HOLMES1997}{TRINDADEBREIER2007}, o pano de fundo para o abuso sexual infantil é a desproporção de poder e conhecimento que existe entre adulto e criança. A criança espera de um adulto proteção, cuidado e condições mínimas para seu desenvolvimento, seja físico ou psicológico, o que caracteriza a infância como etapa peculiar do ciclo vital em que necessidades temporárias se fazem presentes \cite{TRINDADEBREIER2007}, não podendo essas, no entanto, se tornar fatores que justifiquem ou legitimem qualquer forma de abuso por parte dos mais velhos.

O \citeonline{MJ1997} aponta que, devido à sua complexidade, a compreensão do abuso sexual de crianças precisa se dar através da análise de todo um contexto histórico, cultural, jurídico, econômico e político. A sociedade brasileira é marcada por desigualdades constituídas não apenas pela dominação de classes, mas também de gênero e raça, e é marcada por um autoritarismo presente nas relações entre adulto e criança. Para \citeonline[p. 198]{PFEIFFERSALVAGNI2005}: 

\begin{citacao}
	Em todos os tempos, o domínio do mais forte sobre o mais fraco foi exercido sob as diversas formas de poder, nas diferentes esferas da sociedade, desde as políticas estatais, às sociais e familiares. A essa relação de poder, de busca dos excessos, do diferente e até mesmo do anormal, soma-se a pouca importância dada às crianças e aos adolescentes e às consequências dos maus-tratos dos adultos sobre eles. Dessa forma, mesmo com a evolução dos princípios morais e legais em defesa das crianças e adolescentes, os casos de abuso sexual não deixaram de acontecer, nem passaram a ser vistos de maneira uniforme pela sociedade como um crime que deixa sequelas, muitas vezes irreparáveis.
\end{citacao}

A problematização do abuso sexual infantil está cada vez mais presente em jornais e pesquisas acadêmicas, o que não o torna um tema esgotado, tampouco isento de desatenção e da necessidade de maior aprofundamento. Além disso, o abuso sexual ainda é repleto de mitos e situações que ao invés de o desvelarem   apenas reforçam a discriminação e sua naturalização. Aguns desses mitos são: a crença de que o abuso sexual infantil ocorre apenas em classes mais pobres ou somente em meninas, ou, ainda, que todos que o praticam são pedófilos e que vítimas de abuso sexual são violadores sexuais em potencial \cite[p. 177]{ALENCAR2009}. \citeonline[p. 80]{CAMPOS2002} destaca mais alguns, tais como a ideia de que o ``abusador é um psicopata'' com características facilmente reconhecíveis ou que o abuso sexual é, normalmente, uma ação isolada, ``contemplada num único ato violento e envolvendo somente a conjunção carnal'' (p. 81).

O abuso sexual pode ser classificado como verbal, que é aquele em que o contato físico não ocorre, e como abuso sexual com contato físico. O primeiro tipo abrange práticas como o exibicionismo, o assédio sexual, telefonemas obscenos, voyeurismo e manuseio de material pornográfico infantil. O segundo tipo pode se dar por atentado violento ao pudor, que consiste em forçar a criança a praticar algum ato, por estupro ou violação, por incesto, por prostituição infantil dentre outros \cite{VIEIRA2006}.

Opiniões quanto à capacidade de a criança se defender em situações de iminência do abuso sexual são divergentes. Alguns acreditam que por aprenderem desde cedo o que é socialmente aceitável com relação a seus genitais e de outrem, as crianças podem ser capazes de perceber as ações de risco \cite{BRINOWILLIAMS2008}. Dessa forma, é de extrema importância considerar que se instituições como escola e família fizerem seu papel nessa tarefa de instrução e orientação, a possibilidade de a criança poder discernir o perigo é maior. \citeonline[n.p.]{BRINOWILLIAMS2008} afirmam que ``o abuso sexual pode ser prevenido se as crianças forem capazes de reconhecer o comportamento inapropriado do adulto, reagir rapidamente, deixar a situação e relatar para alguém o ocorrido''.  

Em se tratando de instituições escolares, seus profissionais possuem outra importante tarefa diante da suspeita de abuso: a identificação e notificação das ocorrências. O grande contato e o vínculo a que as crianças estão submetidas com seus educadores permitem o fácil reconhecimento de sinais de violência que elas possam estar sofrendo. Para tanto, é fundamental que os profissionais da educação estejam familiarizados com o assunto do abuso sexual para que mais facilmente reconheçam e contribuam na interrupção do ciclo de violência \cite{CHILDHOODBRASIL2009}.

Os efeitos do abuso sexual podem atingir muitos aspectos da vida da vítima, tais como ``psicológicos, físicos, comportamentais, acadêmicos, sexuais e interpessoais'' \apud[p. 70]{DAY2003}{HABIGZANG2012}.  Para \citeonline{PFEIFFERSALVAGNI2005} é indiscutível seu grande impacto na saúde física e/ou mental da criança, com marcas no desenvolvimento e danos que podem perdurar por toda a vida. Para \apudonline{SANCHEZ1997}{TRINDADEBREIER2007} e \apudonline{MATTOS2002}{TRINDADEBREIER2007}, as consequências traumáticas decorrentes do abuso sexual podem vir a curto ou a longo prazo e se relacionam a múltiplos fatores, tais como:

\begin{citacao}
	idade da criança na época do abuso sexual; duração e frequência; grau de violência ou ameaça; diferença de idade entre a pessoa que cometeu o abuso e a vítima; proximidade da relação entre abusador e vítima; ausência de figuras parentais protetoras e o grau de segredo e de ameaças contra a criança; reação dos outros; dissolução da família depois da revelação; criança se responsabilizando pela interação sexual; perpetrador negando que o abuso aconteceu; considerados agravantes para o desenvolvimento de reações negativas à experiência de abuso sexual (\citeauthor{FURNISS1993}, \citeyear{FURNISS1993}, \citeauthor{KAPLANSADOCKGREBB1997}, \citeyear{KAPLANSADOCKGREBB1997}, \citeauthor{SANDERSON2005}, \citeyear{SANDERSON2005} apud \citeauthor{HABIGZANG2012}, \citeyear{HABIGZANG2012}, p. 70).
\end{citacao}

Embora existam argumentos a respeito da influência direta do abuso sexual no comportamento da criança, tais como ``a apresentação de condutas sexualizadas'', ``sentimentos de culpa, fracasso ou dificuldades escolares'' dentre muitas outras, \apudonline[p. 79]{SANCHEZ1997}{TRINDADEBREIER2007} afirma que nem sempre se pode estabelecer relação direta de causa-efeito entre a ocorrência do abuso e comportamentos posteriores da criança. Ainda assim, é possível e necessário se atentar para alguns indícios que a criança venha apresentar, decorrentes de uma ou mais situações de abuso. Para \apudonline{IPPOLITO2003}{JUNG2006}, as crianças dão aviso de que estão sendo vítimas de abuso sexual de diversas formas, em sua maioria não-verbais. Salienta-se que não se deve considerar apenas um sinal isoladamente, mas por meio de um cruzamento com outros dados.

A criança vítima de abuso sexual pode apresentar apenas sintomas psicológicos e, além disso, o abuso nem sempre ocorre de forma agressiva ou violenta, o que inclui atitudes de leves toques, beijos, promessas de presente e de atenção, o que pode contribuir para o surgimento de um sentimento dúbio na criança e a impressão de que ela possa ter consentido com o ato \cite{RAMOS2009}. 

Para \apudonline{KAPLANSADOCKGREBB1997}{HABIGZANG2012}, não se associa diretamente nenhum sintoma psiquiátrico específico decorrente do abuso sexual, no entanto há maior risco de crianças que sofreram abuso sexual desenvolverem problemas interpessoais e psicológicos, a curto ou a longo prazo, em comparação com outras crianças da mesma idade que não tenham passado por isso. Em uma análise de \apudonline{CARMENMILLS}{JUNG2006}, constatou-se que 43\% dos pacientes psiquiátricos por ele observados apresentaram história anterior de abuso sexual na infância. Portanto, é pertinente se atentar para a possibilidade de correlação existente entre o abuso e tais consequências.

As consequências físicas do abuso sexual variam de  pequenas cicatrizes, traumas físicos na região genital, doenças sexualmente transmissíveis, até danos cerebrais permanentes e morte \apud{FERREIRASCHARAMM2000}{HABIGZANG2012}. Outras consequências físicas podem ser os distúrbios de sono, da alimentação, baixo controle dos esfíncteres e até aumento ou perda de peso afim de parecer menos atraente aos olhos do abusador (\citeauthor{IPPOLITO2003}, \citeyear{IPPOLITO2003}, \citeauthor{SANTOS1991}, \citeyear{SANTOS1991}, \citeauthor{VITIELLO1989}, \citeyear{VITIELLO1989} apud \citeauthor{JUNG2006}, \citeyear{JUNG2006}).

Os efeitos psicológicos vão desde a ``baixa autoestima até desordens psíquicas severas'' \apud[p. 71]{FERREIRASCHARAMM2000}{HABIGZANG2012}, tais como depressão, sentimento de culpa e de vergonha, ansiedade social, transtorno do pânico, distúrbios de conduta, transtorno dissociativo, transtorno de estresse pós-traumático (TEPT) etc \apud{HETZELRIGGINBRAUSCHMONTGOMERY2007}{HABIGZANG2012}. Podem também ser observados sintomas de ``déficit de atenção, hipervigilância e distúrbios de aprendizado'' \apud[p. 71]{SANDERSON2005}{HABIGZANG2012}. É importante destacar que, a longo prazo, muitos desses efeitos podem levar a vítima a possuir ideias de suicídio, à tentativa ou à própria consumação do ato \apud{SANCHEZ1991}{JUNG2006}.

No âmbito afetivo, conforme \citeonline{JUNG2006} afirma, as crianças vítimas de abuso sexual costumam experienciar sentimentos de auto-desvalorização, culpa e depressão. Na esfera sexual, uma das principais afetadas pelo abuso sexual, os efeitos se dão em forma de ``medo da intimidade'', se manifestando pela recusa de qualquer relacionamento sexual ou dificuldade em manter relacionamento sexual satisfatório \apud{AZEVEDO1989}{JUNG2006}. 

Sejam notificadas ou não, as situações de abuso sexual infantil ocorrem cotidianamente, a despeito da idade, do sexo, da raça e dos níveis social, cultural e econômico da criança. Atualmente se tem contado com a mídia na denúncia e notícia a respeito de tais casos. Embora se tenha noção da importância do papel que ela exerce, é importante se atentar para a forma como isso vem sendo efetivado, muitas vezes sensacionalista, causando assim ``revitimização'' e danos secundários às vítimas desse tipo de violência \cite[p. 95]{KUNG2009}. 

Diante da suspeita do abuso sexual contra crianças, todos têm como dever a denúncia às autoridades policiais, que daí em diante serão responsáveis por investigar o caso. A notícia de um abuso sexual deve ser sempre investigada, o que significa considerar que se precisa cumprir uma série de protocolos que vão desde a necessidade de a vítima se submeter a exames médicos para obtenção de evidências até entrevistas com profissionais psicólogos, assistentes sociais, do Conselho Tutelar, da Polícia, do Ministério Público e da Justiça. Ainda, em casos mais graves, pode se fazer necessário o afastamento da criança de seu convívio familiar devido à impossibilidade de permanência próxima ao abusador. Depreende-se então que:

\begin{citacao}
	\ldots as consequências do abuso sexual contra a criança se estendem para além dos efeitos do abuso em si, conduzindo a variadas experiências estressoras capazes de provocar uma segunda vitimização. Por isso deve-se procurar reduzir a necessidade de múltiplas entrevistas, diminuir as formalidades legais e minorar a frieza dos ambientes por onde a criança precisará transitar, bem como disponibilizar um quadro de profissionais -- tanto pelo lado do direito quanto do serviço social e de saúde física e mental -- especialmente treinado e preparado para acolher a criança e evitar a sua revitimização \cite[p. 81]{TRINDADEBREIER2007}.
\end{citacao}

Reforça-se então que, diante da possível revitimização a que a criança está exposta quando é submetida a procedimentos interrogativos que verifiquem a ocorrência do abuso sexual, é de fundamental importância que profissionais que atuam diretamente com ela estejam alertas para o cuidado em suas intervenções \apud{FURNISS1993}{JUNG2006}.

Entre os psicólogos, há que se preocupar com o cuidado no questionamento dos fatos à criança, de forma que não se façam perguntas diretas sobre o abuso em si, mas questões que permitam que ela relate como está se sentindo; ou que sejam utilizados métodos lúdicos através dos quais ela poderá se expressar, testes verbais que poderão dar informação do abuso de maneira simbólica etc \cite{JUNG2006}, até visitas domiciliares, quando necessário. De acordo com \apudonline[p. 42]{MILLER1987}{JUNG2006}, ``as crianças abusadas sexualmente precisam de meios apropriados para expressar sua raiva, medo, hostilidade e outros sentimentos que possam estar inibidos ou reprimidos''. 

Apesar de todo o avanço obtido na história concernente ao lugar de criança, ela ainda é, com frequência, considerada como inferior, inábil e dependente, tendo por referência o adulto'' \cite[p. 6]{JUNG2006}. É fato que essa relação de referência e de autoridade está posta e socialmente aceita. No entanto, o aspecto de adultocentrismo valida historicamente os adultos a exercitarem seu poder sobre os mais novos num sentido assimétrico ou desigual de poder, com atitudes danosas e desregradas. 

Outro aspecto que permite analisar as relações de poder existentes em nossa sociedade é o machismo, crença de que o homem se encontra em posição superior à mulher. Considerando-o, é possível refletir sobre o porquê de ser mais frequente a vitimização de crianças do sexo feminino em situações de abuso sexual.

De forma geral, as causas do abuso sexual infantil não são precisamente descritas. Na verdade, elas compõem um conjunto de fatores sociais, econômicos, culturais, psicológicos e situacionais \cite{JUNG2006}. Diante disso, cabe a profissionais o constante estudo, além do tratamento, tanto às vítimas quanto aos abusadores, de maneira que suas ações possam contribuir para diversas mudanças tanto na esfera relacional da vítima e do abusador quanto na própria estrutura social.