% ==================================
% Considerações Finais
% ==================================

\chapter*[Considerações Finais]{Considerações Finais}
\addcontentsline{toc}{chapter}{Considerações Finais}

O objetivo do presente trabalho foi realizar um breve estudo sobre a maneira como a Psicologia Jurídica brasileira vem atuando frente aos casos de abuso sexual infantil bem como sobre alguns limites e possibilidades com os quais esse campo pode se deparar.

Para tanto foi fundamental a anterior compreensão sobre alguns elementos. O primeiro deles envolveu saber como a criança moderna se constituiu e que fatores engendraram mudanças no modo de ver e de considerá-la ao longo dos séculos. Abrangeu-se também a constituição do interesse pela criança no Brasil que, embora tenha sofrido influência do contexto europeu, teve suas próprias características. 

É importante ressaltar sobre as mudanças na forma de tratar a criança, decorrentes do processo de cristianização dos costumes e, ainda, o seu afastamento de questões sexuais -- inclusive, e principalmente, das próprias práticas sexuais a que eram submetidas na sociedade europeia --, o que, ao longo dos séculos e relacionados a outros fatores, se traduziu num contexto repleto de contradições, tabus e controvérsias, com desdobramentos cruéis, sobretudo à criança.

Quanto ao caminho histórico percorrido no primeiro capítulo a fim de contextualizar e demarcar o lugar que a infância ocupa nos tempos modernos, ressalta-se que sua complexidade é bem maior, tendo sido exposto aqui somente um resumo sobre os fatos e não o aprofundamento deles. Uma sistematização e um detalhamento maiores permitiriam realizar questionamentos e reflexões mais críticas diante dos acontecimentos históricos. No entanto, objetivava-se apenas uma noção das mudanças pelas quais a concepção de infância e/ou a própria criança sofreu.

Com relação à infância no Brasil, focou-se na questão do contexto jurídico e do tratamento da criança nesse contexto a fim de ressaltar a diferença social na sociedade brasileira que marginaliza e explora crianças, conferindo-lhes e reforçando lugares que não lhes cabem como ser em desenvolvimento, detentor de direitos humanos e proteção garantida por lei. A exposição realizada permitirá um profundo estudo futuro sobre os aspectos políticos, sociais, econômicos e culturais que envolvem essa realidade do “menor” -- a criança marginalizada no Brasil --, objetivando repensar as leis voltadas à criança e como podia ser dar uma mudança efetiva em nossa estrutura social.

Referente à sexualidade infantil, buscou-se a contribuição das teorias de Freud e de Foucault, embora elas estejam envoltas de debates e controvérsias, por terem colaborado para a ideia de que a sexualidade é inerente aos indivíduos, portanto, separar os dois é cindir com o próprio sujeito, dando vazão ao exercício do domínio externo sobre ele de maneira funesta.

O segundo aspecto que foi necessário compreender se refere ao abuso sexual infantil. Por meio do estudo realizado foi possível depreender que ele não se constitui em um fato isolado tampouco decorre de um fator apenas, mas se relaciona aos processos sociais ao longo da História como um todo, chegando a atingir níveis consideráveis de ocorrência, desprezando os direitos da criança e contribuindo para que se perpetue um estigma de subjulgo da mesma ao poder do adulto. Foi possível refletir ainda sobre a força com que esse fenômeno se mantém, não conseguindo as punições previstas em lei conter a atuação dos abusadores. Diante disso, cabe analisar de forma mais profunda os aspectos que engendram essa perpetuação da violência infantil.
É importante ressaltar que no que concerne às consequências do abuso priorizou-se por não apresentar uma visão determinista, embora se tenha clara a noção de que as consequências para a vida da criança abusada inegavelmente existem e lhe sobrevêm. Além disso, é necessário pontuar que não foram abrangidas todas as consequências do abuso sexual infantil.
Uma provável discussão a ser feita se refere à possibilidade de, mesmo diante de uma denúncia, o abuso sexual não ter ocorrido; são os casos de falsa denúncia do abuso sexual infantil. A literatura já traz algumas discussões nesse sentido, que podem ser interessantes para profissionais que desejam atuar diretamente com a denúncia desse tipo de violência infantil.
De maneira geral, observou-se que é extensa a literatura sobre o abuso sexual infantil, o que se constituiu em uma importante contribuição para profissionais e para a própria população como um todo que deseje conhecer ou se aprofundar no tema. Em contrapartida, verificou-se que a bibliografia brasileira que trata especificamente do abusador -- embora ele não tenha sido o foco do capítulo deste trabalho -- é escassa, sendo, portanto, necessário que mais pesquisas nessa área sejam realizadas.
Por fim, o terceiro ponto  refere-se às práticas, limites e possibilidades atuais da Psicologia Jurídica brasileira frente ao abuso sexual infantil. Antes de qualquer consideração, há que se ressaltar sobre a carência de estudos aprofundados sobre o histórico da Psicologia Jurídica, sobretudo a brasileira. 
Observou-se que existem divergências de opiniões quanto a consolidação das práticas do psicólogo jurídico, o que indica a necessidade de se configurar melhor o lugar desse profissional nas várias áreas dentro do Sistema. Notou-se ainda frequentes questionamentos a respeito da carência de preparo durante a graduação para a atuação do psicólogo no âmbito jurídico. Raros são os cursos de graduação em Psicologia que oferecem a disciplina de Psicologia Jurídica, seja de forma obrigatória ou optativa (ROVINSKI, 2009).
Sobre a atuação específica do psicólogo jurídico em casos de abuso sexual infantil que chegam ao Sistema Judiciário, percebeu-se que se encontra de maneira quase completa permeada pelo exercício da perícia psicológica ou estudo psicossocial, portanto, constituída das práticas de avaliações psicológicas e psicodiagnósticos. Há que se perceber a contribuição que essa atuação tem trazido, no sentido de poder colaborar para a proteção da vítima e para seu desenvolvimento psicossocial. 
 
É importante ressaltar que se faz extremamente necessária a qualificação profissional específica dos psicólogos jurídicos em sua área de interesse bem como a busca de conhecimento sobre o próprio sistema em que atua: a Justiça. Além da qualificação específica, é absolutamente fundamental o desenvolvimento da escuta diferenciada, caso se pretenda ir além de uma simples perícia, percebendo e acolhendo o indivíduo em sua totalidade, possibilitando, a partir dos contatos periciais, a mudança de significados e a autonomia deste. Essa forma de atuar é um desafio atual para a Psicologia Jurídica em casos de abuso sexual infantil. Isso porque o que é mais comum é a simples conclusão de perícias psicológicas, materializadas por meio do Laudo Pericial, após as quais não se observa mudanças na vida dos envolvidos. Constitui-se em um desafio por ir contra o que está posto e estabelecido ao próprio cargo de psicólogo jurídico como perito. É claro que aqui não se fala de se desconsiderar o âmbito de trabalho ou de fugir às atribuições do próprio cargo -- inclusive porque isso se estabeleceria como irresponsabilidade profissional --, mas se trata de ir além de barreiras muitas vezes impostas não pelo sistema judiciário, mas pela própria categoria profissional; trata-se também de vencer certo conformismo e pessimismo presente na sociedade quanto à (in)eficiência de qualquer atuação no Jurídico e, por fim, trata-se de relembrar o aspecto e a condição humana que vêm junto com a vítima no momento da perícia.

Referente à perícia psicológica de crianças abusadas sexualmente existe uma gama de instrumental psicológico específico, além de entrevista clínica estruturada, que aqui não se abordou de forma minuciosa. Seria interessante a produção de mais estudos abrangendo a especificidade e a descrição desse material a fim de auxiliar no conhecimento de psicólogos jurídicos que estejam iniciando sua prática nesse âmbito.

Um ponto importante que precisa de atenção é a forma como se dá a atuação do psicólogo jurídico, assim como de qualquer profissional nessa área, tendo em vista que a criança possivelmente passou ou passará por uma série de procedimentos de verificação do abuso sexual. Muitas vezes esses procedimentos sequer possuem conexão entre si, só reforçando ainda mais o sofrimento e a revitimização. Há casos em que o tempo entre a ocorrência do abuso sexual e sua denúncia é extenso, portanto, retomar todos os fatos e trazer à tona os sentimentos da criança pode ser mais árduo ainda para ela, daí a importância de um trabalho integrado, associado e minimizador do sofrimento e da revitimização.

É importante mencionar que não foi abordado nesse capítulo o chamado \emph{Depoimento sem Dano}, técnica recente que objetiva a tomada especial do depoimento da criança vítima de abuso sexual infantil. Apenas salienta-se que ao psicólogo atualmente essa prática está vedada, conforme decisão do Conselho Federal de Psicologia. Debates sobre esse tema foram e ainda vêm sendo realizados e as controvérsias são grandes. Em virtude dessa suspensão da atuação do psicólogo e da considerável discordância de opiniões, priorizou-se por não torná-lo ponto de discussão aqui. No entanto, isso não significa que as discussões devam cessar, pelo contrário, é preciso que se chegue num argumento definitivo sobre a contribuição ou não do psicólogo jurídico nessa técnica, em benefício final da criança.

De tudo, depreende-se que a Psicologia Jurídica ainda tem muito a contribuir e a desenvolver em sua atuação como um todo e em casos específicos de abuso sexual infantil. O profissional desse contexto, não apenas psicólogo, não pode se esquecer de que o cliente que lhe chega não está isolado, mas inserido em um contexto complexo, em uma rede de relações sociais sujeitas a todo instante a alterações; é um ser particular, com sua individualidade, detentor de vontades e também de direitos e permeado por valores e crenças. Sendo assim, é possível pensar em medidas de reorganização, atribuição de novos significados à experiência, prevenção e outros conjuntos de ações que possam contribuir para restabelecer não somente o sentido à criança, mas à relação familiar e até as próprias relações entre adultos e crianças.